\append{Computations with the 2D DFT}


In the present Appendix, we deduce equation \ref{eq:DFT-system}
by using the row-ordered $vec$-operator (here designated simply as $vec$-operator).
This equation can be efficiently computed by using the 2D 
fast Fourier Transform. 
This operator was implicitly used by \citet[][ p. 31]{jain1989} to 
show the relationship between Kronecker products and separable 
transformations. The $vec$-operator defined here 
transforms a matrix into a column vector by stacking its rows. 

Let $\mathbf{M}$ be an arbitrary $N \times M$ matrix given by:
\begin{equation}
\mathbf{M} = \begin{bmatrix}
\mathbf{m}^{\top}_{1} \\ 
\vdots \\
\mathbf{m}^{\top}_{N}
\end{bmatrix} \: ,
\label{eq:matrix-M}
\end{equation}
where $\mathbf{m}_{i}$, $i = 1, \dots, N$, are $M \times 1$ vectors containing 
the rows of $\mathbf{M}$.
The elements of this matrix can be rearranged into a column vector by using the
$vec$-operator \citep[][ p. 31]{jain1989} as follows:
\begin{equation}
vec \left( \mathbf{M} \right) = \begin{bmatrix}
\mathbf{m}_{1} \\
\vdots \\
\mathbf{m}_{N}
\end{bmatrix}_{NM \times 1} \: .
\label{eq:vec-operator}
\end{equation}
This rearrangement is known as lexicographic ordering \citep[][ p. 150]{jain1989}.

Two important properties of the $vec$-operator (equation \ref{eq:vec-operator}) 
are necessary to us. 
To define the first one, consider an 
$N \times M$ matrix $\mathbf{H}$ given by
\begin{equation}
\mathbf{H} = \mathbf{P} \circ \mathbf{Q} \: ,
\label{eq:matrix-H}
\end{equation}
where $\mathbf{P}$ and $\mathbf{Q}$ are arbitrary $N \times M$ matrices and 
``$\circ$" represents the Hadamard product \citep[][ p. 298]{horn_johnson1991}.
By applying the $vec$-operator to $\mathbf{H}$ (equation \ref{eq:matrix-H}), 
it can be shown that
\begin{equation}
vec \left( \mathbf{H} \right) = 
vec \left( \mathbf{P} \right) \circ vec \left( \mathbf{Q} \right) \: .
\label{eq:vec-matrix-H}
\end{equation}
To define the second important property of $vec$-operator, 
consider an $N \times M$ matrix $\mathbf{S}$ defined by 
the separable transformation \citet[][ p. 31]{jain1989}:
\begin{equation}
\mathbf{S} = \mathbf{P \, M \, Q} \: ,
\label{eq:matrix-S}
\end{equation}
where $\mathbf{P}$ and $\mathbf{Q}$ are arbitrary $N \times N$ and $M \times M$ 
matrices, respectively.
By implicitly applying the $vec$-operator to 
the $\mathbf{S}$ (equation \ref{eq:matrix-S}), 
\citet[][ p. 31]{jain1989} show that:
\begin{equation}
vec \left( \mathbf{S} \right) = 
\left( \mathbf{P} \otimes \mathbf{Q}^{\top} \right) 
vec \left( \mathbf{M} \right) \: ,
\label{eq:vec-matrix-S}
\end{equation}
where ``$\otimes$" denotes the Kronecker product \citep{neudecker1969}.
It is important to stress the difference between equation \ref{eq:vec-matrix-S}
and that presented by \citet{neudecker1969}, which is more commonly found in 
the literature.
While that equation uses a $vec$-operator that transforms a matrix into a column 
vector by stacking its columns, equation \ref{eq:vec-matrix-S} 
uses the $vec$-operator defined by equation \ref{eq:vec-operator}, which 
transforms a matrix into a column vector by stacking its rows.

Now, let us deduce equation \ref{eq:DFT-system} by 
using the above-defined properties (equation \ref{eq:vec-matrix-H}
and \ref{eq:vec-matrix-S}).
We start calling attention to the right side of equation \ref{eq:vec-DFT-system}.
Consider that vector $\mathbf{w}$ (equation \ref{eq:vec-DFT-system}) 
is obtained by applying the $vec$-operator (equation \ref{eq:vec-operator}) to a matrix 
$\mathbf{W}$, whose 2D DFT $\tilde{\mathbf{W}}$ is represented by the 
following separable transformation \citep[][ p. 146]{jain1989}:
\begin{equation}
\tilde{\mathbf{W}} = \mathbf{F}_{2Q} \, \mathbf{W} \, \mathbf{F}_{2P} \: ,
\label{eq:2D-DFT-W}
\end{equation}
where $\mathbf{F}_{2Q}$ and $\mathbf{F}_{2P}$ are the $2Q \times 2Q$ and $2P \times 2P$ 
unitary DFT matrices. 
Using equation \ref{eq:vec-matrix-S} and the symmetry of unitary DFT 
matrices, we rewrite the right side of equation \ref{eq:vec-DFT-system} 
as follows:
\begin{equation}
vec \left( \tilde{\mathbf{W}} \right) = 
\left( \mathbf{F}_{2Q} \otimes \mathbf{F}_{2P} \right) 
vec \left( \mathbf{W} \right) \: .
\label{eq:right_side_DFT_system_1}
\end{equation}
Similarly, consider that $\mathbf{v}$ (equation \ref{eq:vec-DFT-system}) 
is obtained by applying the $vec$-operator (equation \ref{eq:vec-operator}) to a matrix 
$\mathbf{V}$, whose 2D DFT (equation \ref{eq:2D-DFT-W}) is 
represented by $\tilde{\mathbf{V}}$. Using equation \ref{eq:vec-matrix-S} and the symmetry 
of unitary DFT matrices, we can rewrite the 
left side of equation \ref{eq:vec-DFT-system} as follows:
\begin{equation}
\boldsymbol{\Lambda} \, vec \left( \tilde{\mathbf{V}} \right) = 
\boldsymbol{\Lambda}
\left( \mathbf{F}_{2Q} \otimes \mathbf{F}_{2P} \right) 
vec \left( \mathbf{V} \right) \: .
\label{eq:left_side_DFT_system_1}
\end{equation}
Note that both sides of equation \ref{eq:left_side_DFT_system_1}
are defined as the product of the diagonal matrix $\boldsymbol{\Lambda}$ (equation \ref{eq:C-diagonalized}) 
and a vector. In this case, the matrix-vector product can be conveniently replaced by
\begin{equation}
\boldsymbol{\lambda} \circ vec \left( \tilde{\mathbf{V}} \right) = 
\boldsymbol{\lambda} \circ
\left( \mathbf{F}_{2Q} \otimes \mathbf{F}_{2P} \right) 
vec \left( \mathbf{V} \right) \: ,
\label{eq:left_side_DFT_system_2}
\end{equation}
where $\boldsymbol{\lambda}$ is a $4QP \times 1$ vector containing the diagonal of 
$\boldsymbol{\Lambda}$ (equation \ref{eq:C-diagonalized}).
Then, consider that $\boldsymbol{\lambda}$ is obtained by applying the $vec$-operator 
(equation \ref{eq:vec-operator}) to a $2Q \times 2P$ matrix $\mathbf{L}$, we can use 
equations \ref{eq:vec-matrix-H} and \ref{eq:vec-matrix-S} to rewrite equation 
\ref{eq:left_side_DFT_system_2} as follows:
\begin{equation}
vec \left( \mathbf{L} \circ \tilde{\mathbf{V}} \right) = 
vec \left[ \mathbf{L} \circ 
\left( \mathbf{F}_{2Q} \, \mathbf{V} \, \mathbf{F}_{2P} \right) 
\right] \: .
\label{eq:left_side_DFT_system_3}
\end{equation}
Equations \ref{eq:2D-DFT-W}, \ref{eq:right_side_DFT_system_1} and 
\ref{eq:left_side_DFT_system_3} show that equation \ref{eq:vec-DFT-system}
is obtained by applying the $vec$-operator to 
\begin{equation}
\mathbf{L} \circ \left( \mathbf{F}_{2Q} \, \mathbf{V} \, \mathbf{F}_{2P} \right) = 
\mathbf{F}_{2Q} \, \mathbf{W} \, \mathbf{F}_{2P} \: .
\label{eq:DFT-system-preliminary}
\end{equation}
Finally, we premultiply both sides of equation \ref{eq:DFT-system-preliminary} by 
$\mathbf{F}_{2Q}^{\ast}$ and then postmultiply both sides of the result by 
$\mathbf{F}_{2P}^{\ast}$ to deduce equation \ref{eq:DFT-system}.