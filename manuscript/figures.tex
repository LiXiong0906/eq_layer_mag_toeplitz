\renewcommand{\figdir}{Fig} % figure directory

%% Methodology
%\plot{Figure1}{width=\textwidth}{
%	{Schematic representation of an $N_{x} \times N_{y}$ regular grid of points (black dots) defined by 
%	$N_{x} = 4$ and $N_{y} = 3$. The grids are oriented along the (a) $x$-axis and (b) $y$-axis. The grid 
%	coordinates are $x_{k}$ and $y_{l}$, where the $k = 1, \dots, N_{x}$ and $l = 1, \dots, N_{y}$ are 
%	called the grid indices. The insets show the grid indices $k$ and $l$.}
%	\label{fig:methodology}
%}

\plot{schematic_regular_grids}{width=\textwidth}{
	{Schematic representation of an $N_{x} \times N_{y}$ regular grid of points (black dots) with
		$N_{x} = 3$ and $N_{y} = 2$, where each point has an associated index. This index may represent
		$i$ or $j$, that are associated with observation points $(x_{i}, y_{i}, z_{0})$ and 
		equivalent sources $(x_{j}, y_{j}, z_{c})$. Left panel shows an example of $x$-oriented grid,
		with indices varying along $x$-axis, while right panel shows an example of $y$-oriented grid, 
		with indices varying along $y$-axis.}
	\label{fig:regular-grids}
}

\plot{4_equivalent_sources}{width=\textwidth}{
	{Representation of the four equivalent sources (black dots) needed to reconstruct the non-symmetric matrix $\mathbf{A}$ (equation \ref{eq:aij_mag}). Each of the equivalent sources are located in the corner of the simulated regular grid of $M_x = 4$ and $M_y = 3$ sources. The influence of these sources on each of the observation points (blue dots) i the regular grid of $N_x = 4$ and $N_y = 3$ will give the four columns necessary of matrix $\mathbf{A}$.}
	\label{fig:4_equivalent_sources}
}

%% Computational performance

\plot{flops_mag}{width=\textwidth}{
	{Number of flops necessary to estimate the parameter vector $\mathbf{\hat{p}}$ using the non-iterative classical method (equation \ref{eq:flops-normal-cholesky}) the CGLS (equation \ref{eq:flops-cgls}) and our modified CGLS method (equation \ref{eq:flops-cgls-bccb}) with $N^{it} = 50$. 
	The observation point $N$ varied from $5,000$ to $1,000,000$. The \emph{radix-2} 2D FFT
	algorithm was considered for our method, with $\kappa = 5$.}
	\label{fig:flops}
}

\plot{time_comparison_mag}{width=\textwidth}{
	{Comparison between the runtime of the equivalent-layer technique using the classical, the CGLS algorithm and our method. The values for the CGLS and our methods were obtained for $N^{it} = 50$ iterations.}
	\label{fig:solve_time}
}

\plot{time_sources_mag}{width=\textwidth}{
	{Comparison between the runtime to calculate the first column of the BCCB matrix embbeded from $\mathbf{A}$ (equation \ref{eq:aij_mag}) using only one and using four equivalent sources. Although the time is very similar, with one source a small advantage can be observed as the number of data $N$ increases. This test was done from $N = 10,000$ to $N = 700,000$ with increases of $5,625$ observation points.}
	\label{fig:sources_time}
}

%% Synthetic data part I

\plot{synthetic_data_comparison}{width=\textwidth}{
{Synthetic tests: the simulated airborne magnetic surveys  - 
The first column shows the grids of observation points or the undulating observation surfaces that simulate airborne magnetic surveys. 
The second column shows the noise-corrupted total-field anomaly produced by the synthetic sources and calculated on the simulate airborne magnetic survey shown in the first column. 
The third column shows the  noise-free total-field anomaly produced by the synthetic sources 
at $z = −1,300$ m.
The results shown in these three columns were obtained by using the simulated airborne magnetic surveys as follows: 
(a)-(c) A regular grid of $100 \times 50$ observation points in the $x-$ and $y-$directions and a flat observation surface at $900$ m height. 
(d)-(f) A irregular grid with uncertainty of $20\%$  in the $x-$ and $y-$coordinates and a flat observation surface at $900$ m height. 
(g)-(i) A irregular grid with uncertainty of $30\%$  in the $x-$ and $y-$coordinates and a flat observation surface at $900$ m height. 
(j)-(l) A regular grid  of $100 \times 50$ observation points in the $x-$ and $y-$coordinates and an undulating observation surface with uncertainty of $5\%$. 
(m)-(o) A regular grid  of $100 \times 50$ observation points in the $x-$ and $y-$coordinates and an undulating observation surface with uncertainty of $10\%$. 
The black lines represent the horizontal projection of the sources
.}
\label{fig:synthetic_data_comparison}
}

\plot{synthetic_residuals_convergence_comparison}{width=\textwidth}{
{Synthetic tests: the data residuals and convergence - 
The first column shows the data residuals using the classical method.
The second and third columns show, respectively, the data residuals and the convergence curves using the convolutional equivalent layer (our method).
The results shown in these three columns were obtained by using the simulated airborne magnetic surveys shown in Figure \ref{fig:synthetic_data_comparison}, i.e.:
(a)-(c) A regular grid of $100 \times 50$ observation points in the $x-$ and $y-$directions and a flat observation surface at $900$ m height. 
(d)-(f) A irregular grid with uncertainty of $20\%$  in the $x-$ and $y-$coordinates and a flat observation surface at $900$ m height. 
(g)-(i) A irregular grid with uncertainty of $30\%$  in the $x-$ and $y-$coordinates and a flat observation surface at $900$ m height. 
(j)-(l) A regular grid  of $100 \times 50$ observation points in the $x-$ and $y-$coordinates and an undulating observation surface with uncertainty of $5\%$. 
(m)-(o) A regular grid  of $100 \times 50$ observation points in the $x-$ and $y-$coordinates and an undulating observation surface with uncertainty of $10\%$. 
The black lines represent the horizontal projection of the sources
.}
\label{fig:synthetic_residuals_convergence_comparison}
}


\plot{synthetic_upward_residuals_comparison}{width=\textwidth}{
{Synthetic tests: the data residuals of the upward-continued total-field anomalies (second column 
of the Figure \ref{fig:synthetic_data_comparison}).
The data residuals of the upward-continued total-field anomalies are defined as the difference between 
the noise-free total-field anomaly produced by the synthetic sources at $z = −1,300$ m 
(third  column of the Figure \ref{fig:synthetic_data_comparison}) and
the predicted total-field anomaly at $z = −1,300$ m  obtained by using three methods:
the classical method (first column); the convolutional equivalent layer (second column); and 
the classic approach in the Fourier domain (third column).
The results shown in these three columns were obtained by using the simulated airborne magnetic surveys shown in Figure \ref{fig:synthetic_data_comparison}, i.e.:
(a)-(c) A regular grid of $100 \times 50$ observation points in the $x-$ and $y-$directions and a flat observation surface at $900$ m height. 
(d)-(f) A irregular grid with uncertainty of $20\%$  in the $x-$ and $y-$coordinates and a flat observation surface at $900$ m height. 
(g)-(i) A irregular grid with uncertainty of $30\%$  in the $x-$ and $y-$coordinates and a flat observation surface at $900$ m height. 
(j)-(l) A regular grid  of $100 \times 50$ observation points in the $x-$ and $y-$coordinates and an undulating observation surface with uncertainty of $5\%$. 
(m)-(o) A regular grid  of $100 \times 50$ observation points in the $x-$ and $y-$coordinates and an undulating observation surface with uncertainty of $10\%$. 
The black lines represent the horizontal projection of the sources
.}
\label{fig:synthetic_upward_residuals_comparison}
}

%% Field Data

\plot{carajas_real_data_mag}{width=\textwidth}{
	{Observed magnetic field data of the Carajás, Brazil area. The aeromagnetic survey was done with $131$ N-S lines at approximately $-900 m$ height, totaling $N = 6,081,345$ observation points.}
	\label{fig:carajas_real_data_mag}
}

\plot{carajas_real_data_decimated_gridline}{width=8cm}{
	{(a) Observed magnetic field data of the Carajás, Brazil area, interpolated for a regular grid of $10,000 \times 131$, totaling $N = 1,310,000$ observation points. (b) Observed magnetic field data of the Carajás, Brazil area, decimated from the flight lines resulting in an irregular grid of $10,000 \times 131$, also totaling $N = 1,310,000$ observation points.}
	\label{fig:carajas_real_data_decimated_gridline}
}

\plot{carajas_tf_predicted_gridline}{width=8cm}{
	{(a) Predicted data using our method for the interpolated $10,000 \times 131$ regular grid. (b) Residuals between the observed (\ref{fig:carajas_real_data_decimated_gridline}) and the predicted data (panel a), with mean of $0.07979$ nT and standart deviation of $0.5060$ nT.}
	\label{fig:carajas_gz_predito_mag_gridline}
}

\plot{carajas_tf_predicted_decimated}{width=8cm}{
	{(a) Predicted data using our method for the decimated $10,000 \times 131$ irregular grid. (b) Residuals between the observed (\ref{fig:carajas_real_data_decimated_gridline}) and the predicted data (panel b), with mean of $0.07348$ nT and standart deviation of $0.3172$ nT.}
	\label{fig:carajas_gz_predito_mag_decimated}
}

\plot{convergence_carajas_mag_decimated}{width=\textwidth}{
	{Convergence analysis of the CGLS method for the field data of Carajás, Brazil using the magnetic equivalent layer with a decimated irregular grid of $10,000 \times 131$ observation points up to 2,000 iterations.}
	\label{fig:convergence_carajas_mag_decimated}
}

\plot{up5000_carajas_mag_decimated}{width=\textwidth}{
	{Upward continuation transformation of real data of Carajás, Brazil at $5,000$ meter. It was necessary $2.64$ seconds to complete the process.}
	\label{fig:up5000_carajas_decimated_mag}
}