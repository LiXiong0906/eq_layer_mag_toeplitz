\append{BTTB matrix-vector product}

This appendix follows a similar approach to that presented by \citet{takahashi2020convolutional}
to efficiently compute the product of the sensitivity matrix
$\mathbf{A}$ (equations \ref{eq:A_expand}) and a generic vector $\mathbf{b}$. 
Let this product be represented by
\begin{equation}
	\mathbf{t} = 
	\mathbf{A} \, \mathbf{b} \: ,
	\label{eq:t}
\end{equation}
where 
\begin{equation}
	\mathbf{t} = \mathbf{t}_{\boldsymbol{xx}} + \mathbf{t}_{\boldsymbol{xy}} + \mathbf{t}_{\boldsymbol{xz}} +
	\mathbf{t}_{\boldsymbol{yy}} + \mathbf{t}_{\boldsymbol{yz}} + \mathbf{t}_{\boldsymbol{zz}}
	\label{eq:t-components}
\end{equation}
and
\begin{equation}
\mathbf{t}_{\boldsymbol{\alpha\beta}} = 
\mathbf{A_{\boldsymbol{\alpha\beta}}} \, \mathbf{b} \: .
\label{eq:t-alpha-beta}
\end{equation}
Let us also consider that vectors
\begin{equation}
\mathbf{t}_{\boldsymbol{\alpha\beta}} = \begin{bmatrix}
\mathbf{t}^{0}_{\boldsymbol{\alpha\beta}} \\
\vdots \\
\mathbf{t}^{Q-1}_{\boldsymbol{\alpha\beta}} \\
\end{bmatrix}_{N \times 1}
\label{eq:t-alpha-beta-partitioned}
\end{equation} 
and
\begin{equation}
\mathbf{b} = \begin{bmatrix}
\mathbf{b}^{0} \\
\vdots \\
\mathbf{b}^{Q-1} \\
\end{bmatrix}_{N \times 1}
\label{eq:b-partitioned}
\end{equation}
are composed of $P \times 1$ vectors $\mathbf{t}^{q}_{\boldsymbol{\alpha\beta}}$ and $\mathbf{b}^{q}$,
respectively, where $q$ is the block index (equations \ref{eq:q-x-oriented} and \ref{eq:q-y-oriented}). 
From equation \ref{eq:t-alpha-beta}, we obtain an auxiliary matrix-vector product given by
\begin{equation}
\mathbf{w}_{\boldsymbol{\alpha\beta}} = \mathbf{C}_{\boldsymbol{\alpha\beta}} \, \mathbf{v} \: ,
\label{eq:w_alpha_beta}
\end{equation}
where $\mathbf{C}_{\boldsymbol{\alpha\beta}}$ is a $4N \times 4N$ 
block circulant matrix with circulant blocks (BCCB) \citep[e.g., ][ p. 184]{davis1979},
\begin{equation}
\mathbf{w}_{\boldsymbol{\alpha\beta}} = \begin{bmatrix}
\mathbf{w}_{\boldsymbol{\alpha\beta}}^{0} \\
\vdots \\
\mathbf{w}_{\boldsymbol{\alpha\beta}}^{Q - 1} \\
\mathbf{0}_{2N \times 1}
\end{bmatrix}_{4N \times 1} \quad ,
\label{eq:w_alpha_beta_partitioned}
\end{equation}
\begin{equation}
\mathbf{w}_{\boldsymbol{\alpha\beta}}^{q} = \begin{bmatrix}
\mathbf{t}^{q}_{\boldsymbol{\alpha\beta}} \\
\mathbf{0}_{P \times 1}
\end{bmatrix}_{2P \times 1}
\label{eq:wq-vector} \quad ,
\end{equation}
\begin{equation}
\mathbf{v} = \begin{bmatrix}
\mathbf{v}^{0} \\
\vdots \\
\mathbf{v}^{Q - 1} \\
\mathbf{0}_{2N \times 1}
\end{bmatrix}_{4N \times 1}
\label{eq:v-vector}
\end{equation}
and
\begin{equation}
\mathbf{v}^{q} = \begin{bmatrix}
\mathbf{b}^{q} \\
\mathbf{0}_{P \times 1}
\end{bmatrix}_{2P \times 1} \: ,
\label{eq:vq-vector} 
\end{equation}
with $\mathbf{0}_{2N \times 1}$ and $\mathbf{0}_{P \times 1}$ being vectors of zeros.
The key point here is that the auxiliary matrix-vector product
(equation \ref{eq:w_alpha_beta}) represents a 2D discrete convolution and can be 
efficiently computed by using the 2D Fast Fourier Transform (2D FFT). 

The BCCB matrix $\mathbf{C}_{\boldsymbol{\alpha\beta}}$ (equation \ref{eq:w_alpha_beta})
is formed by $2Q \times 2Q$ blocks, 
where each block $\mathbf{C}_{\boldsymbol{\alpha\beta}}^{q}$ is a $2P \times 2P$ circulant matrix.
The entire BCCB matrix $\mathbf{C}_{\boldsymbol{\alpha\beta}}$ is defined by properly
downshifting its first block column
\begin{equation}
	\left[ \mathbf{C}_{\boldsymbol{\alpha\beta}} \right]_{(0)} = \begin{bmatrix}
		\mathbf{C}_{\boldsymbol{\alpha\beta}}^{0} \\
		\vdots \\
		\mathbf{C}_{\boldsymbol{\alpha\beta}}^{Q-1} \\
		\mathbf{0}_{2P \times 2P} \\
		\mathbf{C}_{\boldsymbol{\alpha\beta}}^{-Q+1} \\
		\vdots \\
		\mathbf{C}_{\boldsymbol{\alpha\beta}}^{-1}
	\end{bmatrix}_{4N \times 2P} \: ,
	\label{eq:C_alpha_beta_first_block_column}
\end{equation}
where $\mathbf{0}_{2P \times 2P}$ is a matrix of zeros. Similarly, 
each block $\mathbf{C}_{\boldsymbol{\alpha\beta}}^{q}$, $q = -Q+1, \dots, Q-1$,
is obtained by properly downshifting its first column
\begin{equation}
	\mathbf{c}_{\boldsymbol{\alpha\beta}}^{q} = \begin{bmatrix}
		a^{\alpha\beta}_{q0} \\
		\vdots \\
		a^{\alpha\beta}_{q(P-1)} \\
		0 \\
		a^{\alpha\beta}_{q(-P+1)} \\
		\vdots \\
		a^{\alpha\beta}_{q(-1)}
	\end{bmatrix}_{2P \times 1} \: ,
	\label{eq:cq_alpha_beta}
\end{equation}
where $a^{\alpha\beta}_{qp}$, $p = -P+1, \dots, P-1$, are the elements of
matrix component $\mathbf{A_{\boldsymbol{\alpha\beta}}}$ described in terms of
block indices $q$ and $p$ (equations \ref{eq:q-x-oriented}--\ref{eq:p-y-oriented}). 
The BCCB matrix $\mathbf{C}_{\boldsymbol{\alpha\beta}}$ is diagonalized by
$\mathbf{F}_{2Q} \otimes \mathbf{F}_{2P}$, where ``$\otimes$" denotes the Kronecker product
\citep[e.g.,][ p. 242]{horn_johnson1991} and $\mathbf{F}_{2Q}$ and $\mathbf{F}_{2P}$ are 
the $2Q \times 2Q$ and $2P \times 2P$ unitary DFT matrices \citep[][ p. 31]{davis1979}.
Due to this property, the auxiliary matrix-vector product (equation \ref{eq:w_alpha_beta}) 
can be computed as follows \citep{takahashi2020convolutional}:
\begin{equation}
\mathbf{F}_{2Q}^{\ast} \left[ 
\mathbf{L}_{\boldsymbol{\alpha\beta}} \circ \left(\mathbf{F}_{2Q} \, \mathbf{V} \, \mathbf{F}_{2P} \right) 
\right] \mathbf{F}_{2P}^{\ast} = \mathbf{W}_{\boldsymbol{\alpha\beta}} \: ,
\label{eq:2d-discrete-convolution}
\end{equation}
where ``$\circ$" denotes the Hadamard (element-wise) product \citep[e.g.,][ p. 298]{horn_johnson1991},
``$\ast$" denotes the complex conjugate, 
$\mathbf{W}_{\boldsymbol{\alpha\beta}}$ and $\mathbf{V}$ are $2Q \times 2P$ matrices obtained
by rearranging, respectively, vectors $\mathbf{w}_{\boldsymbol{\alpha\beta}}$ 
(equation \ref{eq:w_alpha_beta_partitioned}) and $\mathbf{v}$ (equation \ref{eq:v-vector})
along their rows and $\mathbf{L}_{\boldsymbol{\alpha\beta}}$ is a $2Q \times 2P$ matrix
given by
\begin{equation}
\mathbf{L}_{\boldsymbol{\alpha\beta}} = \sqrt{4QP} \; 
\mathbf{F}_{2Q} \, \mathbf{G}_{\boldsymbol{\alpha\beta}} \, \mathbf{F}_{2P} \: ,
\label{eq:L_alpha_beta}
\end{equation}
with
\begin{equation}
\mathbf{G}_{\boldsymbol{\alpha\beta}} = \begin{bmatrix}
\left( \mathbf{c}_{\boldsymbol{\alpha\beta}}^{0} \right)^{\top} \\
\vdots \\
\left( \mathbf{c}_{\boldsymbol{\alpha\beta}}^{Q-1} \right)^{\top} \\
\mathbf{0}_{1 \times 2P} \\
\left( \mathbf{c}_{\boldsymbol{\alpha\beta}}^{-Q+1} \right)^{\top} \\
\vdots \\
\left( \mathbf{c}_{\boldsymbol{\alpha\beta}}^{-1} \right)^{\top}
\end{bmatrix}_{2Q \times 2P} \: ,
\label{eq:G_alpha_beta}
\end{equation}
defined by the first columns $\mathbf{c}_{\boldsymbol{\alpha\beta}}^{q}$ 
(equation \ref{eq:cq_alpha_beta}), $q = -Q+1, \dots, Q-1$, of all circulant blocks
$\mathbf{C}_{\boldsymbol{\alpha\beta}}^{q}$ (equation \ref{eq:C_alpha_beta_first_block_column}).
Hence, the whole BCCB matrix $\mathbf{C}_{\boldsymbol{\alpha\beta}}$ does not have to be
formed, but only its first column. 
Besides, the symmetries defined by equations 
\ref{eq:Axx_symmetry}--\ref{eq:Ayz_q_internal_block_symmetry_y_oriented} imply
that all elements of $\mathbf{G}_{\boldsymbol{\alpha\beta}}$ can be obtained by 
using only the first column of $\mathbf{A_{\boldsymbol{\alpha\beta}}}$. Consequently, 
the whole matrices $\mathbf{A_{\boldsymbol{\alpha\beta}}}$ do not have to be formed
as well, but only their first columns.

It is important noting that the left side of equation \ref{eq:2d-discrete-convolution} represents 
the 2D Inverse Discrete Fourier Transform (2D IDFT) of the term in brackets. This term, in turn,
represents the Hadamard product of $\mathbf{L}_{\boldsymbol{\alpha\beta}}$ 
(equation \ref{eq:L_alpha_beta}) and the 2D Discrete Fourier Transform (2D DFT) of $\mathbf{V}$.
Similarly, equation \ref{eq:L_alpha_beta} shows that $\mathbf{L}_{\boldsymbol{\alpha\beta}}$ is
obtained by computing the 2D DFT of matrix $\mathbf{G}_{\boldsymbol{\alpha\beta}}$
(equation \ref{eq:G_alpha_beta}). Hence, equations \ref{eq:2d-discrete-convolution} and
\ref{eq:L_alpha_beta} can be efficiently computed by using the 2D FFT. After that, the elements of
vector $\mathbf{t}_{\boldsymbol{\alpha\beta}}$ (equation \ref{eq:t-alpha-beta}) can be
retrieved from the first quadrant of matrix $\mathbf{W}_{\boldsymbol{\alpha\beta}}$
(equation \ref{eq:2d-discrete-convolution}). By combining the results obtained for all
components $\boldsymbol{\alpha\beta}$, 
$\boldsymbol{\alpha}, \boldsymbol{\beta} = \boldsymbol{x}, \boldsymbol{y}, \boldsymbol{z}$,
we can show that
\begin{equation}
\mathbf{F}_{2Q}^{\ast} \left[ 
\mathbf{L} \circ \left(\mathbf{F}_{2Q} \, \mathbf{V} \, \mathbf{F}_{2P} \right) 
\right] \mathbf{F}_{2P}^{\ast} = \mathbf{W} \: ,
\label{eq:2d-discrete-convolution-complete}
\end{equation}
where
\begin{equation}
\mathbf{W} = \mathbf{W}_{\boldsymbol{xx}} + \mathbf{W}_{\boldsymbol{xy}} + \mathbf{W}_{\boldsymbol{xz}} + \mathbf{W}_{\boldsymbol{yy}} + \mathbf{W}_{\boldsymbol{yz}} + \mathbf{W}_{\boldsymbol{zz}}
\label{eq:W}
\end{equation}
and
\begin{equation}
\mathbf{L} = \mathbf{L}_{\boldsymbol{xx}} + \mathbf{L}_{\boldsymbol{xy}} + \mathbf{L}_{\boldsymbol{xz}} + \mathbf{L}_{\boldsymbol{yy}} + \mathbf{L}_{\boldsymbol{yz}} + \mathbf{L}_{\boldsymbol{zz}} \: ,
\label{eq:L}
\end{equation}
with $\mathbf{L}_{\boldsymbol{\alpha\beta}}$ defined by equation \ref{eq:L_alpha_beta}.
Then, the elements of $\mathbf{t}$ (equation \ref{eq:t}) are obtained from the first
quadrant of $\mathbf{W}$ (equations \ref{eq:2d-discrete-convolution-complete} and
\ref{eq:W}).

Finally, it can be shown that the product 
\begin{equation}
\mathbf{t} = 
\mathbf{A}^{\top} \mathbf{b}
\label{eq:t_AT}
\end{equation}
can be computed by using equation \ref{eq:2d-discrete-convolution-complete}.
The difference is that, in this case, matrices $\mathbf{G}_{\boldsymbol{\alpha\beta}}$
(equation \ref{eq:G_alpha_beta}) are defined by using the new vectors
\begin{equation}
\mathbf{c}_{\boldsymbol{\alpha\beta}}^{q} = \begin{bmatrix}
a^{\alpha\beta}_{q0} \\
\vdots \\
a^{\alpha\beta}_{q(-P+1)} \\
0 \\
a^{\alpha\beta}_{q(P-1)} \\
\vdots \\
a^{\alpha\beta}_{q1}
\end{bmatrix}_{2P \times 1} \: .
\label{eq:cq_alpha_beta_trnasposed}
\end{equation}


