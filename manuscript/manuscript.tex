%\documentclass[paper,twocolumn,twoside]{geophysics}
\documentclass[manuscript,noblind]{geophysics}
%\documentclass[manuscript]{geophysics}

% An example of defining macros
\newcommand{\rs}[1]{\mathstrut\mbox{\scriptsize\rm #1}}
\newcommand{\rr}[1]{\mbox{\rm #1}}

% Extra packages
\usepackage{amsmath}
%\usepackage[]{algorithm2e}
\usepackage{algorithm}
\usepackage{bm}
\usepackage[hyphens,spaces]{url}
\usepackage[pdftex,colorlinks=true]{hyperref}
\hypersetup{
	allcolors=black,
}
\usepackage{lipsum}
\usepackage[table]{xcolor}

\begin{document}

\title{Convolutional equivalent layer for magnetic data processing}

\renewcommand{\thefootnote}{\fnsymbol{footnote}} 

\ms{GEO-XXXX} % manuscript number

\address{
\footnotemark[2]Observat\'{o}rio Nacional, Department of Geophysics, Rio de Janeiro, Brazil\\
\footnotemark[1] Corresponding author: vanderlei@on.br
}
\author{Diego Takahashi\footnotemark[2], Vanderlei C. Oliveira{ }Jr.\footnotemark[2]\footnotemark[1] and 
Val{\'e}ria C. F. Barbosa\footnotemark[2]}

%\footer{Example}
\lefthead{Takahashi, Oliveira{ }Jr. \& Barbosa}
\righthead{Magnetic convolutional equivalent layer}

\maketitle

% Main body
\begin{abstract}



\end{abstract}
\section{Introduction}

Equivalent layer background:

\cite{dampney1969}, \cite{blakely1996}, \cite{emilia1973}, \cite{li2014using}
\\\\
Fast methods for equivalent layer:

\cite{leao-silva1989}, \cite{mendonca-silva1994}, \cite{oliveirajr-etal2013}, \cite{li-oldenburg2010}, \cite{siqueira-etal2017}, \cite{mendoncca2020subspace}
\\\\
Methods for solving Toeplitz matrices:

\cite{golub-vanloan2013}, \cite{levinson1946}, \cite{chan-jin2007}
\\\\
Toeplitz matrices in Potential methods:

\cite{zhang-wong2015}, \citealp{zhang-etal2016}, \cite{hogue2020tutorial}, \cite{renaut2020fast}
\\\\
Convolutional equivalent layer grav:

\cite{siqueira-etal2017}, \cite{takahashi2020convolutional}
\\\\
About this work:

\section{Methodology}

%======================================================================================
\subsection*{The total-field anomaly}
%======================================================================================

The Earth's magnetic field is commonly divided in three parts: main field, crustal field and external field. The main field is generated in the outter core in a process of self-sustaining dynamo, the crustal field is generated by magnetic bodies in the litosphere and the external field is generated by electrical currents in the ionosphere and magnetosphere. For exploration geophysics, the crustal field is the object of study, which makes the separation of this data from the acquisition dataset a very important step.

The combination of the main field and crustal field is known as internal field or total-field. Taking the difference between the main field given by a model (e.g. IGRF) and this internal field, at the same location, we have the total-field anomaly.

Let $\Delta T(x_i, y_i, z_i), i =  1,...,N$, be a observed dataset in a region considering a right-handed Cartesian coordinate system with the $x$-axis pointing north, $y$-axis pointing east and $z$-axis pointing downward. The total-field anomaly at the $i$th observation can be approximated to:
\begin{equation}
	\Delta T(x_i, y_i, z_i) = \hat{\mathbf{F}}^{\top} \mathbf{B}(x_i, y_i, z_i) \: ,
	\label{eq:tfanomaly}
\end{equation}
where, $\mathbf{B}(x_i, y_i, z_i)$ is the crustal field, $\hat{\mathbf{F}}^{\top}$ is the transposed unitary vector with the main field directions, with $\hat{\mathbf{F}}$ described as:
\begin{equation}
	\hat{\mathbf{F}} = \left[
	\begin{array}{c}
		F_x \\
		F_y \\
		F_z
	\end{array} \right] = 
	\left[
	\begin{array}{c}
		\cos(I_{0}) \, \cos(D_{0}) \\
		\cos(I_{0}) \, \sin(D_{0}) \\
		\sin(I_{0})
	\end{array} \right],
	\label{eq:unit_vector_F}
\end{equation}
where $I_{0}$ is the inclination and $D_{0}$ the declination of the main field, respectively.

Considering a uniform magnetized body with volume $v$ and a total magnetization vector $\mathbf{m}$, the induced magnetic field at the $i$th observation is:
\begin{equation}
	\mathbf{B}(x_i, y_i, z_i) = c_{m} \, \frac{\mu_{0}}{4\pi} \: \mathbf{M}(x_i, y_i, z_i) \: 
	\mathbf{m} \: ,
	\label{eq:induction-general}
\end{equation}
where, $\mu_{0} = 4\pi \, 10^{-7}$ H/m is the magnetic constant, 
$c_{m} = 10^{9}$ is a constant that transforms the induced magnetic field from Tesla (T) to nanotesla (nT) and 
$\mathbf{M}(x_i, y_i, z_i)$ is a $3 \times 3$ matrix given by:
\begin{equation}
	\mathbf{M}(x_i, y_i, z_i) =
	\left[
	\begin{array}{ccc}
		\partial_{xx} \phi(x_i,y_i,z_i) & \partial_{xy} \phi(x_i,y_i,z_i) & 
		\partial_{xz} \phi(x_i,y_i,z_i) \\
		\partial_{xy} \phi(x_i,y_i,z_i) & \partial_{yy} \phi(x_i,y_i,z_i) & 
		\partial_{yz} \phi(x_i,y_i,z_i) \\
		\partial_{xz} \phi(x_i,y_i,z_i) & \partial_{yz} \phi(x_i,y_i,z_i) & 
		\partial_{zz} \phi(x_i,y_i,z_i)
	\end{array}
	\right] \: ,
	\label{eq:M}
\end{equation}
where, $\partial_{\alpha\beta} \phi(x_i,y_i,z_i)$ with $\alpha = x, y, z$ and 
$\beta = x, y, z$, are the second derivatives of the function $\phi(x_i,y_i,z_i)$
with respect to $x$, $y$ and $z$:

\begin{equation}
	\phi(x_i,y_i,z_i) = \int\int\limits_{v}\int \frac{1}{r} \: dv,
	\label{eq:phi}
\end{equation}
where 

\begin{equation}
	\dfrac{1}{r} = \dfrac{1}{\sqrt{(x_i - x_j)^{2} + 
			(y_i - y_j)^{2} + (z_i - z_j)^{2}}} \,
	\label{eq:r}
\end{equation}
and $x_j, y_j, z_j$ are the $j$th Cartesian coordinates within the volume element of the magnetized body with volume $v$, where the integral (equation \ref{eq:phi}) is conducted.

Rewriting the equation \ref{eq:tfanomaly} using the magnetic induction given by equation \ref{eq:induction-general} the total-field anomaly $\Delta T(x, y, z)$ is given by:
\begin{equation}
\Delta T(x_i, y_i, z_i) = c_{m} \, \frac{\mu_{0}}{4\pi} \, m \: \hat{\mathbf{F}}^{\top} 
\mathbf{M}(x_i, y_i, z_i) \: \hat{\mathbf{m}} \: ,
\label{eq:tfanomaly-general}
\end{equation}
where $m$ is the magnetization intensity and $\hat{\mathbf{m}}$ is the directional unitary vector.

%======================================================================================
\subsection*{Equivalent layer for magnetic data}
%======================================================================================

Grounded on the equivalent layer principle it is possible to calculate the total-field anomaly $\Delta T(x_i, y_i, z_i)$ (equation \ref{eq:tfanomaly-general}) with the convolution between the harmonic function and the physical property:
\begin{equation}
\Delta T(x_{i}, y_{i}, z_{i})
= \int \limits_{-\infty}^{+\infty}
\int \limits_{-\infty}^{+\infty}
p(x_j, y_j, z_{c}) \,
\left[ c_{m} \, \frac{\mu_{0}}{4\pi} \,
\hat{\mathbf{F}}^{\top} \mathbf{H} \, 
\hat{\mathbf{h}} \right] \,
dx \, dy \: ,
\label{eq:tf-p-continuous-mag-positive}
\end{equation}
where $z_c$ is a constant representing the depth of the equivalent layer with $z_i < z_c$. The unitary vector $\hat{\mathbf{h}}$ is the magnetization directions of the equivalent sources over the layer:
\begin{equation}
\hat{\mathbf{h}} = \left[
\begin{array}{c}
{h}_x \\
{h}_y \\
{h}_z
\end{array} \right] = 
 \left[
\begin{array}{c}
\cos(I) \, \cos(D) \\
\cos(I) \, \sin(D) \\
\sin(I)
\end{array}
\right] \: ,
\label{eq:h_hat}
\end{equation}
where $I$ and $D$ are, respectively, the inclination and declination of the equivalent sources, $\mathbf{H}$ is a $3 \times 3$ matrix that contains the second derivatives in relation to the observed Cartesian coordinates $x, y, z$ as presented in equation \ref{eq:r}:
\begin{equation}
\mathbf{H} =
\left[
\begin{array}{ccc}
\partial_{xx} \frac{1}{r} & 
\partial_{xy} \frac{1}{r} & 
\partial_{xz} \frac{1}{r} \\
\partial_{xy} \frac{1}{r} & 
\partial_{yy} \frac{1}{r} & 
\partial_{yz} \frac{1}{r} \\
\partial_{xz} \frac{1}{r} & 
\partial_{yz} \frac{1}{r} & 
\partial_{zz} \frac{1}{r}
\end{array}
\right] \: =
\left[
\begin{array}{ccc}
H_{xx} & 
H_{xy} & 
H_{xz} \\
H_{xy} & 
H_{yy} & 
H_{yz} \\
H_{xz} & 
H_{yz} & 
H_{zz}
\end{array}
\right] \: ,
\label{eq:Hi}
\end{equation}
and the physical property $p(x_j, y_j, z_{c})$ represents the $j$th magnetic dipole moment (in $Am^2$) produced by the $j$th dipole locate over the equivalent layer at the $j$th Cartesian coordinates 
$(x_j, y_j, z_{c})$.

Discretizing equation \ref{eq:tf-p-continuous-mag-positive} we get:
\begin{equation}
\Delta T(x_{i}, y_{i}, z_{i}) = \sum_{j=1}^{M} p_j a_{ij}\: ,
\label{eq:integral-sum_mag}
\end{equation}
where the subscript $j$ denotes a discrete equivalent source, totaling $M$ equivalent sources 
that are distributed over the layer and $a_{ij}$ is given by:
\begin{equation}
a_{ij}
= c_{m} \, \frac{\mu_{0}}{4\pi} \, \hat{\mathbf{F}}^{\top} \mathbf{H}_{ij} \: \hat{\mathbf{h}} \: .
\label{eq:aij_mag}
\end{equation}
Equation \ref{eq:integral-sum_mag} can be written in matrix notation as:
\begin{equation}
\mathbf{d}(\mathbf{p}) = \mathbf{A} \mathbf{p} \: ,
\label{eq:predicted-data-vector_mag}
\end{equation}
where $\mathbf{d}(\mathbf{p})$ is the $N-$dimensional vector of total-field anomaly 
$(\Delta T(x_i, y_i, z_i), i =  1,...,N)$, $\mathbf{A}$ is a matrix containing the elements given by equation \ref{eq:aij_mag}, also known as the sensitivity matrix and $\mathbf{p}$ is the vector containing the dipole moments of each equivalent source. 

Let $\mathbf{d}^{o}$ be an $N-$dimensional vector that contains the observed total-field anomaly.
By solving the least-squares normal equation associated with equation \ref{eq:predicted-data-vector_mag}, 
we get
\begin{equation}
	\mathbf{A}^{\top}\mathbf{A}{\mathbf{p}} = 
	\mathbf{A}^{\top} \mathbf{d}^{o} \: ,
	\label{eq:normal-equations}
\end{equation} 
and we estimate the parameter vector that contains the unknown dipole moments over the equivalent layer, i.e.,
\begin{equation}
\hat{\mathbf{p}} = \left( \mathbf{A}^{\top}\mathbf{A} %+ 
%\mu \, \mathbf{I} 
\right)^{-1}
\mathbf{A}^{\top} \mathbf{d}^{o} \:
\label{eq:estimated-p-parameter-space}
\end{equation}

Equation \ref{eq:estimated-p-parameter-space} will be referenced throughout this work as the classical method for solving the equivalent layer.

%======================================================================================
\subsection{Conjugate Gradient Least Square method (CGLS)}
%======================================================================================

As an alternative from the clasical method of parameter estimative, the conjugate gradient (CG) is a well-known iterative method for solving linear systems with symmetric positive definite matrices. By minimizing the quadratic form:
\begin{equation}
\, \Phi(\mathbf{p}) = \frac{1}{2} \, \mathbf{p}^{\top} \, \mathbf{A} \, \mathbf{p} \,
- \mathbf{{d}^{o}}^{\top} \, \mathbf{p}\: ,
\label{eq:estimated-p-cg}
\end{equation}
it is possible to solve the system by constructing a basis of conjugate directions $c \in R^N$ \citep{aster2018parameter}. As we are solving a general least square problem and matrix $\mathbf{A}$ (equation \ref{eq:aij_mag}) is not symmetric, instead we minimize:
\begin{equation}
\, || \mathbf{A} \, \mathbf{p} - \mathbf{d}^o||_2 \: ,
\label{eq:estimated-p-cgls}
\end{equation}
by applying the conjugate gradient to the normal equations (equation \ref{eq:normal-equations}).

In theory, this method is bound to converge at $N$ iterations maximum, but in a later part of this work we show with numerical results that the convergence is much faster for the linear system we are solving.

A pseudocode for the CGLS method follows:

\begin{algorithm}[H]
	Input: $\mathbf{A} \in R^{N \times M} $ and $\mathbf{d}^o \in R^N$.
	
	Output: Estimative of parameter vector $\hat{\mathbf{p}}$.
	
	Let $it = 0$, $\hat{\mathbf{p}}^{(0)} = {\mathbf{0}}$, $\mathbf{c}^{(-1)} = {\mathbf{0}}$, $\beta_0 = 0$, $\mathbf{s}^{(0)} = \mathbf{d}^{o} - \mathbf{A} \hat{\mathbf{p}}^{(0)}$ and $\mathbf{r}^{(0)} = \mathbf{A}^{\top} \mathbf{s}^{(0)}$.
	
	1 - If $it > 0$, let $\beta_{(it)} = \dfrac{{\mathbf{r}^{(it)}}^{\top} \, \mathbf{r}^{(it)}} {{\mathbf{r}^{(it - 1)}}^{\top} \, \mathbf{r}^{(it - 1)}}$
	
	2 - $\mathbf{c}^{(it)} = \mathbf{r}^{(it)} - \alpha_{(it)} \,\beta_{(it)} \, \mathbf{c}^{(it - 1)}$.
	
	3 - $\alpha_{(it)} = \dfrac{{||\mathbf{r}^{(it)}||^2_2}}{({\mathbf{c}^{(it)}}^{\top} \, \mathbf{A}^{\top})(\mathbf{A} \, \mathbf{c}^{(it)})}$.
	
	4 - $\hat{\mathbf{p}}^{(it + 1)} = \hat{\mathbf{p}}^{(it)} - \alpha_{(it)} \, \mathbf{c}^{(it)}$.
	
	5 - $\mathbf{s}^{(it + 1)} = \mathbf{s}^{(it)} - \alpha_{(it)} \, \mathbf{A} \, \mathbf{c}^{(it)}$.
	
	6 - $\mathbf{r}^{(it + 1)} = \mathbf{A}^{\top} \, \mathbf{s}^{(it + 1)}$.
	
	7 - $it = it + 1$.
	
	8 - Repeat previous steps until convergence.
	
	\caption{Conjugate Gradient Least Square pseudocode \citep[][ p. 166]{aster2018parameter}.}
\label{al:cgls-algorithm}
\end{algorithm}

Different from the classical least-square solution (equation \ref{eq:estimated-p-parameter-space}), the CGLS solution (Algorithm 1) requires neither inverse matrix nor matrix-matrix product.The CGLS method only requires: one matrix-vector multiplication out of the loop and two matrix-vector multiplications, in steps 3 and 6, at each iteration. 

In this work, we will reduce the computational cost of the equivalent layer by substituting exactly these two matrix-vector products with a much more efficient algorithm.

%======================================================================================
\subsection{Conjugate Gradient Least Square method convergence criteria}
%======================================================================================

In theory, this method is bound to converge at $N$ iterations maximum, but in a later part of this work we show with numerical results that the convergence is much faster for the linear system we are solving. Setting a minimum tolerance of the residuals is a good option to carry out this algorithm efficiently and still obtaining very good results. Another possibility is to set an invariance to the euclidian norm of residuals between iteractions, wich would increase algorithm runtime, but with smaller residuals. We chose the first option, as we achieve satisfatory results.

%======================================================================================
\subsection{Non-symmetric Block-Toeplitz Toeplitz-Block structure of matrix $\mathbf{A}$}
%======================================================================================

Let us consider that the observed total-field anomaly is located on an $N_x \times N_y$ regular grid of points spaced by $\Delta_x$ and $\Delta_y$ along the $x$- and $y$-directions, respectively.
The notation used in this work will be the same as the one presented in \cite{takahashi2020convolutional}, where the authors described the structure of the symmetric Block-Toeplitz Toeplitz-Block matrix of the gravimetric equivalent layer. Here, we also establish a relation between the pair of \emph{matrix coordinates} $(x_i, y_i)$, $i = 1, ..., N$ or $(x_j, y_j)$, $j = 1, ..., M = N $ and a pair of \emph{grid coordinates} $(x_k, y_l)$ given as:
\begin{equation}
x_{i} \equiv x_{k} = x_{1} + \left[ k(i) - 1 \right] \, \Delta x \: , 
\label{eq:xi}
\end{equation}
and
\begin{equation}
y_{i} \equiv y_{l} = y_{1} + \left[ l(i) - 1 \right] \, \Delta y \: .
\label{eq:yi}
\end{equation}
In a $x$-\textit{oriented grid} the indices $i$ (or $j$) relate as integer functions of $k$ and $l$ by:
\begin{equation}
i(k, l) = (l - 1) \, N_{x} + k \: ,
\label{eq:i-x-oriented}
\end{equation}
\begin{equation}
l(i) = \Bigg\lceil \frac{i}{N_{x}} \Bigg\rceil
\label{eq:l-x-oriented}
\end{equation}
and
\begin{equation}
k(i)  = i - \Bigg\lceil \frac{i}{N_{x}} \Bigg\rceil N_{x} + N_{x} \: .
\label{eq:k-x-oriented}
\end{equation}
For  $y$-\textit{oriented grid} they are given by:
\begin{equation}
i(k, l) = (k - 1) \, N_{y} + l \: ,
\label{eq:i-y-oriented}
\end{equation}
\begin{equation}
k(i) = \Bigg\lceil \frac{i}{N_{y}} \Bigg\rceil
\label{eq:k-y-oriented}
\end{equation}
and
\begin{equation}
l(i) = i - \Bigg\lceil \frac{i}{N_{y}} \Bigg\rceil N_{y} + N_{y} \: ,
\label{eq:l-y-oriented}
\end{equation}
where in equations \ref{eq:l-x-oriented} to \ref{eq:l-y-oriented}, $\lceil .\rceil$ is the celing function.
Figure \ref{fig:methodology} shows an example of a grid $N_{x} \times N_{y}$, where $N_{x} = 4$ and $N_{y} = 3$ demonstrating the relation between the \emph{matrix coordinates} with $k(i)$ and $l(i)$ depending on the orientation of the grid.

The $N \times M$ sensitivity matrix  $\mathbf{A}$ (equation \ref{eq:aij_mag}) can be represented as a grid of $Q \times Q$ blocks $\mathbf{A}^q$, $q = -Q+1,...,0,..., Q-1$. Each block  $\mathbf{A}^q$ has $P \times P$ elements $a^{q}_p$ where $p = -P+1^,...,0,...,P-1$

In a $x$-\textit{oriented grid} $q$ and $p$ give the number of blocks ($Q = N_{y}$) and the number of elements of each block ($P = N_{x}$). They can be defined by the integer functions:
\begin{equation}
q(i, j) = \; l(i) - l(j)
\label{eq:q-x-oriented}
\end{equation}
and
\begin{equation}
p(i, j) = \; k(i) - k(j) \: ,
\label{eq:p-x-oriented}
\end{equation}
where equations \ref{eq:l-x-oriented} and \ref{eq:k-x-oriented} describe $l(i)$ and $l(j)$ and $k(i)$ and $k(j)$, respectively. When using $y$-oriented grids, $q$ and $p$ still define the number of block and block elements, but $Q = N_{x}$ and $P = N_{y}$. Moreover, the integer functions changes to:
\begin{equation}
q(i, j) = \; k(i) - k(j) 
\label{eq:q-y-oriented}
\end{equation}
and
\begin{equation}
p(i, j) = \; l(i) - l(j) \: ,
\label{eq:p-y-oriented}
\end{equation}
where equation \ref{eq:k-y-oriented} now defines $k(i)$ and $k(j)$ and equation \ref{eq:l-y-oriented} defines $l(i)$ and $l(j)$. Note that equations \ref{eq:q-x-oriented}, \ref{eq:p-x-oriented}, \ref{eq:q-y-oriented} and \ref{eq:q-y-oriented} differs from the ones presented in \cite{takahashi2020convolutional} by the absence of the module.

In both $x$- or $y$-\textit{orientation}, matrix $\mathbf{A}$ (equation \ref{eq:aij_mag}) can be rewritten by the indices $q = -Q + 1,...,0,..., Q-1$ defining the number of its blocks:
\begin{equation}
\mathbf{A} = \begin{bmatrix}
\mathbf{A}^{0}   & \mathbf{A}^{-1} & \cdots          & \mathbf{A}^{-Q+1} \\
\mathbf{A}^{1}   & \mathbf{A}^{0}  & \mathbf{A}^{-1} & \vdots           \\ 
\vdots           & \mathbf{A}^{1}  & \ddots          & \mathbf{A}^{-1}   \\
\mathbf{A}^{Q-1} & \cdots          & \mathbf{A}^{1}  & \mathbf{A}^{0}                 
\end{bmatrix}_{N \times N} \: ,
\label{eq:BTTB_A}
\end{equation}
and by indice $p$, where each block has $P \times P$ elements $a^{q}_{p}$, $p = -P + 1,..., 0, \dots, P - 1$:
\begin{equation}
\mathbf{A}^{q} = \begin{bmatrix}
a^{q}_{0}   & a^{q}_{-1} & \cdots     & a^{q}_{-P+1} \\
a^{q}_{1}   & a^{q}_{0}  & a^{q}_{-1} & \vdots           \\ 
\vdots      & a^{q}_{1}  & \ddots     & a^{q}_{-1}   \\
a^{q}_{P-1} & \cdots     & a^{q}_{1}  & a^{q}_{0}                 
\end{bmatrix}_{P \times P} \: ,
\label{eq:Aq_block}
\end{equation}
In general, matrix $\mathbf{A}$ (equation \ref{eq:aij_mag}) is a non-symmetric BTTB, i.e., its blocks are non-symmetric ($\mathbf{A}^{-Q+1} \neq \mathbf{A}^{Q-1} $) and its elements also are non-symmetric ($a^{q}_{-1} \neq a^{q}_{1}$). Depending on specific values of the main field direction and the equivalent sources magnetization directions, matrix $\mathbf{A}$ can assume other structures, for example, when $\hat{\mathbf{F}} = [0, 0, 1]$ and $\hat{\mathbf{h}} = [0, 0, 1]$ it becomes symmetric. In this work, we are considering the more commom situation for the matrix $\mathbf{A}$.

Also differently for the symmetric sensitivity matrix described by \cite{takahashi2020convolutional}, the non-symmetric BTTB matrix cannot be reconstructed only by its first column. The construction of the matrix $\mathbf{A}$ (equation \ref{eq:aij_mag}) needs four columns: the first and last columns of the first column of blocks and the first and last columns of the last column of blocks. This has a physical implication in the equivalent layer which is not possible to use only one equivalent source to reprduce all elements of matrix $\mathbf{A}$, such as in the gravity case as demonstrared by \cite{takahashi2020convolutional}. Rather, in the magnetic case it takes four equivalent sources positioned at each corner of the equivalent layer. Figure \ref{fig:4_equivalent_sources} shows the positioning of the equivalent sources in a regular grid $N_x = 4$ $N_y = 3$ necessary to calculate the four columns capable of recover the matrix $\mathbf{A}$.

In this work, we propose a different approach, by calculating the first column of all six different components of second derivatives matrices from $\mathbf{H}_{ij}$ (equation \ref{eq:Hi}). These matrices are, in fact, symmetrics or skew-symmetrics BTTBs, meaning that the first column has all elements of each matrix.

By substituting equations \ref{eq:unit_vector_F}, \ref{eq:h_hat} and \ref{eq:Hi} into equation \ref{eq:aij_mag}, it is possible to describe each element of the sensitivity matrix by the second derivative components of $\mathbf{H}_{ij}$:
\begin{equation}
\begin{split}
a_{ij} = c_{m} \, \frac{\mu_{0}}{4\pi} \, (F_x H_{xx} + F_y H_{xy} + F_z H_{xz}) \, h_x  \, + \\
(F_x H_{xy} + F_y H_{yy} + F_z H_{yz}) \, h_y  \, + \\
(F_x H_{xz} + F_y H_{yz} + F_z H_{zz}) \, h_z \: .
\end{split}
\label{eq:aij_mag_expand}
\end{equation}

If we consider that $c_{m} \, \frac{\mu_{0}}{4\pi}$, $\hat{\mathbf{F}}$ (equation \ref{eq:unit_vector_F}) and $\hat{\mathbf{h}}$ (equation\ref{eq:h_hat}) are constants multiplying the second derivatives $\mathbf{H}$ (equation \ref{eq:Hi}), the sensitivity matrix $\mathbf{A}$ (equation \ref{eq:aij_mag}) is purely the sum of the components $H_{xx} + H_{xy} + H_{xz} + H_{xy} + H_{yy} + H_{yz} + H_{xz} + H_{yz} + H_{zz}$ multiplied by the respectives constants of each component. Thus, despite $\mathbf{A}$ not being a symmetric BTTB matrix, it can be in fact, written by calculating only the first column of these components.
In the next few sections we will describe each component $\mathbf{H}$ as its own matrix.

%======================================================================================
\subsection{Structure of matrices components $\mathbf{H_{xx}}$, $\mathbf{H_{yy}}$ and $\mathbf{H_{zz}}$}
%======================================================================================

We can describe the elements of $\mathbf{H_{xx}}$, $\mathbf{H_{yy}}$ and $\mathbf{H_{zz}}$ by substituting equations \ref{eq:xi} and \ref{eq:yi} in equation \ref{eq:Hi}  as:
\begin{equation}
h^{xx}_{ij} = \frac{-1}{ \left[ 
		\left( \Delta k_{ij} \, \Delta x \right)^{2} + 
		\left( \Delta l_{ij} \, \Delta y \right)^{2} + 
		\left( \Delta z \right)^{2} \right]^{\frac{3}{2}}} + 
		\frac{3 (\Delta k_{ij} \, \Delta x )^{2}}{\left[ 
		\left( \Delta k_{ij} \, \Delta x \right)^{2} + 
		\left( \Delta l_{ij} \, \Delta y \right)^{2} + 
		\left( \Delta z \right)^{2} \right]^{\frac{5}{2}}} \: ,
\label{eq:hxx_mag}
\end{equation}

\begin{equation}
	h^{yy}_{ij} = \frac{-1}{ \left[ 
		\left( \Delta k_{ij} \, \Delta x \right)^{2} + 
		\left( \Delta l_{ij} \, \Delta y \right)^{2} + 
		\left( \Delta z \right)^{2} \right]^{\frac{3}{2}}} + 
	\frac{3 (\Delta l_{ij} \, \Delta y )^{2}}{\left[ 
		\left( \Delta k_{ij} \, \Delta x \right)^{2} + 
		\left( \Delta l_{ij} \, \Delta y \right)^{2} + 
		\left( \Delta z \right)^{2} \right]^{\frac{5}{2}}} \: 
	\label{eq:hyy_mag}
\end{equation}
and
\begin{equation}
	h^{zz}_{ij} = \frac{-1}{ \left[ 
		\left( \Delta k_{ij} \, \Delta x \right)^{2} + 
		\left( \Delta l_{ij} \, \Delta y \right)^{2} + 
		\left( \Delta z \right)^{2} \right]^{\frac{3}{2}}} + 
	\frac{3 (\Delta z )^{2}}{\left[ 
		\left( \Delta k_{ij} \, \Delta x \right)^{2} + 
		\left( \Delta l_{ij} \, \Delta y \right)^{2} + 
		\left( \Delta z \right)^{2} \right]^{\frac{5}{2}}} \: ,
	\label{eq:hzz_mag}
\end{equation}
where $\Delta z = z_j - z_i$, $\Delta k_{ij} =k(i) - k(j)$ (equations \ref{eq:k-x-oriented} or \ref{eq:k-y-oriented}) and $\Delta l_{ij} = l(i) - l(j)$ (equations \ref{eq:l-x-oriented} or \ref{eq:l-y-oriented}).
The principal components $\mathbf{H_{xx}}$, $\mathbf{H_{yy}}$ and $\mathbf{H_{zz}}$ (equations \ref{eq:hxx_mag}, \ref{eq:hyy_mag} and \ref{eq:hzz_mag}, respectively) of matrix $\mathbf{H}$ (equation \ref{eq:Hi}) are symmetric-Block-Toeplitz symmetric-Toeplitz-Block matrices. This means that $\mathbf{H_{xx}}$, $\mathbf{H_{yy}}$ and $\mathbf{H_{zz}}$ are Toeplitz and symmetric by its blocks and each of the blocks are symmetric Toeplitz matrices. 
For example, $\mathbf{H_{xx}}$ can be described by the \textit{block indice} $q$ that represent the block diagonals of this matrix as a grid of $Q \times Q$ blocks $\mathbf{H}^{q}_\mathbf{xx}$, $q = 0, \dots, Q - 1$:
\begin{equation}
\mathbf{H_{xx}} = \begin{bmatrix}
\mathbf{H}^{0}_\mathbf{xx}  & \mathbf{H}^{1}_\mathbf{xx} & \cdots         & \mathbf{H}^{Q-1}_\mathbf{xx} \\
\mathbf{H}^{1}_\mathbf{xx}  & \mathbf{H}^{0}_\mathbf{xx} & \ddots         & \vdots           \\ 
\vdots           & \ddots         & \ddots         & \mathbf{H}^{1}_\mathbf{xx}   \\
\mathbf{H}^{Q-1}_\mathbf{xx} & \cdots         & \mathbf{H}^{1}_\mathbf{xx} & \mathbf{H}^{0}_\mathbf{xx}                
\end{bmatrix}_{N \times N} \: .
\label{eq:BTTB_Hxx}
\end{equation}
And each diagonal of the blocks are represented by $P \times P$ elements $h^{q}_{p}$, $p = 0, \dots, P - 1$:
\begin{equation}
\mathbf{H}^{q}_\mathbf{xx} = \{h^{q}_p\} = \begin{bmatrix}
h^{q}_{0}   & h^{q}_{1} & \cdots    & h^{q}_{P-1} \\
h^{q}_{1}   & h^{q}_{0} & \ddots    & \vdots           \\ 
\vdots      & \ddots    & \ddots    & h^{q}_{1}   \\
h^{q}_{P-1} & \cdots    & h^{q}_{1} & h^{q}_{0}                 
\end{bmatrix}_{P \times P} \: .
\label{eq:Hxx_block}
\end{equation}
In a $x$-\textit{oriented grid} $Q = N_{x}$, $P = N_{y}$ and $q$ and $p$ can be defined by the functions:
\begin{equation}
q(i, j) = \; \mid l(i) - l(j) \mid
\label{eq:Hxx-q-x-oriented}
\end{equation}
and
\begin{equation}
p(i, j) = \; \mid k(i) - k(j) \mid \quad ,
\label{eq:Hxx-p-x-oriented}
\end{equation}
where $l(i)$ and $l(j)$ are defined by equation \ref{eq:l-x-oriented} 
and $k(i)$ and $k(j)$ are defined by equation \ref{eq:k-x-oriented}.
For $y$-oriented grids, $Q = N_{x}$, $P = N_{y}$ and the block indices
$q$ and $p$ are defined, respectively, by the following integer functions 
of the matrix indices $i$ and $j$:
\begin{equation}
q(i, j) = \; \mid k(i) - k(j) \mid 
\label{eq:Hxx-q-y-oriented}
\end{equation}
and
\begin{equation}
p(i, j) = \; \mid l(i) - l(j) \mid \quad ,
\label{eq:Hxx-p-y-oriented}
\end{equation}
This struture can also describe matrices $\mathbf{H_{yy}}$ and $\mathbf{H_{zz}}$ in the same manner and they are identical to the structure of the gravity sensitivity matrix from \cite{takahashi2020convolutional}.

%======================================================================================
\subsection{Structure of the components matrices $\mathbf{H_{xy}}$}
%======================================================================================

By substituting equations \ref{eq:xi} and \ref{eq:yi} in equation \ref{eq:Hi}, we can also describe the elements of $\mathbf{H_{xy}}$, as:
\begin{equation}
	h^{xy}_{ij} = \frac{3 (\Delta k_{ij} \, \Delta x )(\Delta l_{ij} \, \Delta y )}{\left[ 
		\left( \Delta k_{ij} \, \Delta x \right)^{2} + 
		\left( \Delta l_{ij} \, \Delta y \right)^{2} + 
		\left( \Delta z \right)^{2} \right]^{\frac{5}{2}}} \: ,
	\label{eq:hxy_mag}
\end{equation}

The component $\mathbf{H_{xy}}$ (equation \ref{eq:hxy_mag}) of matrix $\mathbf{H}$ (equation \ref{eq:Hi}) are skew symmetric-Block-Toeplitz skew symmetric-Toeplitz-Block matrices. This means that $\mathbf{H_{xy}}$ is Toeplitz and skew symmetric by its blocks and each of the blocks are skew symmetric Toeplitz matrices. 
This way, matrix $\mathbf{H_{xy}}$ can be described by the \textit{block indice} $q$ that represent the block diagonals of this matrix as a grid of $Q \times Q$ blocks $\mathbf{H}^{q}_\mathbf{xy}$, $q = -Q + 1, \dots, 0, \dots, Q - 1$:
\begin{equation}
	\mathbf{H_{xy}} = \begin{bmatrix}
		\mathbf{H}^{0}_\mathbf{xy}  & \mathbf{H}^{-1}_\mathbf{xy} & \cdots         & \mathbf{H}^{-Q+1}_\mathbf{xy} \\
		\mathbf{H}^{1}_\mathbf{xy}  & \mathbf{H}^{0}_\mathbf{xy} & \ddots         & \vdots           \\ 
		\vdots           & \ddots         & \ddots         & \mathbf{H}^{-1}_\mathbf{xy}   \\
		\mathbf{H}^{Q-1}_\mathbf{xy} & \cdots         & \mathbf{H}^{1}_\mathbf{xy} & \mathbf{H}^{0}_\mathbf{xy}                
	\end{bmatrix}_{N \times N} \: .
	\label{eq:BTTB_Hxy}
\end{equation}
And each diagonal of the blocks are represented by $P \times P$ elements $h^{q}_{p}$, $p = -P + 1, \dots, 0, \dots, P - 1$:
\begin{equation}
	\mathbf{H}^{q}_\mathbf{xy} =  \{h^{q}_p\} = \begin{bmatrix}
		h^{q}_{0}   & h^{q}_{-1} & \cdots    & h^{q}_{-P+1} \\
		h^{q}_{1}   & h^{q}_{0} & \ddots    & \vdots           \\ 
		\vdots      & \ddots    & \ddots    & h^{q}_{-1}   \\
		h^{q}_{P-1} & \cdots    & h^{q}_{1} & h^{q}_{0}                 
	\end{bmatrix}_{P \times P} \: .
	\label{eq:Hxy_block}
\end{equation}
In a $x$-\textit{oriented grid} $Q = N_{x}$, $P = N_{y}$ and $q$ and $p$ can be defined by the functions:
\begin{equation}
	q(i, j) = \; l(i) - l(j) 
	\label{eq:Hxy-q-x-oriented}
\end{equation}
and
\begin{equation}
	p(i, j) = \; k(i) - k(j) \quad ,
	\label{eq:Hxy-p-x-oriented}
\end{equation}
where $l(i)$ and $l(j)$ are defined by equation \ref{eq:l-x-oriented} 
and $k(i)$ and $k(j)$ are defined by equation \ref{eq:k-x-oriented}.
For $y$-oriented grids, $Q = N_{x}$, $P = N_{y}$ and the block indices
$q$ and $p$ are defined, respectively, by the following integer functions 
of the matrix indices $i$ and $j$:
\begin{equation}
	q(i, j) = \;  k(i) - k(j)  
	\label{eq:Hxy-q-y-oriented}
\end{equation}
and
\begin{equation}
	p(i, j) = \;  l(i) - l(j)  \quad ,
	\label{eq:Hxy-p-y-oriented}
\end{equation}
Important to clarify that in this case, as a skew symmetric matrix, the values of oposing diagonals have oposing signals, e.g., $\mathbf{H}^{-1}_\mathbf{xy} = -\mathbf{H}^{1}_\mathbf{xy}$ and $h^{q}_{-1} = -h^{q}_{1} $.

%======================================================================================
\subsection{Structure of the components matrices $\mathbf{H_{xz}}$}
%======================================================================================

By substituting equations \ref{eq:xi} and \ref{eq:yi} in equation \ref{eq:Hi}, the elements of $\mathbf{H_{xz}}$, are given by:
\begin{equation}
	h^{xz}_{ij} = \frac{3 (\Delta k_{ij} \, \Delta x)(\Delta z)}{\left[ 
		\left( \Delta k_{ij} \, \Delta x \right)^{2} + 
		\left( \Delta l_{ij} \, \Delta y \right)^{2} + 
		\left( \Delta z \right)^{2} \right]^{\frac{5}{2}}} \: ,
	\label{eq:hxz_mag}
\end{equation}

The component $\mathbf{H_{xz}}$ (equation \ref{eq:hxz_mag}) of matrix $\mathbf{H}$ (equation \ref{eq:Hi}) are skew symmetric-Block-Toeplitz symmetric-Toeplitz-Block matrices. This means that $\mathbf{H_{xz}}$ is Toeplitz and skew symmetric by its blocks and each of the blocks are symmetric Toeplitz matrices. 
Thus, matrix $\mathbf{H_{xz}}$ can be described by the \textit{block indice} $q$ that represent the block diagonals of this matrix as a grid of $Q \times Q$ blocks $\mathbf{H}^{q}_\mathbf{xz}$, $q = -Q + 1, \dots, 0, \dots, Q - 1$:
\begin{equation}
	\mathbf{H_{xz}} = \begin{bmatrix}
		\mathbf{H}^{0}_\mathbf{xz}  & \mathbf{H}^{-1}_\mathbf{xz} & \cdots         & \mathbf{H}^{-Q+1}_\mathbf{xz} \\
		\mathbf{H}^{1}_\mathbf{xz}  & \mathbf{H}^{0}_\mathbf{xz} & \ddots         & \vdots           \\ 
		\vdots           & \ddots         & \ddots         & \mathbf{H}^{-1}_\mathbf{xz}   \\
		\mathbf{H}^{Q-1}_\mathbf{xz} & \cdots         & \mathbf{H}^{1}_\mathbf{xz} & \mathbf{H}^{0}_\mathbf{xz}                
	\end{bmatrix}_{N \times N} \: .
	\label{eq:BTTB_Hxz}
\end{equation}
And each diagonal of the blocks are represented by $P \times P$ elements $h^{q}_{p}$, $p = 0, \dots, P - 1$:
\begin{equation}
	\mathbf{H}^{q}_\mathbf{xz} =  \{h^{q}_p\} = \begin{bmatrix}
		h^{q}_{0}   & h^{q}_{1} & \cdots    & h^{q}_{P-1} \\
		h^{q}_{1}   & h^{q}_{0} & \ddots    & \vdots           \\ 
		\vdots      & \ddots    & \ddots    & h^{q}_{1}   \\
		h^{q}_{P-1} & \cdots    & h^{q}_{1} & h^{q}_{0}                 
	\end{bmatrix}_{P \times P} \: .
	\label{eq:Hxz_block}
\end{equation}
In a $x$-\textit{oriented grid} $Q = N_{x}$, $P = N_{y}$ and $q$ and $p$ can be defined by the functions:
\begin{equation}
	q(i, j) = \; l(i) - l(j) 
	\label{eq:Hxz-q-x-oriented}
\end{equation}
and
\begin{equation}
	p(i, j) = \; \mid k(i) - k(j) \mid \quad ,
	\label{eq:Hxz-p-x-oriented}
\end{equation}
where $l(i)$ and $l(j)$ are defined by equation \ref{eq:l-x-oriented} 
and $k(i)$ and $k(j)$ are defined by equation \ref{eq:k-x-oriented}.
For $y$-oriented grids, $Q = N_{x}$, $P = N_{y}$ and the block indices
$q$ and $p$ are defined, respectively, by the following integer functions 
of the matrix indices $i$ and $j$:
\begin{equation}
	q(i, j) = \;  k(i) - k(j)  
	\label{eq:Hxz-q-y-oriented}
\end{equation}
and
\begin{equation}
	p(i, j) = \; \mid l(i) - l(j) \mid \quad ,
	\label{eq:Hxz-p-y-oriented}
\end{equation}
In this case as a skew symmetric matrix by blocks, the values of oposing diagonals blocks have oposing signals, e.g., $\mathbf{H}^{-1}_\mathbf{xz} = -\mathbf{H}^{1}_\mathbf{xz}$ but each block is a symmetric matrix.

%======================================================================================
\subsection{Structure of the components matrices $\mathbf{H_{yz}}$}
%======================================================================================

Finally, by substituting equations \ref{eq:xi} and \ref{eq:yi} in equation \ref{eq:Hi}, we can also describe the elements of $\mathbf{H_{yz}}$, as:
\begin{equation}
	h^{yz}_{ij} = \frac{3 (\Delta l_{ij} \, \Delta y )(\Delta z)}{\left[ 
		\left( \Delta k_{ij} \, \Delta x \right)^{2} + 
		\left( \Delta l_{ij} \, \Delta y \right)^{2} + 
		\left( \Delta z \right)^{2} \right]^{\frac{5}{2}}} \: ,
	\label{eq:hyz_mag}
\end{equation}

The component $\mathbf{H_{yz}}$ (equation \ref{eq:hyz_mag}) of matrix $\mathbf{H}$ (equation \ref{eq:Hi}) are symmetric-Block-Toeplitz skew symmetric-Toeplitz-Block matrices. This means that $\mathbf{H_{yz}}$ is Toeplitz and symmetric by its blocks and each of the blocks are skew symmetric Toeplitz matrices.
Thus, matrix $\mathbf{H_{yz}}$ can be described by the \textit{block indice} $q$ that represent the block diagonals of this matrix as a grid of $Q \times Q$ blocks $\mathbf{H}^{q}_\mathbf{yz}$, $q = 0, \dots, Q - 1$:
\begin{equation}
	\mathbf{H_{yz}} = \begin{bmatrix}
		\mathbf{H}^{0}_\mathbf{yz}  & \mathbf{H}^{1}_\mathbf{yz} & \cdots         & \mathbf{H}^{Q-1}_\mathbf{yz} \\
		\mathbf{H}^{1}_\mathbf{yz}  & \mathbf{H}^{0}_\mathbf{yz} & \ddots         & \vdots           \\ 
		\vdots           & \ddots         & \ddots         & \mathbf{H}^{1}_\mathbf{yz}   \\
		\mathbf{H}^{Q-1}_\mathbf{yz} & \cdots         & \mathbf{H}^{1}_\mathbf{yz} & \mathbf{H}^{0}_\mathbf{yz}                
	\end{bmatrix}_{N \times N} \: .
	\label{eq:BTTB_Hyz}
\end{equation}
And each diagonal of the blocks are represented by $P \times P$ elements $h^{q}_{p}$, $p = -P + 1, \dots, 0, \dots, P - 1$:
\begin{equation}
	\mathbf{H}^{q}_\mathbf{yz} =  \{h^{q}_p\} = \begin{bmatrix}
		h^{q}_{0}   & h^{q}_{-1} & \cdots    & h^{q}_{-P+1} \\
		h^{q}_{1}   & h^{q}_{0} & \ddots    & \vdots           \\ 
		\vdots      & \ddots    & \ddots    & h^{q}_{-1}   \\
		h^{q}_{P-1} & \cdots    & h^{q}_{1} & h^{q}_{0}                 
	\end{bmatrix}_{P \times P} \: .
	\label{eq:Hyz_block}
\end{equation}
In a $x$-\textit{oriented grid} $Q = N_{x}$, $P = N_{y}$ and $q$ and $p$ can be defined by the functions:
\begin{equation}
	q(i, j) = \; \mid l(i) - l(j) \mid
	\label{eq:Hyz-q-x-oriented}
\end{equation}
and
\begin{equation}
	p(i, j) = \; k(i) - k(j) \quad ,
	\label{eq:Hyz-p-x-oriented}
\end{equation}
where $l(i)$ and $l(j)$ are defined by equation \ref{eq:l-x-oriented} 
and $k(i)$ and $k(j)$ are defined by equation \ref{eq:k-x-oriented}.
For $y$-oriented grids, $Q = N_{x}$, $P = N_{y}$ and the block indices
$q$ and $p$ are defined, respectively, by the following integer functions 
of the matrix indices $i$ and $j$:
\begin{equation}
	q(i, j) = \; \mid k(i) - k(j) \mid
	\label{eq:Hyz-q-y-oriented}
\end{equation}
and
\begin{equation}
	p(i, j) = \; l(i) - l(j) \quad ,
	\label{eq:Hyz-p-y-oriented}
\end{equation}
Being a symmetric matrix by blocks, the values of $\mathbf{H_{yz}}$ from oposing diagonals blocks are equal, but each block have skew symmetric oposing diagonals, i.e., $h^{q}_{-1} = - h^{q}_{1}$.

%======================================================================================
\subsection{CGLS matrix-vector substitution}
%======================================================================================

As pointed earlier in this work, the main improvement inside the CGLS method (Algorithm \ref{al:cgls-algorithm}) for estimating the parameter vector $\hat{\mathbf{p}}$ (equation \ref{eq:estimated-p-parameter-space}) is to substitute the matrix-vector multiplication $\mathbf{A}^{\top} \mathbf{s}^{(0)}$ out of the loop and the two matrix-vector multiplications inside the loop at steps 3 an 6, $\mathbf{A} \, \mathbf{c}^{(it)}$ and $\mathbf{A}^{\top} \, \mathbf{s}^{(it + 1)}$, that is necessary at each iteration and takes most of its runtime.

Our method consists in calculating the six first columns of the second derivatives of $\mathbf{H}$ (equation \ref{eq:Hi}) and embbed them into the first six columns of the block-circulant circulant-block (BCCB) matrices related to the $\mathbf{H}$ components. Thus, it is possible to calculate the first column of the BCCB matrix embbeded from matrix $\mathbf{A}$ (equation \ref{eq:aij_mag}) by multiplying each component with its respective constants and summing as shown in equation \ref{eq:aij_mag_expand}. In \cite{takahashi2020convolutional}, Appendix A, the authors demonstrated in details how to transform a symmetric BTTB matrix into a BCCB matrix $\mathbf{C}$. The process here is the same and that work can be referenced to achieve the same results.

A new auxuliary linear system is constructed to carry the matrix-vector product:
\begin{equation}
\mathbf{w} = \mathbf{C} \mathbf{v} \: ,
\label{eq:w_Cv}
\end{equation}
where
\begin{equation}
\mathbf{w} = \begin{bmatrix}
\mathbf{w}_{0} \\
\vdots \\
\mathbf{w}_{Q - 1} \\
\mathbf{0}_{2N \times 1}
\end{bmatrix}_{4N \times 1} \quad ,
\label{eq:w-vector}
\end{equation}
\begin{equation}
\mathbf{w}_{q} = \begin{bmatrix}
\mathbf{d}_{q}(\mathbf{p}) \\
\mathbf{0}_{P \times 1}
\end{bmatrix}_{2P \times 1}
\label{eq:wq-vector} \quad ,
\end{equation}
\begin{equation}
\mathbf{v} = \begin{bmatrix}
\mathbf{v}_{0} \\
\vdots \\
\mathbf{v}_{Q - 1} \\
\mathbf{0}_{2N \times 1}
\end{bmatrix}_{4N \times 1} \quad ,
\label{eq:v-vector}
\end{equation}
and
\begin{equation}
\mathbf{v}_{q} = \begin{bmatrix}
\mathbf{p}_{q} \\
\mathbf{0}_{P \times 1}
\end{bmatrix}_{2P \times 1}
\label{eq:vq-vector} \quad ,
\end{equation}
where $\mathbf{C}$ (equation \ref{eq:w_Cv}) is a $4N \times 4N$ non-symmetric (BCCB) resulted from transforming $\mathbf{A}$ (equation \ref{eq:aij_mag}). Without having to calculate the whole BCCB matrix, its first column can be used to carry the multiplication of this new system (equation \ref{eq:w_Cv}). Appendix A and C in \cite{takahashi2020convolutional} shows how to use the 2D-FFT to compute the eigenvalues of matrix $\mathbf{C}$, store in a $2Q \times 2P$ matrix using the $vec$-operator and to carry the matrix-vector product. Denoting matrix $\mathbf{L}$ as the eigenvalues matrix follows:
\begin{equation}
	\mathbf{F}_{2Q}^{\ast} \left[ 
	\mathbf{L} \circ \left(\mathbf{F}_{2Q} \, \mathbf{V} \, \mathbf{F}_{2P} \right) 
	\right] \mathbf{F}_{2P}^{\ast} = \mathbf{W} \: ,
	\label{eq:DFT-system}
\end{equation}
where the symbol ``$\circ$'' references the Hadamard product, i.e., a element-wise complex multiplication between the eingenvalues and the 2D-FFT of the matrix rearranged along the rows of the parameters $\mathbf{V}$ (equation \ref{eq:v-vector}) using the same $vec$-operator. The resulting inverse 2D-FFT denoted by $\mathbf{F}_{2Q}^{\ast}  \otimes \mathbf{F}_{2P}^{\ast}$ is also a $2Q \times 2P$ matrix ($\mathbf{W}$) that can be rearranged to the predicted data vector $\mathbf{d}(\hat{\mathbf{p}})$ size $N$.

%======================================================================================
\subsection{Computational performance}
%======================================================================================

To compare the efficiency of our algorithm we will use a numerical approach and calculate the floating-point operations (\emph{flops}), i.e., count the number of mathematical operations necessary to complete the estimative of parameter vector $\mathbf{\hat{p}}$ of the normal equations (equation \ref{eq:estimated-p-parameter-space}) and both the CGLS methods (algorithm \ref{al:cgls-algorithm}) for calculating the matrix-vector product by its standart way and our approach.

The \emph{flops} needed to solve the linear system in equation \ref{eq:estimated-p-parameter-space} using the Cholesky factorization is:
\begin{equation}
f_{classical} =  \dfrac{7}{3} N^{3} + 6 N^{2}\: ,
\label{eq:flops-normal-cholesky}
\end{equation}
where $N$ is the total number of observation points and also the size of estimated parameter vector $\mathbf{\hat{p}}$.

For the more efficient CGLS algorithm the estimative can be done in:
\begin{equation}
f_{cgls} =  2 N^{2} + it \, (4 N^{2} + 12 N) \: .
\label{eq:flops-cgls}
\end{equation}
However, our approach reduces further to:
\begin{equation}
f_{ours} =  \kappa  \, 16 N \log_2(4 N) + 24 N + it \, (\kappa  \, 16 N \log_2 (4 N) + 60 N) \: ,
\label{eq:flops-cgls-bccb}
\end{equation}
where $\kappa$ depends on the FFT algorithm. By default, in this work we will use $\kappa = 5$ for the \emph{radix-2} algorithm \citep{vanloan1992}.

Figure \ref{fig:flops} shows a comparative between the methods varying the number of observation points up to $1,000,000$, where it is possible to observe a reduction of $10^7$ orders of magnitude to estimate parameter vector $\mathbf{\hat{p}}$ in relation to the non-iterative classical method and $10^3$ orders of magnitude in relation to the standart CGLS algorithm using $50$ iterations. A more detailed, step by step, flops count of the classical and CGLS algorithm can be found in Appendix A.

In figure \ref{fig:solve_time} we show the time necessary to construct matrix $\mathbf{A}$ (equation \ref{eq:aij_mag}) and solve the linear system up to $10,000$ points of observation. With this dataset the classical method takes more than sixty-three seconds, the CGLS more than twelve seconds, while our method takes only half a second. The cpu used for this test was a intel core i7-7700HQ@2.8GHz.

In figure \ref{fig:sources_time} a comparison between the time to complete the task to calculate the first column of the BCCB matrix embbeded from the from $\mathbf{A}$ (equation \ref{eq:aij_mag}) by using only one equivalent source, i.e., calculating all six first column of the second derivatives matrices from $\mathbf{H}$ (equation \ref{eq:Hi}) and using four equivalent sources to calculate the four necessary columns from the non-symmetric matrix $\mathbf{A}$ (equation \ref{eq:aij_mag}). Although, very similar in time, with one source a small advantage can be observed as the number of data $N$ increases and goes beyond $N = 200,000$. This test was done from $N = 10,000$ to $N = 700,000$ with increases of $5,625$ observation points.

In Table \ref{tab:RAM-usage} there is comparison between how much RAM memory is adressed to store the sensitivity matrix for each of the methods. The classical approach and the CGLS have to store the whole matrix $\mathbf{A}$ (equation \ref{eq:aij_mag}), this means that a dataset  with for example $N = 10,000$ observation points, the sensitivity matrix has $N^2 = 100,000,000$ elements and takes approximately $763$ Megabytes of memory (8 bytes per element). For our method, it is necessary to store the first six columns of each of the components from matrix $\mathbf{H}$ (equation \ref{eq:Hi}) embedded into the BCCB matrices. With the same dataset $N = 10,000$ it needs $1.831$ Megabytes. After completing the steps to store the eigenvalues of matrix $\mathbf{C}$ (equation \ref{eq:w_Cv}) it takes only $0.6104$ Megabytes. Here, we are considering 16 bytes per element as the eigenvalues are complex numbers resulting from the 2D FFT. For a bigger dataset as $N = 1,000,000$ the amount of RAM necessary goes to $7,629,395$, $183.096$ and $61.035$ Megabytes, respectively, showing the necessity to find improved and efficient methods for the equivalent layer technique as the one presented in this work. We remember that throughout our work we are always considering $N = M$.










\section{Application to synthetic data}

Our convolutional equivalent layer method requires a regular data grid located on a 
horizontal and flat observation surface.
Here, we evaluate the performance of our method by applying it to simulated airborne magnetic 
surveys formed by
i) a regular data grid on a flat surface;
ii) irregular data grids on a flat surface; and 
iii) regular data grid on an undulating surfaces.
Note that the simulated surveys in (ii) and (iii) violate the premises of our method. 
The simulated airborne magnetic surveys and the corresponding results 
are illustrated in Figures \ref{fig:synthetic_data_comparison_v2}, 
\ref{fig:synthetic_residuals_convergence_comparison_v2} and 
\ref{fig:synthetic_upward_residuals_comparison_v2}.

The upper and middle rows in Figure \ref{fig:synthetic_data_comparison_v2} show, respectively, 
the simulated flight patterns and noise-corrupted total-field anomalies of the airborne magnetic 
surveys. The lower row in Figure \ref{fig:synthetic_data_comparison_v2} shows the true 
upward-continued total-field anomalies at $z = -1, \, 300$ m.

All magnetic data (middle and lower rows in Figure \ref{fig:synthetic_data_comparison_v2}) 
are produced by the same three synthetic bodies: two prisms and one sphere with 
constant total-magnetization vector having inclination, declination and intensity of 
$0^{\circ}$, $45^{\circ}$, and $2.8284$ A/m, respectively. 
The simulated main geomagnetic field has inclination and declination of $10^{\circ}$ and $37^{\circ}$,
respectively. 

PAREI AQUI

The  three rows in Figure \ref{fig:synthetic_residuals_convergence_comparison_v2}  show the data residuals 
obtained by using the classical method (the upper row), the data residuals obtained by using the
convolutional equivalent layer (the middle row), and the convergence curve of the convolutional equivalent layer (the lower row).

From now on, we call "the convolutional equivalent layer" as "our method" and we use the phrase "data residuals" to define the difference between the observed (middle row in Figure \ref{fig:synthetic_data_comparison_v2}) and the predicted data (not shown) obtained by the classical method or by the convolutional equivalent layer (our method). 

%% Regular grid

Figure \ref{fig:synthetic_data_comparison_v2}a shows a regular grid of  $100 \times 50$ observation points in the $x$- and $y$-directions, totaling  $N = 5,\, 000$ observation points. 
The noise-corrupted total-field anomaly (Figure \ref{fig:synthetic_data_comparison_v2}) is calculated at $900$ m height. 
The data residuals using the classical method (equation \ref{eq:estimated-p-parameter-space}) are shown in the upper panel in Figure \ref{fig:synthetic_residuals_convergence_comparison_v2}a, 
with mean of $0.3627$ nT and standart deviation of $0.2724$ nT.
This process took $17.10$ seconds.
Using our method, the running time to estimate the data residuals (the middle panel in Figure \ref{fig:synthetic_residuals_convergence_comparison_v2}a), with mean of $0.5223$ nT and standart deviation of $0.4323$ nT. The processing time was $0.18$ seconds.
Figure \ref{fig:synthetic_residuals_convergence_comparison_v2}a (lower panel) shows the convergence of our method. The Euclidean norm of the data residuals decreases as expected when the
convergence criterion was satisfied, close to iteration 50. 

This result shows that, in practice, it is not necessary to run the conjugate gradient least square method at $N$ iterations to get an exactly solution.
Actually, the exactly solution  would never occur due to roundoff errors.
Hence, by setting the convergence to  $N$ iterations besides being unnecessary it also demands large computer processing time, even in this synthetic test with a small layer 
($N = 5,\, 000$ equivalent sources). 

%======================================================================================
\subsection*{Tests with data on irregular grids}
%======================================================================================

As shown in the methodology, a regular grid of observation points is needed to arise the BTTB matrix. 
Here, we show the results when our method is applied directly to irregular grids of $N = 5 \, 000$ observation points. 
First, we start from the  regular grid of $100 \times 50$ observation points, shown in Figure 
\ref{fig:synthetic_data_comparison_v2}a,  with a grid spacing of $\Delta x$ of $101.01$ m along the $x$-axis and $\Delta y$ of $163.265$ m along the $y$-axis. 
Next, the $x$- and $y$-coordinates of the observations were also contaminated with additive pseudorandom Gaussian noise with zero mean and standard deviations of $20\%$ and $30\%$ of the $\Delta x$ and $\Delta y$ spacing.

%% irregular grid with 20%  

Figure \ref{fig:synthetic_data_comparison_v2}b (upper panel) shows an irregular grid with uncertainty  of $20\%$ along both $x$- and $y$-directions of the observation points. 
Hence, the $x$- and $y$-coordinates of the observations shown in the regular grid (upper panel in Figure
\ref{fig:synthetic_data_comparison_v2}a)  were corrupted with sequences of pseudorandom Gaussian noise 
having zero means and standard deviations of $20.2$ m and $32.65$ m, respectively.
The noise-corrupted total-field anomaly (middle panel in Figure \ref{fig:synthetic_data_comparison_v2}b) is calculated on this irregular grid and on a flat observation surface at $900$ m height.

Figure \ref{fig:synthetic_residuals_convergence_comparison_v2}b shows that the data residuals using the classical approach (upper panel) yield a good data fit with mean of $0.3630$ nT and standart deviation of $0.2731$ nT. 
Using our method, the data residuals (middle panel in Figure \ref{fig:synthetic_residuals_convergence_comparison_v2}b) also produced an acceptable data fitting with mean of  $0.7147$ nT and standart deviation of $0.5622$ nT. 
The  Euclidean norm of the data residuals obtained by our method 
(Figure \ref{fig:synthetic_residuals_convergence_comparison_v2}b) decreases as expected and close to iteration 50 congerves to a constant value. 


%%  irregular grid with 30%

Figure \ref{fig:synthetic_data_comparison_v2}c (upper panel) shows an irregular grid with uncertainty  of $30\%$ along both $x$- and $y$-directions of the observation points.
Hence, Gaussian pseudorandom noise sequences with zero means and standard deviations of 
$30.3$ m and $48,98$ m were added, respectively, to the $x$- and $y$-coordinates of the observations shown in the regular grid (upper panel in Figure \ref{fig:synthetic_data_comparison_v2}a), producing the simulated irregular grid shown in Figure \ref{fig:synthetic_data_comparison_v2}c.
Figure \ref{fig:synthetic_data_comparison_v2}c shows the noise-corrupted total-field anomaly 
(middle panel) calculated on the irregular grid (upper panel) and on a flat observation surface at $900$ m height.

Figure \ref{fig:synthetic_residuals_convergence_comparison_v2}c shows that the data residuals obtained by the classical approach (upper panel) produced an acceptable data fitting, having mean of $0.3634$ nT and standart deviation of $0.2735$ nT. 
Using our method, the data residuals (middle panel in Figure \ref{fig:synthetic_residuals_convergence_comparison_v2}c) with mean of $0.9788$ nT and standart deviation of $0.7462$ nT produced a poor data fitting.
Figure \ref{fig:synthetic_residuals_convergence_comparison_v2}c shows the convergence analysis of our method.
Similarly to the previous results, in the begining of the iterations, the Euclidean norm of the data residuals obtained by our method decreases; however it starts increasing without achieving an invariance.
Hence, the convergence is not achieved. 


%======================================================================================
\subsection*{Tests with data over an undulating observation surface}
%======================================================================================

Here, we simulate an airborne magnetic survey considering the same regular grid of $100 \times 50$ observation points shown in Figure  \ref{fig:synthetic_data_comparison_v2}a.
However, the observation points were no longer in a plane at $900$ m height, 
but they are over an undulating observation surface.
The next tests, the $z$-coordinates of the observations were contaminated with 
pseudorandom Gaussian noise mean of $- 900$ m and standard deviations of $5\%$ and $10\%$ of the $900$ m height.

%% Synthetic test with data over an undulating observation surface with uncertainty of $5\%$ 

Figure \ref{fig:synthetic_data_comparison_v2}d (upper panel) shows an uneven surface of observations where  the $z$-coordinates of the observations were corrupted with additive  pseudorandom Gaussian noise having mean of  $- 900$ m  and a standard deviation of $45$ m.
The middle panel in Figure \ref{fig:synthetic_data_comparison_v2}d shows the noise-corrupted total-field anomaly calculated on a regular grid  of $100 \times 50$ observation points in the $x-$ and $y-$coordinates 
(upper panel in Figure  \ref{fig:synthetic_data_comparison_v2}a) and
over the undulating observation surface (upper panel in Figure \ref{fig:synthetic_data_comparison_v2}d).
 
The data residuals either using classical approach 
(upper panel in Figure \ref{fig:synthetic_residuals_convergence_comparison_v2}d) or
using our method (middle panel in Figure \ref{fig:synthetic_residuals_convergence_comparison_v2}d) reveal acceptable data fitting.
Using the classical approach, the data residuals have mean of $0.3712$ nT 
and standart deviation of $0.2870$ nT.
Using our method, the data residuals have mean of $0.9542$ nT and standart deviation of $0.8943$ nT. 
Likewise,  Figure \ref{fig:synthetic_residuals_convergence_comparison_v2}d shows that the Euclidean norm of the data residuals, which were obtained by using our method, decreases up to the iteration 50  
and reaches an invariance in the subsequent iterations.  

%% Synthetic test with data over an undulating observation surface with uncertainty of $10\%$ 

The upper panel in Figure \ref{fig:synthetic_data_comparison_v2}e shows an uneven surface of observations where  the $z$-coordinates of the observations were corrupted with additive pseudorandom Gaussian noise having mean of  $- 900$ m  and a standard deviation of $90$ m.
The noise-corrupted total-field anomaly calculated on a regular grid  of $100 \times 50$ observation points in the $x-$ and $y-$coordinates (upper panel in Figure  \ref{fig:synthetic_data_comparison_v2}a) and
over the undulating observation surface (upper panel in Figure \ref{fig:synthetic_data_comparison_v2}e)
is shown in Figure \ref{fig:synthetic_data_comparison_v2}e (middle panel).

By using the classical approach, the upper panel in 
Figure \ref{fig:synthetic_residuals_convergence_comparison_v2}e shows that the data residuals 
yielded a good data fitting, with  mean of $0.3865$ nT and standart deviation of $0.3216$ nT. 
By using our method,  the data residuals (middle panel in 
Figure \ref{fig:synthetic_residuals_convergence_comparison_v2}e) yielded a poor data  fitting with mean of $1.6109$ nT and standart deviation of $1.6231$ nT.
The convergence analysis (Figure\ref{fig:synthetic_residuals_convergence_comparison_v2}e)
reveals the inadequacy of our method in dealing with rugged  surface of observations, as 
the Euclidean norm of the data residuals decreases slower than previous tests. 

Although our method is formulated to deal with magnetic observations measured on a regular grid, in 
the $x$- and $y$-directions, and on a flat surface, the synthetic results show that our method is 
robust in dealing either with irregular grids in the horizontal directions or with uneven surface.
However, the robustness of our method has limitations.
The performance limitation of our method depends on the degree of the 
departure of the $x$- and $y$-coordinates of the data from there corresponding coordinates on a regular grid
and from the amplitude of the undulating observation surface.
High departures of the $x$- and $y$-coordinates  from a regular grid and large variations in the data elevation ($z$-coordinates of the data) are associated with unacceptable data fittings (large data residuals) as shown the middle panels in Figures \ref{fig:synthetic_residuals_convergence_comparison_v2}c and \ref{fig:synthetic_residuals_convergence_comparison_v2}e, respectively.
However, the poor performance of our method in cases of irregular grid and uneven observation surface can be detected easily because, besides it leads to poor data fitting, it does not show an expected convergence as shown the lower panels in Figures \ref{fig:synthetic_residuals_convergence_comparison_v2}c  and \ref{fig:synthetic_residuals_convergence_comparison_v2}e.

%======================================================================================
\subsection*{Magnetic data processing}
%======================================================================================

We performed the upward continuations of the noise-corrupted total-field anomalies 
(second row in Figure \ref{fig:synthetic_data_comparison_v2}) by using 
the classical method, the convolutional equivalent layer method (our method), and 
the classical approach in the Fourier domain.
The noise-free total-field anomalies produced by the synthetic sources at $z = -1, \, 300$ m 
(third row in Figure \ref{fig:synthetic_data_comparison_v2}) are called the true upward-continued total-field anomalies at $z = -1, \, 300$ m.

Figure \ref{fig:synthetic_upward_residuals_comparison_v2} shows the data residuals of the upward-continued total-field anomalies obtained by the classical method (first row), our method (second row) and the classical approach in the Fourier domain (third row).
The upward continuation by using the classical approach in Fourier domain consists in  
computing the Fourier transform of the total-field anomaly \citep[][ p. 317]{blakely1996}. 
From now on, we use the phrase "data residuals of the upward-continued total-field anomalies" to define the difference between the true upward-continued total-field anomalies (third row in Figure \ref{fig:synthetic_data_comparison_v2}) and the predicted upward-continued total-field anomalies (not shown).

%% Classical and our method 

Figure \ref{fig:synthetic_upward_residuals_comparison_v2} shows that the data residuals of the upward-continued total-field anomalies obtained by using the classical method (first row) and our method (second row) are, in most of the tests, similar to each other. 
One exception is the synthetic test with data over an undulating observation surface with uncertainty 
of $10\%$ shown in Figures \ref{fig:synthetic_data_comparison_v2}e and 
\ref{fig:synthetic_residuals_convergence_comparison_v2}e.
We can note that the absolute value of the data residuals of the upward-continued total-field anomalies produced by using our method (middle panel in Figure \ref{fig:synthetic_upward_residuals_comparison_v2}e) 
are $\approx 2.5$ times greater than those produced by the classical method 
(upper panel in Figure \ref{fig:synthetic_upward_residuals_comparison_v2}e).
However, we stress that the simulated undulating observation surface in this test  
(upper panel in Figure \ref{fig:synthetic_data_comparison_v2}e) varies in a broad range from $z = - 570$ m to about $z = -1,\, 230$ m; thus, this simulated airborne magnetic survey greatly violates the requirement 
of a flat observation surface demanded by our method.

%%  Fourier 

In contrast, the data residuals of the upward-continued total-field anomalies obtained by using the 
the classic Fourier approach (third row in Figure \ref{fig:synthetic_upward_residuals_comparison_v2})
are, in most of the tests, approximately  $6$ times greater than those produced by the classical method 
(first  row in Figure \ref{fig:synthetic_upward_residuals_comparison_v2}) and $4$ times greater than those produced by our method (second row in Figure \ref{fig:synthetic_upward_residuals_comparison_v2}).
We can note that the maximum and minimum  values of the the data residuals of the upward-continued total-field anomalies obtained by using the the classic Fourier approach are located at the boundaries of the simulated area.

We call attention to the following aspects. 
In applying the classical method, our method, or the classical Fourier approach, we do not expand the data by using a padding function.
The data residuals (first and second rows in Figure \ref{fig:synthetic_residuals_convergence_comparison_v2})
and the data residuals of the upward-continued total-field anomalies 
(Figure \ref{fig:synthetic_upward_residuals_comparison_v2}) are fully shown  without removing the  edge effects at the borders of the simulated area. 
As pointed out in the methodology, the computational time required by our method is much lower than the one required by the classical method.
However, the computational time required by the classical Fourier approach is the lowest one.
Although, the classical Fourier approach is faster than our method, the upward-continued data exhibit strong border effects if no one padding function to expand the data was applied. 



\section{Application to field data}

We applied the convolutional equivalent layer method to the aeromagnetic data of Carajás, 
northern Brazil.
The survey is composed of $131$ flight lines along north-south direction with line spacing of 
$\Delta y = 3,000$ m. 
Data were measured with spacing $\Delta x = 7.65$ m along lines, with an average distance 
to the ground of $900$ m. % (TEM QUE CHECAR)
The total number of observation points is $N = 6,081,345$. Figure \ref{fig:carajas_real_data_mag} 
shows the observed total-field anomaly data over the study area.

We compare the results obtained with an interpolated regular grid of $10,000 \times 131$ points, 
by using the nearest neighbor algorithm, and a decimated irregular grid, also with $10,000 \times 131$
points, totaling $N = 1,310,000$ observation points in both cases. 
Both application were made with an Intel core i7 7700HQ@2.8GHz processor in single-processing and 
single-threading modes. 
Figures \ref{fig:carajas_real_data_decimated_gridline}a and 
\ref{fig:carajas_real_data_decimated_gridline}b show, respectively, the data obtained by interpolation
and decimation. 
With $1,310,000$ observation points, it would be necessary $12.49$ Terabytes of RAM to store the full
sensitivity matrix with the classical method. 
In this case, our method uses only $59.97$ Megabytes, allowing regular desktop computers to be able 
to process this amount of data.

As the study area is very large, the main magnetic field varies with position.
For this application, we set the main field direction as that of a mid location 
(latitude $-6.5^{\circ}$ and longitude $-50.75^{\circ}$) where the declination is $-19.86^{\circ}$ and
the inclination is $-7.4391^{\circ}$. Both values were calculated using the magnetic field calculator from NOAA
at 1st January, 2014 (epoch of the survey). 
%
%\textbf{Alternative text} $\rightarrow$
%
%As the study area is very large, the main magnetic field varies with position.
%For this application, we set the main field direction as that of a mid location 
%(latitude $-6.5^{\circ}$ and longitude $-50.75^{\circ}$) where the declination is $-19.86^{\circ}$ 
%and inclination is $XXXXXX^{\circ}$ according to IGRF model at 1st January, 2014 (epoch of the survey).
%(É ESTRANHO CALCULAR A DECLINAÇÃO COM UM MODELO E A INCLINAÇÃO COM OUTRO)
%
%$\leftarrow$ \textbf{Alternative text}
%
We set the equivalent layer at $300$ meters above the ground ($600$ m below the data).
By applying our method to the interpolated regular grid 
(Figure \ref{fig:carajas_real_data_decimated_gridline}a), we obtain the  predicted data shown in 
Figure \ref{fig:carajas_gz_predito_mag_gridline}a and data residuals 
(Figure \ref{fig:carajas_gz_predito_mag_gridline}b) with mean $0.0762$ nT and the standard deviation 
$0.4886$ nT, revealing an acceptable data fitting.
Our method took $\approx 390.80$ seconds to converge at about $200$ iterations. % (TEM QUE CHECAR)
%
%POR QUE A CURVA DE CONVERGENCIA NÃO FOI MOSTRADA?
%
By applying our method to the decimated irregular grid 
(Figure \ref{fig:carajas_real_data_decimated_gridline}b), we obtain the predicted data shown in 
Figure \ref{fig:carajas_gz_predito_mag_decimated}a and data residuals 
(Figure \ref{fig:carajas_gz_predito_mag_decimated}b) with mean $0.0717$ nT and standard deviation 
$0.3144$ nT. In this case, our method took $\approx 385.56$ seconds 
to converge at about $200$ iterations (Figure \ref{fig:convergence_carajas_mag_decimated}).
%(TEM ALGO ESTRANHO AQUI. COMO QUE 2000 ITERAÇÕES FOI MAIS MAIS RÁPIDO DO QUE AS 200 ITERAÇÕES DO GRID INTERPOLADO?).
The convergence curve reveals a good convergence rate obtained with the decimated 
irregular grid. This result shows the robustness of our method in processing irregular grid.

%We have found, in applying our method to the decimated irregular grid, that the data residual amplitude (Figure \ref{fig:carajas_gz_predito_mag_decimated}b) is lower than the data residual amplitude (Figure 
%\ref{fig:carajas_gz_predito_mag_gridline}b) obtained by applying our method to the interpolated regular grid.
%It occurs because  the  process of decimating the original irregularly data creates neither new observation points nor new data. Rather, the interpolation of the original irregularly data creates either new observation points or new data. 

Finally, Figure \ref{fig:up5000_carajas_decimated_mag} shows the upward-continued magnetic data to a
horizontal plane located at $5, \,000$ m using the estimated equivalent layer obtained by applying our
method to the decimated irregular grid (Figure \ref{fig:carajas_real_data_decimated_gridline}b).
This process took $\approx 2.64$ seconds, showing good results without visible errors or border 
effects.
\section{Conclusions}

In this work, we were able to develop a fast equivalent layer technique for processing magnetic data with the method of Conjugate Gradient Least Square using the convolutional equivalent layer theory to obtain results of performance more than four orders of magnitude less than the classical equivalent layer. The sensitivity matrix of the magnetic equivalent layer carries the structure of BTTB matrices, which means a very low computational cost matrix-vector product and also the possibility to store only the first column of the matrix BCCB. In this work we propose a novel method to use only one equivalent source and calculating the first six columns of the inverse of distance second derivatives matrices to arrive in the first column of the BCCB matrix embbeded from the original magnetic kernel sensitivity matrix.

Synthetic tests showed similar results estimating the physical property using a classical approach to solve a linear system and our method using the CGLS combined with the BTTB matrix-vector product. The difference in time, however, is noticeable: $2.04$ seconds using the classical approach and $0.083$ seconds using our approach. This difference was obtained with a mid-size mesh of $80 \times 80$ points, greater results can be obtained if more observation points are used.

Real data test were also conducted in the region of Carajás, Pará, Brazil. With an irregular grid of $1,310,000$ observation points, store the full sensitivity matrix it would be necessary $12.49$ Terabytes of RAM.  However, taking advantage of the symmetric or skew-symmetric matrices structures, it is possible to reconstruct the whole sensitivity matrix using only $59.97$ Megabytes.
Using 200 iterations of the CGLS method took $385.56$ seconds and very good results of property estimative were obtained. Also the upward-continuation transformation showed good results and took only $2.64$ seconds.
\section{Acknowledgements}

Diego Takahashi was supported by a Phd scholarship from CAPES. 
Val{\'e}ria C.F. Barbosa was supported by fellowships from CNPq (grant 307135/2014-4) 
and FAPERJ (grant 26/202.582/2019). Vanderlei C. Oliveira Jr. was supported 
by fellowships from CNPq (grant 308945/2017-4) and FAPERJ (grant E-26/202.729/2018). 
The authors thank the Geological Survey of Brazil (CPRM) for providing the field data.

\clearpage

% Tables and figures
\renewcommand{\figdir}{Fig} % figure directory

%% Methodology
\plot{Figure1}{width=\textwidth}{
	{Schematic representation of an $N_{x} \times N_{y}$ regular grid of points (black dots) defined by 
	$N_{x} = 4$ and $N_{y} = 3$. The grids are oriented along the (a) $x$-axis and (b) $y$-axis. The grid 
	coordinates are $x_{k}$ and $y_{l}$, where the $k = 1, \dots, N_{x}$ and $l = 1, \dots, N_{y}$ are 
	called the grid indices. The insets show the grid indices $k$ and $l$.}
	\label{fig:methodology}
}

\plot{4_equivalent_sources}{width=\textwidth}{
	{Representation of the four equivalent sources (black dots) needed to reconstruct the non-symmetric matrix $\mathbf{A}$ (equation \ref{eq:aij_mag}). Each of the equivalent sources are located in the corner of the simulated regular grid of $M_x = 4$ and $M_y = 3$ sources. The influence of these sources on each of the observation points (blue dots) i the regular grid of $N_x = 4$ and $N_y = 3$ will give the four columns necessary of matrix $\mathbf{A}$.}
	\label{fig:4_equivalent_sources}
}

%% Computational performance

\plot{flops_mag}{width=\textwidth}{
	{Number of flops necessary to estimate the parameter vector $\mathbf{\hat{p}}$ using the non-iterative classical method (equation \ref{eq:flops-normal-cholesky}) the CGLS (equation \ref{eq:flops-cgls}) and our modified CGLS method (equation \ref{eq:flops-cgls-bccb}) with $N^{it} = 50$. 
	The observation point $N$ varied from $5,000$ to $1,000,000$. The \emph{radix-2} 2D FFT
	algorithm was considered for our method, with $\kappa = 5$.}
	\label{fig:flops}
}

\plot{time_comparison_mag}{width=\textwidth}{
	{Comparison between the runtime of the equivalent-layer technique using the classical, the CGLS algorithm and our method. The values for the CGLS and our methods were obtained for $N^{it} = 50$ iterations.}
	\label{fig:solve_time}
}

\plot{time_sources_mag}{width=\textwidth}{
	{Comparison between the runtime to calculate the first column of the BCCB matrix embbeded from $\mathbf{A}$ (equation \ref{eq:aij_mag}) using only one and using four equivalent sources. Although the time is very similar, with one source a small advantage can be observed as the number of data $N$ increases. This test was done from $N = 10,000$ to $N = 700,000$ with increases of $5,625$ observation points.}
	\label{fig:sources_time}
}

%% Synthetic data part I

\plot{synthetic_data_comparison}{width=\textwidth}{
{Synthetic tests: the simulated airborne magnetic surveys  - 
The first column shows the grids of observation points or the undulating observation surfaces that simulate airborne magnetic surveys. 
The second column shows the noise-corrupted total-field anomaly produced by the synthetic sources and calculated on the simulate airborne magnetic survey shown in the first column. 
The third column shows the  noise-free total-field anomaly produced by the synthetic sources 
at $z = −1,300$ m.
The results shown in these three columns were obtained by using the simulated airborne magnetic surveys as follows: 
(a)-(c) A regular grid of $100 \times 50$ observation points in the $x-$ and $y-$directions and a flat observation surface at $900$ m height. 
(d)-(f) A irregular grid with uncertainty of $20\%$  in the $x-$ and $y-$coordinates and a flat observation surface at $900$ m height. 
(g)-(i) A irregular grid with uncertainty of $30\%$  in the $x-$ and $y-$coordinates and a flat observation surface at $900$ m height. 
(j)-(l) A regular grid  of $100 \times 50$ observation points in the $x-$ and $y-$coordinates and an undulating observation surface with uncertainty of $5\%$. 
(m)-(o) A regular grid  of $100 \times 50$ observation points in the $x-$ and $y-$coordinates and an undulating observation surface with uncertainty of $10\%$. 
The black lines represent the horizontal projection of the sources
.}
\label{fig:synthetic_data_comparison}
}

\plot{synthetic_residuals_convergence_comparison}{width=\textwidth}{
{Synthetic tests: the data residuals and convergence - 
The first column shows the data residuals using the classical method.
The second and third columns show, respectively, the data residuals and the convergence curves using the convolutional equivalent layer (our method).
The results shown in these three columns were obtained by using the simulated airborne magnetic surveys shown in Figure \ref{fig:synthetic_data_comparison}, i.e.:
(a)-(c) A regular grid of $100 \times 50$ observation points in the $x-$ and $y-$directions and a flat observation surface at $900$ m height. 
(d)-(f) A irregular grid with uncertainty of $20\%$  in the $x-$ and $y-$coordinates and a flat observation surface at $900$ m height. 
(g)-(i) A irregular grid with uncertainty of $30\%$  in the $x-$ and $y-$coordinates and a flat observation surface at $900$ m height. 
(j)-(l) A regular grid  of $100 \times 50$ observation points in the $x-$ and $y-$coordinates and an undulating observation surface with uncertainty of $5\%$. 
(m)-(o) A regular grid  of $100 \times 50$ observation points in the $x-$ and $y-$coordinates and an undulating observation surface with uncertainty of $10\%$. 
The black lines represent the horizontal projection of the sources
.}
\label{fig:synthetic_residuals_convergence_comparison}
}


\plot{synthetic_upward_residuals_comparison}{width=\textwidth}{
{Synthetic tests: the data residuals of the upward-continued total-field anomalies (second column 
of the Figure \ref{fig:synthetic_data_comparison}).
The data residuals of the upward-continued total-field anomalies are defined as the difference between 
the noise-free total-field anomaly produced by the synthetic sources at $z = −1,300$ m 
(third  column of the Figure \ref{fig:synthetic_data_comparison}) and
the predicted total-field anomaly at $z = −1,300$ m  obtained by using three methods:
the classical method (first column); the convolutional equivalent layer (second column); and 
the classic approach in the Fourier domain (third column).
The results shown in these three columns were obtained by using the simulated airborne magnetic surveys shown in Figure \ref{fig:synthetic_data_comparison}, i.e.:
(a)-(c) A regular grid of $100 \times 50$ observation points in the $x-$ and $y-$directions and a flat observation surface at $900$ m height. 
(d)-(f) A irregular grid with uncertainty of $20\%$  in the $x-$ and $y-$coordinates and a flat observation surface at $900$ m height. 
(g)-(i) A irregular grid with uncertainty of $30\%$  in the $x-$ and $y-$coordinates and a flat observation surface at $900$ m height. 
(j)-(l) A regular grid  of $100 \times 50$ observation points in the $x-$ and $y-$coordinates and an undulating observation surface with uncertainty of $5\%$. 
(m)-(o) A regular grid  of $100 \times 50$ observation points in the $x-$ and $y-$coordinates and an undulating observation surface with uncertainty of $10\%$. 
The black lines represent the horizontal projection of the sources
.}
\label{fig:synthetic_upward_residuals_comparison}
}

%% Field Data

\plot{carajas_real_data_mag}{width=\textwidth}{
	{Observed magnetic field data of the Carajás, Brazil area. The aeromagnetic survey was done with $131$ N-S lines at approximately $-900 m$ height, totaling $N = 6,081,345$ observation points.}
	\label{fig:carajas_real_data_mag}
}

\plot{carajas_real_data_decimated_gridline}{width=8cm}{
	{(a) Observed magnetic field data of the Carajás, Brazil area, interpolated for a regular grid of $10,000 \times 131$, totaling $N = 1,310,000$ observation points. (b) Observed magnetic field data of the Carajás, Brazil area, decimated from the flight lines resulting in an irregular grid of $10,000 \times 131$, also totaling $N = 1,310,000$ observation points.}
	\label{fig:carajas_real_data_decimated_gridline}
}

\plot{carajas_tf_predicted_gridline}{width=8cm}{
	{(a) Predicted data using our method for the interpolated $10,000 \times 131$ regular grid. (b) Residuals between the observed (\ref{fig:carajas_real_data_decimated_gridline}) and the predicted data (panel a), with mean of $0.07979$ nT and standart deviation of $0.5060$ nT.}
	\label{fig:carajas_gz_predito_mag_gridline}
}

\plot{carajas_tf_predicted_decimated}{width=8cm}{
	{(a) Predicted data using our method for the decimated $10,000 \times 131$ irregular grid. (b) Residuals between the observed (\ref{fig:carajas_real_data_decimated_gridline}) and the predicted data (panel b), with mean of $0.07348$ nT and standart deviation of $0.3172$ nT.}
	\label{fig:carajas_gz_predito_mag_decimated}
}

\plot{convergence_carajas_mag_decimated}{width=\textwidth}{
	{Convergence analysis of the CGLS method for the field data of Carajás, Brazil using the magnetic equivalent layer with a decimated irregular grid of $10,000 \times 131$ observation points up to 2,000 iterations.}
	\label{fig:convergence_carajas_mag_decimated}
}

\plot{up5000_carajas_mag_decimated}{width=\textwidth}{
	{Upward continuation transformation of real data of Carajás, Brazil at $5,000$ meter. It was necessary $2.64$ seconds to complete the process.}
	\label{fig:up5000_carajas_decimated_mag}
}

\clearpage

% Appendices
%\appendix
\append{Flops computations}

%======================================================================================
\subsection{Classical flops count}
%======================================================================================

The flops count of the classical approach to solve the linear system (equation \ref{eq:estimated-p-parameter-space}) using the Cholesky factorization is given by equation \ref{eq:flops-normal-cholesky}. The step-by-step count follows:
\begin{itemize}
\item[\textbf{(1)}] $\mathbf{A}^{\top}\mathbf{A}$: $2 N^3$ (one matrix-matrix product).

\item[\textbf{(2)}] $\mathbf{A}^{\top} \mathbf{A}$: $\dfrac{1}{3} N^3$ (one Cholesky factorization $\mathbf{C_f}$).

\item[\textbf{(3)}] $\mathbf{A}^{\top} \mathbf{d}^{o}$: $2 N^2$ (one matrix-vector product).

\item[\textbf{(4)}] $\mathbf{C_f} (\mathbf{A}^{\top} \mathbf{d}^{o})$: $2 N^2$ (one matrix-vector product).

\item[\textbf{(5)}] $\mathbf{C_f}^{\top} (\mathbf{C_f} \mathbf{A}^{\top} \mathbf{d}^{o})$: $2 N^2$ (one matrix-vector product).
\end{itemize}
Summing all calculations: 
\begin{equation}
f_{classical} =  \dfrac{7}{3} N^{3} + 6 N^{2}\: ,
\label{eq:flops-normal-cholesky-append}
\end{equation}

%======================================================================================
\subsection{CGLS flops count}
%======================================================================================

The flops count of CGLS algorithm \ref{al:cgls-algorithm} can be summarized as:

Out of the loop:

\begin{itemize}
%\item[\textbf{(1)}] $\mathbf{d}^{o} - \mathbf{A} \hat{\mathbf{p}}^{(0)}$: $2 N^2 + N$ (one matrix-vector product %and one vector subtraction)

\item[\textbf{(1)}] $\mathbf{A}^{\top} \mathbf{s}$: $2 N^2$ (one matrix-vector product).
\end{itemize}

Inside the loop:

\begin{itemize}
\item[\textbf{(1)}] $\dfrac{{\mathbf{r}^{(it)}}^{\top} \, \mathbf{r}^{(it)}} {{\mathbf{r}^{(it - 1)}}^{\top} \, \mathbf{r}^{(it - 1)}}$: $4 N$ (two vector-vector products).

\item[\textbf{(2)}] $\mathbf{r}^{it} - \alpha_{it} \,\beta_{it} \, \mathbf{c}^{(it - 1)}$: $2 N$ (one scalar-vector product and one vector subtraction).

\item[\textbf{(3)}] $\frac{{||\mathbf{r}^{(it)}||_2}^2}{({\mathbf{c}^{(it)}}^{\top} \, \mathbf{A}^{\top})(\mathbf{A} \, \mathbf{c}^{(it)})}$: $2 N^2 + 2N$ (one matrix-vector and one vector-vector product).

\item[\textbf{(4)}] $\hat{\mathbf{p}}^{it} - \alpha_{it} \, \mathbf{c}^{(it)}$: $2 N$ (one vector subtraction).

\item[\textbf{(5)}] $\mathbf{s}^{it} - \alpha_{it} \, \mathbf{A} \, \mathbf{c}^{(it)}$: $2 N$ (one vector subtraction, the matrix-vector product was calculated in step 3).

\item[\textbf{(6)}] $ \mathbf{A}^{\top} \, \mathbf{s}^{(it + 1)}$: $2 N^2$ (one matrix-vector product).
\end{itemize}
The result of all flops count leads to:
\begin{equation}
f_{cgls} =  2 N^{2} + it \, (4 N^{2} + 12 N) \: .
\label{eq:flops-cgls-append}
\end{equation}

%======================================================================================
\subsection{Our modified CGLS flops count}
%======================================================================================

All the flops count presented in previous section for the CGLS remains, only substituting the  out of the loop matrix-vector product in step 1 and the two matrix-vector products inside the loop in steps 3 and 6.
The computations necessary to carry the matrix-vector used in this work are given by:

\begin{itemize}
\item[\textbf{(1)}] $\mathbf{L}$: $\kappa  \, 4 N \log_2(4N)$ (one 2D FFT for the eigenvalues calculation of the sensitivity matrix $\mathbf{A}$ or the transposed sensitivity matrix $\mathbf{A}^{\top}$).

\item[\textbf{(2)}] $\mathbf{F}_{2Q} \, \mathbf{V} \, \mathbf{F}_{2P}$: $\kappa  \, 4 N \log_2(4N)$ (one 2D FFT).

\item[\textbf{(3)}] $\mathbf{L} \circ \left(\mathbf{F}_{2Q} \, \mathbf{V} \, \mathbf{F}_{2P} \right)$: $24 N$ (one complex Hadamard matrix multiplication).

\item[\textbf{(4)}] $\mathbf{F}_{2Q}^{\ast} \left[ 
\mathbf{L} \circ \left(\mathbf{F}_{2Q} \, \mathbf{V} \, \mathbf{F}_{2P} \right) 
\right] \mathbf{F}_{2P}^{\ast}$: $\kappa  \, 4 N \log_2(4N)$ (one inverse 2D FFT).
\end{itemize}
Matrix-vector product total:  $\kappa  \, 12 N \log_2(4 N) + 24 N$.

As matrix $\mathbf{A}$ (equation \ref{eq:aij_mag}) and its transposed never changes, it is not necessary to calculate the eigenvalues inside the loop at each iteration, we are considering that both are calculated out of the loop. Inside the loop, steps 2 to 4 are repeated two times per iteration. Substituting this result into the CGLS flops count (equation \ref{eq:flops-cgls-append}) leads to:
\begin{equation}
f_{ours} =  \kappa  \, 16 N \log_2(4 N) + 24 N + it \, (\kappa  \, 16 N \log_2 (4 N) + 60 N).
\label{eq:flops-cgls-bccb-append}
\end{equation}

%\append{Matrix $\mathbf{C}$}


This appendix illustrates the matrix $\mathbf{C}$ (equation \ref{eq:w_Cv}) 
obtained with the $x$- and $y$-oriented grids illustrated in Figure \ref{fig:methodology}
and also presents some of its relevant properties.

Matrix $\mathbf{C}$ (equation \ref{eq:w_Cv})
is circulant blockwise, formed by $2Q \times 2Q$ blocks, where
each block $\mathbf{C}_{q}$, $q = 0, \dots, Q-1$, is a $2P \times 2P$ circulant matrix. 
Similarly to the BTTB matrix $\mathbf{A}$ (equations \ref{eq:BTTB_A} and 
\ref{eq:A-x-oriented-example}--\ref{eq:Aq-y-oriented}), the index $q$ 
varies from $0$ to $Q - 1$. Additionally, the blocks lying 
above the main diagonal are equal to those located below.

It is well-known that a circulant matrix can be defined by properly downshifting 
its first column \citep[][ p. 206]{vanloan1992}. Hence, the BCCB matrix $\mathbf{C}$ 
(equation \ref{eq:w_Cv}) can be obtained from its 
first column of blocks, which is given by
\begin{equation}
\left[\mathbf{C} \right]_{(0)} = 
\begin{bmatrix}
\mathbf{C}_{0} \\
\vdots \\
\mathbf{C}_{Q-1} \\
\mathbf{0} \\
\mathbf{C}_{Q-1} \\
\vdots \\
\mathbf{C}_{1}
\end{bmatrix}_{4N \times 2P} \: ,
\label{eq:C-first-column-blocks}
\end{equation}
where $\mathbf{0}$ is a $2P \times 2P$ matrix of zeros. Similarly, each block 
$\mathbf{C}_{q}$, $q = 0, \dots, Q-1$, can be obtained by downshifting its first 
column
\begin{equation}
\mathbf{c}^{q}_{0} = 
\begin{bmatrix}
a^{q}_{0} \\
\vdots \\
a^{q}_{P-1} \\
0 \\
a^{q}_{P-1} \\
\vdots \\
a^{q}_{1}
\end{bmatrix}_{2P \times 1} \: ,
\label{eq:Cq-first-column}
\end{equation}
where $a^{q}_{p}$ (equation \ref{eq:aqp_equiv_aij}), $p = 0, \dots, P-1$, are the elements 
forming the block $\mathbf{A}_{q}$ (equations \ref{eq:Aq_block} and 
\ref{eq:A-x-oriented-example}--\ref{eq:Aq-y-oriented}).
The downshift can be thought off as permutation that pushes the components of a column vector 
down one notch with wraparound \citep[][ p. 20]{golub-vanloan2013}.
To illustrate this operation, consider our $y$-oriented grid illustrated in Figure \ref{fig:methodology}b. 
In this case, the resulting 
BCCB matrix $\mathbf{C}$ (equation \ref{eq:w_Cv}) is given by 
\begin{equation}
\mathbf{C} =
\begin{bmatrix}
\mathbf{C_{0}} & \mathbf{C_{1}} & \mathbf{C_{2}} & \mathbf{C_{3}} & \mathbf{0}     & \mathbf{C_{3}} & \mathbf{C_{2}} & \mathbf{C_{1}} \\
\mathbf{C_{1}} & \mathbf{C_{0}} & \mathbf{C_{1}} & \mathbf{C_{2}} & \mathbf{C_{3}} & \mathbf{0}     & \mathbf{C_{3}} & \mathbf{C_{2}} \\
\mathbf{C_{2}} & \mathbf{C_{1}} & \mathbf{C_{0}} & \mathbf{C_{1}} & \mathbf{C_{2}} & \mathbf{C_{3}} & \mathbf{0}     & \mathbf{C_{3}} \\
\mathbf{C_{3}} & \mathbf{C_{2}} & \mathbf{C_{1}} & \mathbf{C_{0}} & \mathbf{C_{1}} & \mathbf{C_{2}} & \mathbf{C_{3}} & \mathbf{0}     \\
\mathbf{0}     & \mathbf{C_{3}} & \mathbf{C_{2}} & \mathbf{C_{1}} & \mathbf{C_{0}} & \mathbf{C_{1}} & \mathbf{C_{2}} & \mathbf{C_{3}} \\
\mathbf{C_{3}} & \mathbf{0}     & \mathbf{C_{3}} & \mathbf{C_{2}} & \mathbf{C_{1}} & \mathbf{C_{0}} & \mathbf{C_{1}} & \mathbf{C_{2}} \\
\mathbf{C_{2}} & \mathbf{C_{3}} & \mathbf{0}     & \mathbf{C_{3}} & \mathbf{C_{2}} & \mathbf{C_{1}} & \mathbf{C_{0}} & \mathbf{C_{1}} \\
\mathbf{C_{1}} & \mathbf{C_{2}} & \mathbf{C_{3}} & \mathbf{0}     & \mathbf{C_{3}} & \mathbf{C_{2}} & \mathbf{C_{1}} & \mathbf{C_{0}}
\end{bmatrix}_{4N \times 4N},
\label{eq:C-y-oriented}
\end{equation}
where each block $\mathbf{C}_{q}$, $q = 0, 1, 2, 3$, is represented as follows 
\begin{equation}
\tensor{C}_{q} =
\begin{bmatrix}
a^{q}_{0} & a^{q}_{1} & a^{q}_{2} & 0         & a^{q}_{2} & a^{q}_{1} \\
a^{q}_{1} & a^{q}_{0} & a^{q}_{1} & a^{q}_{2} & 0         & a^{q}_{2} \\
a^{q}_{2} & a^{q}_{1} & a^{q}_{0} & a^{q}_{1} & a^{q}_{2} & 0         \\
0         & a^{q}_{2} & a^{q}_{1} & a^{q}_{0} & a^{q}_{1} & a^{q}_{2} \\
a^{q}_{2} & 0         & a^{q}_{2} & a^{q}_{0} & a^{q}_{0} & a^{q}_{1} \\
a^{q}_{1} & a^{q}_{2} & 0         & a^{q}_{2} & a^{q}_{1} & a^{q}_{0}
\end{bmatrix}_{2P \times 2P}
\label{eq:Cq-y-oriented}
\end{equation}
in terms of the block elements $a^{q}_{p}$ (equation \ref{eq:aqp_equiv_aij}).
Similar matrices are obtained for our $x$-oriented grid illustrated in Figure \ref{fig:methodology}a.

BCCB matrices are diagonalized by the 2D unitary DFT 
\citep[][ p. 185]{davis1979}. It means that $\mathbf{C}$ (equation \ref{eq:w_Cv}) 
satisfies 
\begin{equation}
\mathbf{C} = 
\left(\mathbf{F}_{2Q} \otimes \mathbf{F}_{2P} \right)^{\ast} 
\boldsymbol{\Lambda}
\left(\mathbf{F}_{2Q} \otimes \mathbf{F}_{2P} \right) \: ,
\label{eq:C-diagonalized}
\end{equation}
where the symbol ``$\otimes$" denotes the Kronecker product \citep{neudecker1969},
$\mathbf{F}_{2Q}$ and $\mathbf{F}_{2P}$ are the $2Q \times 2Q$ and $2P \times 2P$ 
unitary DFT matrices \citep[][ p. 31]{davis1979}, respectively, the superscritpt 
``$\ast$" denotes the complex conjugate and $\boldsymbol{\Lambda}$ is a 
$4QP \times 4QP$ diagonal matrix containing the eigenvalues of $\mathbf{C}$.
%\append{Computations with the 2D DFT}


In the present Appendix, we deduce equation \ref{eq:DFT-system}
by using the row-ordered $vec$-operator (here designated simply as $vec$-operator).
This equation can be efficiently computed by using the 2D 
fast Fourier Transform. 
This operator was implicitly used by \citet[][ p. 31]{jain1989} to 
show the relationship between Kronecker products and separable 
transformations. The $vec$-operator defined here 
transforms a matrix into a column vector by stacking its rows. 

Let $\mathbf{M}$ be an arbitrary $N \times M$ matrix given by:
\begin{equation}
\mathbf{M} = \begin{bmatrix}
\mathbf{m}^{\top}_{1} \\ 
\vdots \\
\mathbf{m}^{\top}_{N}
\end{bmatrix} \: ,
\label{eq:matrix-M}
\end{equation}
where $\mathbf{m}_{i}$, $i = 1, \dots, N$, are $M \times 1$ vectors containing 
the rows of $\mathbf{M}$.
The elements of this matrix can be rearranged into a column vector by using the
$vec$-operator \citep[][ p. 31]{jain1989} as follows:
\begin{equation}
vec \left( \mathbf{M} \right) = \begin{bmatrix}
\mathbf{m}_{1} \\
\vdots \\
\mathbf{m}_{N}
\end{bmatrix}_{NM \times 1} \: .
\label{eq:vec-operator}
\end{equation}
This rearrangement is known as lexicographic ordering \citep[][ p. 150]{jain1989}.

Two important properties of the $vec$-operator (equation \ref{eq:vec-operator}) 
are necessary to us. 
To define the first one, consider an 
$N \times M$ matrix $\mathbf{H}$ given by
\begin{equation}
\mathbf{H} = \mathbf{P} \circ \mathbf{Q} \: ,
\label{eq:matrix-H}
\end{equation}
where $\mathbf{P}$ and $\mathbf{Q}$ are arbitrary $N \times M$ matrices and 
``$\circ$" represents the Hadamard product \citep[][ p. 298]{horn_johnson1991}.
By applying the $vec$-operator to $\mathbf{H}$ (equation \ref{eq:matrix-H}), 
it can be shown that
\begin{equation}
vec \left( \mathbf{H} \right) = 
vec \left( \mathbf{P} \right) \circ vec \left( \mathbf{Q} \right) \: .
\label{eq:vec-matrix-H}
\end{equation}
To define the second important property of $vec$-operator, 
consider an $N \times M$ matrix $\mathbf{S}$ defined by 
the separable transformation \citet[][ p. 31]{jain1989}:
\begin{equation}
\mathbf{S} = \mathbf{P \, M \, Q} \: ,
\label{eq:matrix-S}
\end{equation}
where $\mathbf{P}$ and $\mathbf{Q}$ are arbitrary $N \times N$ and $M \times M$ 
matrices, respectively.
By implicitly applying the $vec$-operator to 
the $\mathbf{S}$ (equation \ref{eq:matrix-S}), 
\citet[][ p. 31]{jain1989} show that:
\begin{equation}
vec \left( \mathbf{S} \right) = 
\left( \mathbf{P} \otimes \mathbf{Q}^{\top} \right) 
vec \left( \mathbf{M} \right) \: ,
\label{eq:vec-matrix-S}
\end{equation}
where ``$\otimes$" denotes the Kronecker product \citep{neudecker1969}.
It is important to stress the difference between equation \ref{eq:vec-matrix-S}
and that presented by \citet{neudecker1969}, which is more commonly found in 
the literature.
While that equation uses a $vec$-operator that transforms a matrix into a column 
vector by stacking its columns, equation \ref{eq:vec-matrix-S} 
uses the $vec$-operator defined by equation \ref{eq:vec-operator}, which 
transforms a matrix into a column vector by stacking its rows.

Now, let us deduce equation \ref{eq:DFT-system} by 
using the above-defined properties (equation \ref{eq:vec-matrix-H}
and \ref{eq:vec-matrix-S}).
We start calling attention to the right side of equation \ref{eq:vec-DFT-system}.
Consider that vector $\mathbf{w}$ (equation \ref{eq:vec-DFT-system}) 
is obtained by applying the $vec$-operator (equation \ref{eq:vec-operator}) to a matrix 
$\mathbf{W}$, whose 2D DFT $\tilde{\mathbf{W}}$ is represented by the 
following separable transformation \citep[][ p. 146]{jain1989}:
\begin{equation}
\tilde{\mathbf{W}} = \mathbf{F}_{2Q} \, \mathbf{W} \, \mathbf{F}_{2P} \: ,
\label{eq:2D-DFT-W}
\end{equation}
where $\mathbf{F}_{2Q}$ and $\mathbf{F}_{2P}$ are the $2Q \times 2Q$ and $2P \times 2P$ 
unitary DFT matrices. 
Using equation \ref{eq:vec-matrix-S} and the symmetry of unitary DFT 
matrices, we rewrite the right side of equation \ref{eq:vec-DFT-system} 
as follows:
\begin{equation}
vec \left( \tilde{\mathbf{W}} \right) = 
\left( \mathbf{F}_{2Q} \otimes \mathbf{F}_{2P} \right) 
vec \left( \mathbf{W} \right) \: .
\label{eq:right_side_DFT_system_1}
\end{equation}
Similarly, consider that $\mathbf{v}$ (equation \ref{eq:vec-DFT-system}) 
is obtained by applying the $vec$-operator (equation \ref{eq:vec-operator}) to a matrix 
$\mathbf{V}$, whose 2D DFT (equation \ref{eq:2D-DFT-W}) is 
represented by $\tilde{\mathbf{V}}$. Using equation \ref{eq:vec-matrix-S} and the symmetry 
of unitary DFT matrices, we can rewrite the 
left side of equation \ref{eq:vec-DFT-system} as follows:
\begin{equation}
\boldsymbol{\Lambda} \, vec \left( \tilde{\mathbf{V}} \right) = 
\boldsymbol{\Lambda}
\left( \mathbf{F}_{2Q} \otimes \mathbf{F}_{2P} \right) 
vec \left( \mathbf{V} \right) \: .
\label{eq:left_side_DFT_system_1}
\end{equation}
Note that both sides of equation \ref{eq:left_side_DFT_system_1}
are defined as the product of the diagonal matrix $\boldsymbol{\Lambda}$ (equation \ref{eq:C-diagonalized}) 
and a vector. In this case, the matrix-vector product can be conveniently replaced by
\begin{equation}
\boldsymbol{\lambda} \circ vec \left( \tilde{\mathbf{V}} \right) = 
\boldsymbol{\lambda} \circ
\left( \mathbf{F}_{2Q} \otimes \mathbf{F}_{2P} \right) 
vec \left( \mathbf{V} \right) \: ,
\label{eq:left_side_DFT_system_2}
\end{equation}
where $\boldsymbol{\lambda}$ is a $4QP \times 1$ vector containing the diagonal of 
$\boldsymbol{\Lambda}$ (equation \ref{eq:C-diagonalized}).
Then, consider that $\boldsymbol{\lambda}$ is obtained by applying the $vec$-operator 
(equation \ref{eq:vec-operator}) to a $2Q \times 2P$ matrix $\mathbf{L}$, we can use 
equations \ref{eq:vec-matrix-H} and \ref{eq:vec-matrix-S} to rewrite equation 
\ref{eq:left_side_DFT_system_2} as follows:
\begin{equation}
vec \left( \mathbf{L} \circ \tilde{\mathbf{V}} \right) = 
vec \left[ \mathbf{L} \circ 
\left( \mathbf{F}_{2Q} \, \mathbf{V} \, \mathbf{F}_{2P} \right) 
\right] \: .
\label{eq:left_side_DFT_system_3}
\end{equation}
Equations \ref{eq:2D-DFT-W}, \ref{eq:right_side_DFT_system_1} and 
\ref{eq:left_side_DFT_system_3} show that equation \ref{eq:vec-DFT-system}
is obtained by applying the $vec$-operator to 
\begin{equation}
\mathbf{L} \circ \left( \mathbf{F}_{2Q} \, \mathbf{V} \, \mathbf{F}_{2P} \right) = 
\mathbf{F}_{2Q} \, \mathbf{W} \, \mathbf{F}_{2P} \: .
\label{eq:DFT-system-preliminary}
\end{equation}
Finally, we premultiply both sides of equation \ref{eq:DFT-system-preliminary} by 
$\mathbf{F}_{2Q}^{\ast}$ and then postmultiply both sides of the result by 
$\mathbf{F}_{2P}^{\ast}$ to deduce equation \ref{eq:DFT-system}.
%\append{The eigenvalues of $\mathbf{C}$}


In the present Appendix, we show how to efficiently compute matrix $\mathbf{L}$
(equations \ref{eq:left_side_DFT_system_3}, \ref{eq:DFT-system-preliminary} 
and \ref{eq:DFT-system}) by using only the first column of the BCCB matrix
$\mathbf{C}$ (equation \ref{eq:w_Cv}).

We need first premultiply both sides of equation \ref{eq:C-diagonalized}
by $\left(\mathbf{F}_{2Q} \otimes \mathbf{F}_{2P} \right)$ to obtain
\begin{equation}
\left(\mathbf{F}_{2Q} \otimes \mathbf{F}_{2P} \right) \mathbf{C} = 
\boldsymbol{\Lambda}
\left(\mathbf{F}_{2Q} \otimes \mathbf{F}_{2P} \right) \: .
\label{eq:C-diagonalized2}
\end{equation}
From equation \ref{eq:C-diagonalized2}, we can easily show that 
\citep[][ p. 77]{chan-jin2007}:
\begin{equation}
	\left(\mathbf{F}_{2Q} \otimes \mathbf{F}_{2P} \right) \, 
	\mathbf{c}_{0} = \frac{1}{\sqrt{4QP}} \, \boldsymbol{\lambda} \: ,
	\label{eq:DFT_C_column}
\end{equation}
where $\mathbf{c}_{0}$ is a $4QP \times 1$ vector representing the first column of 
$\mathbf{C}$ (equation \ref{eq:w_Cv}) and 
$\boldsymbol{\lambda}$ (equation \ref{eq:left_side_DFT_system_2}) is the $4QP \times 1$ 
vector that contains the diagonal of matrix $\boldsymbol{\Lambda}$ (equation \ref{eq:C-diagonalized}) 
and is obtained by applying the $vec$-operator (equation \ref{eq:vec-operator}) to matrix $\mathbf{L}$.
Now, let us conveniently consider that $\mathbf{c}_{0}$ is obtained by applying the $vec$-operator 
to a $2Q \times 2P$ matrix $\mathbf{G}$.
Using this matrix, the property of the $vec$-operator for separable transformations 
(equation \ref{eq:matrix-S}) and the symmetry of unitary DFT matrices, equation \ref{eq:DFT_C_column} 
can be rewritten as follows
\begin{eqnarray}
	\mathbf{F}_{2Q} \, \mathbf{G} \, \mathbf{F}_{2P} = 
	\frac{1}{\sqrt{4QP}} \, \mathbf{L} \: .
	\label{eq:DFT_G}
\end{eqnarray}
This equation shows that the eigenvalues of the BCCB matrix $\mathbf{C}$ 
(equation \ref{eq:w_Cv}), forming the rows of $\mathbf{L}$,
are obtained by computing the 2D DFT of matrix $\mathbf{G}$,
which contains the elements forming the first column of the BCCB matrix 
$\mathbf{C}$ (equation \ref{eq:w_Cv}).

\newpage

\bibliographystyle{seg}  % style file is seg.bst
\bibliography{references}

\clearpage

% Tables and figures
\tabl{RAM-usage}{This table shows the RAM memory usage (in Megabytes) for storing the whole matrix $\mathbf{A}$ (equation \ref{eq:aij_mag}), the sum of all six first columns of the BCCB matrices embedded from the components of the matrix $\mathbf{H}$ from equation \ref{eq:Hi} (both need 8 bytes per element) and the matrix $\mathbf{L}$ containing the eigenvalues complex numbers (16 bytes per element) resulting from the diagonalization of matrix $\mathbf{C}$ (equation \ref{eq:w_Cv}). Here we must consider that $N$ observation points forms a $N \times N$ matrix.
\label{tab:RAM-usage}}
{
	\begin{center}
		\begin{tabular}[]{|l|c|c|c|}
			\hline
			\textbf{$N$} & \textbf{Matrix $\mathbf{A}$} & \textbf{All six first columns of BCCB matrices} & \textbf{Matrix $\mathbf{L}$}\\
			\hline 
			$100$ & 0.0763 & 0.0183 & 0.00610\\
			\hline
			$400$ & 1.22 & 0.0744 & 0.0248\\
			\hline
			$2,500$ & 48 & 0.458 & 0.1528\\
			\hline
			$10,000$ & 763 & 1.831 & 0.6104\\
			\hline
			$40,000$ & 12,207 & 7.32 & 2.4416 \\
			\hline
			$250,000$ & 476,837 & 45.768 & 15.3 \\
			\hline
			$500,000$ & 1,907,349 & 91.56 & 30.518 \\
			\hline
			$1,000,000$ & 7,629,395 & 183.096 & 61.035 \\
			\hline
		\end{tabular}
	\end{center} 
}


\end{document}
