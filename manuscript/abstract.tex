\begin{abstract}

A fast equivalent layer for magnetic data processing is presented. Taking advantage of the sensitivity matrix structure of the magnetic kernel, when observation and equivalent sources are aligned on a regular spaced grid with constant height, it is possible to calculate the matrix-vector product in a very fast manner. The structure is called block-Toeplitz Toeplitz-block (BTTB) and this type of matrix is well known in literature to be embbeded in a block-Circulant Circulant-block (BCCB), wich in turn can have its eigenvalues calculated using only the first column and a 2D fast Fourier transform. We show that, despite this BTTB matrix is not symmetric, by using only the first equivalent source it is possible to calculate all the first columns of the inverse of distance second derivatives that composes the magnetic kernel and reconstruct the first column of the BCCB matrix, saving computational time and system memory. The conjugate gradient iterative method is used to solve the linear system and estimate the physical properties distributed over the equivalent layer. Synthetic tests show a decrease in the order of $10^4$ in floating-point operations, $25$ times in computation runtime with a mid-size $80 \times 80$ grid, and exponential decrease in memory RAM usage, allowing to perform this operation with millions of observation points on desktop computers. Synthetic tests with irregular grids also show that this method can work with directional disturbances under certain limits. A real magnetic data of Caraj{\'a}s Province, Brazil, with $1,310,000$ observation points in an irregular grid was used to successfully perform a data processing with this method, taking $385.56$ seconds to estimate the physical property and $2.64$ seconds for the upward-continuation.

\end{abstract}