\section{Application to field data}

The field data application was made with the aeromagnetic data of Carajás, Pará, Brazil, provided by the CPRM. 
The total number of data is 7,706,153. For this case we gridded the data to $500 \times 500 = 2,500,000$ observation points, which means that storing the full sensitivity matrix would be necessary 45.47 Terabytes of RAM. However, taking advantage that the second derivatives of equation \ref{eq:r} are symmetric or skew-symmetric matrices, it is possible to reconstruct the whole sensitivity matrix storing only the first column of each component of equation \ref{eq:Hi}, thus, using only 114.44 Megabytes.

To achieve high efficiency in property estimative of the equivalent sources, the method CGLS for inversion was used, combined with a fast matrix-vector product, only possible because of the Block-Toeplitz Toeplitz-Block(BTTB) structure of the sensitivity matrix. This fast matrix-vector product was also used for data processing (upward continuation) in a very efficient way.

As this area is very large different values of the magnetic main field can be considered. 
For this test it was considered an approximated mid location of the area (latitude $-6.5^{\circ}$ and longitude $-50.75^{\circ}$) where the declination is $-19.86^{\circ}$ for the IGRF model in 01/01/2014. The inclination was calculated considering the Geocentric axial dipole model ($tang \, I = 2 \times tan \, \lambda$) and is equal to $12.84^{\circ}$.
As the source magnetization is unknown, inclination and declination equals to the main field is being used.

Figure \ref{fig:carajas_real_data_mag} shows the observed magnetic field data of the area. Using a equivalent layer at $300$ meters above the ground the predicted data and its residual are shown in figure \ref{fig:carajas_gz_predito_mag}. The mean of $-2.7135\times 10^{-07}$ nT and the standart deviation of $3.2726\times 10^{-06}$ nT of the residual shows the good result of physical property estimative. It was used 50 iterations of the CGLS method taking $15.5$ seconds with a Intel core i7 7700HQ@2.8GHz processor.

In figure \ref{fig:up5000_carajas_500x500_mag} the upward continuation transformation was made at $5000$ meters and took $0.49$ seconds.