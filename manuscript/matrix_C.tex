\append{Matrix $\mathbf{C}$}


This appendix illustrates the matrix $\mathbf{C}$ (equation \ref{eq:w_Cv}) 
obtained with the $x$- and $y$-oriented grids illustrated in Figure \ref{fig:methodology}
and also presents some of its relevant properties.

Matrix $\mathbf{C}$ (equation \ref{eq:w_Cv})
is circulant blockwise, formed by $2Q \times 2Q$ blocks, where
each block $\mathbf{C}_{q}$, $q = 0, \dots, Q-1$, is a $2P \times 2P$ circulant matrix. 
Similarly to the BTTB matrix $\mathbf{A}$ (equations \ref{eq:BTTB_A} and 
\ref{eq:A-x-oriented-example}--\ref{eq:Aq-y-oriented}), the index $q$ 
varies from $0$ to $Q - 1$. Additionally, the blocks lying 
above the main diagonal are equal to those located below.

It is well-known that a circulant matrix can be defined by properly downshifting 
its first column \citep[][ p. 206]{vanloan1992}. Hence, the BCCB matrix $\mathbf{C}$ 
(equation \ref{eq:w_Cv}) can be obtained from its 
first column of blocks, which is given by
\begin{equation}
\left[\mathbf{C} \right]_{(0)} = 
\begin{bmatrix}
\mathbf{C}_{0} \\
\vdots \\
\mathbf{C}_{Q-1} \\
\mathbf{0} \\
\mathbf{C}_{Q-1} \\
\vdots \\
\mathbf{C}_{1}
\end{bmatrix}_{4N \times 2P} \: ,
\label{eq:C-first-column-blocks}
\end{equation}
where $\mathbf{0}$ is a $2P \times 2P$ matrix of zeros. Similarly, each block 
$\mathbf{C}_{q}$, $q = 0, \dots, Q-1$, can be obtained by downshifting its first 
column
\begin{equation}
\mathbf{c}^{q}_{0} = 
\begin{bmatrix}
a^{q}_{0} \\
\vdots \\
a^{q}_{P-1} \\
0 \\
a^{q}_{P-1} \\
\vdots \\
a^{q}_{1}
\end{bmatrix}_{2P \times 1} \: ,
\label{eq:Cq-first-column}
\end{equation}
where $a^{q}_{p}$ (equation \ref{eq:aqp_equiv_aij}), $p = 0, \dots, P-1$, are the elements 
forming the block $\mathbf{A}_{q}$ (equations \ref{eq:Aq_block} and 
\ref{eq:A-x-oriented-example}--\ref{eq:Aq-y-oriented}).
The downshift can be thought off as permutation that pushes the components of a column vector 
down one notch with wraparound \citep[][ p. 20]{golub-vanloan2013}.
To illustrate this operation, consider our $y$-oriented grid illustrated in Figure \ref{fig:methodology}b. 
In this case, the resulting 
BCCB matrix $\mathbf{C}$ (equation \ref{eq:w_Cv}) is given by 
\begin{equation}
\mathbf{C} =
\begin{bmatrix}
\mathbf{C_{0}} & \mathbf{C_{1}} & \mathbf{C_{2}} & \mathbf{C_{3}} & \mathbf{0}     & \mathbf{C_{3}} & \mathbf{C_{2}} & \mathbf{C_{1}} \\
\mathbf{C_{1}} & \mathbf{C_{0}} & \mathbf{C_{1}} & \mathbf{C_{2}} & \mathbf{C_{3}} & \mathbf{0}     & \mathbf{C_{3}} & \mathbf{C_{2}} \\
\mathbf{C_{2}} & \mathbf{C_{1}} & \mathbf{C_{0}} & \mathbf{C_{1}} & \mathbf{C_{2}} & \mathbf{C_{3}} & \mathbf{0}     & \mathbf{C_{3}} \\
\mathbf{C_{3}} & \mathbf{C_{2}} & \mathbf{C_{1}} & \mathbf{C_{0}} & \mathbf{C_{1}} & \mathbf{C_{2}} & \mathbf{C_{3}} & \mathbf{0}     \\
\mathbf{0}     & \mathbf{C_{3}} & \mathbf{C_{2}} & \mathbf{C_{1}} & \mathbf{C_{0}} & \mathbf{C_{1}} & \mathbf{C_{2}} & \mathbf{C_{3}} \\
\mathbf{C_{3}} & \mathbf{0}     & \mathbf{C_{3}} & \mathbf{C_{2}} & \mathbf{C_{1}} & \mathbf{C_{0}} & \mathbf{C_{1}} & \mathbf{C_{2}} \\
\mathbf{C_{2}} & \mathbf{C_{3}} & \mathbf{0}     & \mathbf{C_{3}} & \mathbf{C_{2}} & \mathbf{C_{1}} & \mathbf{C_{0}} & \mathbf{C_{1}} \\
\mathbf{C_{1}} & \mathbf{C_{2}} & \mathbf{C_{3}} & \mathbf{0}     & \mathbf{C_{3}} & \mathbf{C_{2}} & \mathbf{C_{1}} & \mathbf{C_{0}}
\end{bmatrix}_{4N \times 4N},
\label{eq:C-y-oriented}
\end{equation}
where each block $\mathbf{C}_{q}$, $q = 0, 1, 2, 3$, is represented as follows 
\begin{equation}
\tensor{C}_{q} =
\begin{bmatrix}
a^{q}_{0} & a^{q}_{1} & a^{q}_{2} & 0         & a^{q}_{2} & a^{q}_{1} \\
a^{q}_{1} & a^{q}_{0} & a^{q}_{1} & a^{q}_{2} & 0         & a^{q}_{2} \\
a^{q}_{2} & a^{q}_{1} & a^{q}_{0} & a^{q}_{1} & a^{q}_{2} & 0         \\
0         & a^{q}_{2} & a^{q}_{1} & a^{q}_{0} & a^{q}_{1} & a^{q}_{2} \\
a^{q}_{2} & 0         & a^{q}_{2} & a^{q}_{0} & a^{q}_{0} & a^{q}_{1} \\
a^{q}_{1} & a^{q}_{2} & 0         & a^{q}_{2} & a^{q}_{1} & a^{q}_{0}
\end{bmatrix}_{2P \times 2P}
\label{eq:Cq-y-oriented}
\end{equation}
in terms of the block elements $a^{q}_{p}$ (equation \ref{eq:aqp_equiv_aij}).
Similar matrices are obtained for our $x$-oriented grid illustrated in Figure \ref{fig:methodology}a.

BCCB matrices are diagonalized by the 2D unitary DFT 
\citep[][ p. 185]{davis1979}. It means that $\mathbf{C}$ (equation \ref{eq:w_Cv}) 
satisfies 
\begin{equation}
\mathbf{C} = 
\left(\mathbf{F}_{2Q} \otimes \mathbf{F}_{2P} \right)^{\ast} 
\boldsymbol{\Lambda}
\left(\mathbf{F}_{2Q} \otimes \mathbf{F}_{2P} \right) \: ,
\label{eq:C-diagonalized}
\end{equation}
where the symbol ``$\otimes$" denotes the Kronecker product \citep{neudecker1969},
$\mathbf{F}_{2Q}$ and $\mathbf{F}_{2P}$ are the $2Q \times 2Q$ and $2P \times 2P$ 
unitary DFT matrices \citep[][ p. 31]{davis1979}, respectively, the superscritpt 
``$\ast$" denotes the complex conjugate and $\boldsymbol{\Lambda}$ is a 
$4QP \times 4QP$ diagonal matrix containing the eigenvalues of $\mathbf{C}$.