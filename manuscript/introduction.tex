\section{Introduction}

Large-scale data processing with tens of thousands of data is a reality in all areas of geophysics including the geophysical potential fields. 
The potential-field data processing includes convolution integrals which can be solved either in the space or Fourier domains.
The earliest techniques of potential-field data processing  were developed in the space domain.
For example, \cite{Peters1949} accomplished, in the space domain, the second and fourth derivatives of magnetic data and the upward- and downward-continuations of magnetic data by deriving coefficients that are
used in a  graphical convolution with the magnetic data.
However, the techniques for  processing potential-field data in space-domain were soon substituted by the Fourier-domain techniques. 
\cite{Dean1958} pointed out that the operations of second order derivatives, analytic continuation, smoothing, the removing of residuals or regionals, and others for processing  potential-field data are
similar to the electric filter circuits in Fourier domain.
\cite{Dean1958} was the first to develop the theory of linear filters in Fourier domain for gravity and magnetic processing and to present filters in Fourier domain \citep[][ see Table I, p 113]{Dean1958} 
for some theoretical geophysical operations (e.g., derivatives and upward and downward continuations).
\cite{GUNN1975} presented a comprehensive analysis of processing potential-field data in Fourier domain.


An approach for processing potential-field data in space domain is the equivalent-layer technique.
The equations deductions of the equivalent layer as a solution of the Laplace's equation in the region above the source was first presented by \cite{kellogg1929} and detailed explanations can also be found in \cite{blakely1996}. 
Although the equivalent-layer technique  has been known since the 1960s in geophysical literature \citep{danes1961structure,bott1967solution,dampney1969}, its use has become feasible only recently 
because of the advances in computational power.
In magnetic data procesing, some authors explored this technique for calculating the first and second vertical derivatives of the fields \citep{emilia1973}, reduction to the pole \citep{silva1986,oliveirajr-etal2013,li2014using}, upward/downward continuations \citep{hansen-miyazaki1984,li-oldenburg2010} and total magnetic induction vector components calculation \citep{sun2019constrained}.

Together with the rise in computational processing power, some works tried new implementations to increase the efficiency of the equivalent layer. 
In \cite{leao-silva1989} the authors used a shifting window over the layer to increase the number of linear systems to be solved, but to reduced the size of each linear system at the same time. 
Another approach for a fast equivalent layer was proposed by \cite{li-oldenburg2010}  who transformed the full sensitivity matrix into a sparse one by using  the compression of the coefficient matrix 
using fast wavelet transforms based on orthonormal, compactly supported wavelets.  
\cite{oliveirajr-etal2013} divided the equivalent layer into a grid of fixed source windows.
Instead of directly calculating the the physical-property distribution of a finite set of equivalent
sources (e.g., dipoles, point of masses) arranged in the entire equivalent layer,
\cite{oliveirajr-etal2013} estimated the coefficients of a bivariate polynomial function describing 
the physical-property distribution within each equivalent-source window.
The estimated polynomial-coefficients are transformed into the physical-property distribution
and thus any standard linear transformation of the data can be performed.
Grounded on excess mass constraint, \cite{siqueira-etal2017} proposed an iterative method 
for processing large gravimetric data using the equivalent layer without requiring the solution 
of a linear system. 
In \cite{siqueira-etal2017}, the initial mass distribution over the equivalent layer is
proportional to observed gravity data and it is updated at each iteration by adding mass corrections that are proportional to the residuals of observed and estimated data.

One of the greatest obstacles to the use of the equivalent-layer technique for processing potential-field data is the solution of the associated linear system.
A wide variety of applications in mathematics and engineering that fall into Toeplitz systems propelled the development of a large variety of  methods for solving them. Direct methods were conceived by \cite{levinson1946} and by \cite{trench1964}. Currently, the iterative method of conjugate gradient is used in most cases, in \cite{chan-jin2007} the authors presented an introduction on the topic for 1D data structures of Toeplitz matrices and also for 2D data structures, which they called block-Toeplitz Toeplitz-block matrices. In both cases, the solving strategy is to embed the Toeplitz/BTTB matrix into a Circulant/Block-Circulant Circulant-Block matrix, calculate its eigenvalues by a 1D or 2D fast Fourier transform of its first column, respectively and carry the matrix-vector product between kernel and parameters at each iteration of the conjugate gradient method in a very fast manner.

In potential field methods, the properties of Toeplitz system have been used for downward continuation \citep{zhang-etal2016} and for 3D gravity-data inversion using a 2D multilayer model \citep{zhang-wong2015}. More recently, \cite{hogue2020tutorial} provided an overview on modeling the gravity and magnetic kernels using the BTTB structures and \cite{renaut2020fast} used BTTB the structures for inversion of both gravity and magnetic data to recover sparse subsurface structures.
\cite{takahashi2020convolutional} combined the fast equivalent source technique presented by \cite{siqueira-etal2017} with the concept of symmetric block-Toeplitz Toeplitz-block (BTTB) matrices to introduce the convolutional equivalent layer for gravimetric data technique. 
\cite{takahashi2020convolutional} showed that the BTTB structure appears when the sensitivity matrix of the linear system, required to solve the gravimetric equivalent layer, is calculated on a regular spaced grid of dataset with constant height and each equivalent source is exactly beneath each observed data point. 
This work showed an decrease in the order of $10^4$ in floating-point operations needed to estimate the equivalent sources; thus, the \cite{takahashi2020convolutional} method was able to efficiently process very large gravity data sets. 
Moreover, \cite{takahashi2020convolutional} method yielded neither significant boundary effects nor noise amplification.

In this work, the convolutional equivalent layer using the block-Toeplitz Toeplitz-block idea, presented in \cite{takahashi2020convolutional}, will be used to solve the linear system required to estimate the physical property that produces a magnetic field on regular grids. 
Here, we achieve very fast solutions using a conjugate gradient algorithm combined with the fast Fourier transform. We present a novel method of exploring the symmetric structures of the second derivatives of the inverse of the distance contained in the magnetic kernel, to keep the memory RAM usage to the minimal by using only one equivalent source to carry the calculations of the forward problem. We also show tests of the magnetic convolutional equivalent layer when irregular grids are used. The convergence of the conjugate gradient maintains in an acceptable level even using irregular grids. 
Our results show the good performance of our method in producing fast and robust solutions for processing large amounts of magnetic data using the equivalent layer technique.


