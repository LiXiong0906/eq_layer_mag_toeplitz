%\documentclass[paper,twocolumn,twoside]{geophysics}
%\documentclass[manuscript,noblind]{geophysics}
\documentclass[manuscript]{geophysics}

% An example of defining macros
\newcommand{\rs}[1]{\mathstrut\mbox{\scriptsize\rm #1}}
\newcommand{\rr}[1]{\mbox{\rm #1}}

% Extra packages
\usepackage{amsmath}
%\usepackage[]{algorithm2e}
\usepackage{algorithm}
\usepackage{bm}
\usepackage[hyphens,spaces]{url}
\usepackage[pdftex,colorlinks=true]{hyperref}
\hypersetup{
	allcolors=black,
}
\usepackage{lipsum}
\usepackage[table]{xcolor}
\usepackage[titletoc,title]{appendix}

\renewcommand{\figdir}{Fig} % figure directory

\begin{document}
	
\renewcommand{\thefootnote}{\fnsymbol{footnote}} 
	
%\ms{GEO-XXXX} % manuscript number
	
\maketitle
	
% Main body
\begin{abstract}
		
We present a fast equivalent layer method for processing large-scale magnetic data. 
We demonstrate that the sensitivity matrix associated with an equivalent layer
of dipoles can be arranged to a Block-Toeplitz Toeplitz-Block (BTTB) structure for the 
case in which observations and dipoles are aligned on a 
horizontal and regularly-spaced grid.
The product of a BTTB matrix and an arbitrary vector represents a discrete 
convolution and can be efficiently computed via 2D Fast Fourier Transform.
In this case, the matrix-vector product uses only the elements forming the first column
of the BTTB matrix, saving computational time and memory. 
Our convolutional equivalent layer method uses this approach to compute 
the matrix-vector products in the iterative conjugate gradient algorithm with the purpose 
of estimating the physical-property distribution over the equivalent layer for 
large data sets.
Synthetic tests with a mid-size $100 \times 50$ grid of total-field anomaly data
show a decrease of $\approx 10^4$ in floating-point operations and $\approx 25\times$ 
in computation runtime of our method compared to the classical approach of solving
the least-squares normal equations via Cholesky decomposition. 
Faster results are obtained for millions of data, showing drastic decreases in RAM usage
and runtime, allowing to perform magnetic data processing of large data sets on regular 
desktop computers. 
Our results also show that, compared to the classical Fourier approach, the magnetic
data processing with our method requires similar computation time, but produces significantly 
smaller border effects without using any padding scheme and also is more robust to 
deal with data on irregularly spaced points or on undulating observation surfaces.
A test with $1,310,000$ irregularly spaced field data over the Caraj{\'a}s Province, Brazil, 
confirms the efficiency of our method by taking $\approx 385.56$ seconds to estimate the physical-property
distribution over the equivalent layer and $\approx 2.64$ seconds to compute the upward 
continuation.
		
\end{abstract}

\section{Introduction}

Large-scale data processing with tens of thousands of data is a reality in all areas of geophysics including the geophysical potential fields. 
The potential-field data processing includes convolution integrals which can be solved either in the space or Fourier domains.
The earliest techniques of potential-field data processing  were developed in the space domain.
For example, \cite{Peters1949} accomplished, in the space domain, the second and fourth derivatives of magnetic data and the upward- and downward-continuations of magnetic data by deriving coefficients that are
used in a  graphical convolution with the magnetic data.
However, the techniques for  processing potential-field data in space-domain were soon substituted by the Fourier-domain techniques. 
\cite{Dean1958} pointed out that the operations of second order derivatives, analytic continuation, smoothing, the removing of residuals or regionals, and others for processing  potential-field data are
similar to the electric filter circuits in Fourier domain.
\cite{Dean1958} was the first to develop the theory of linear filters in Fourier domain for gravity and magnetic processing and to present filters in Fourier domain \citep[][ see Table I, p 113]{Dean1958} 
for some theoretical geophysical operations (e.g., derivatives and upward and downward continuations).
\cite{GUNN1975} presented a comprehensive analysis of processing potential-field data in Fourier domain.


An approach for processing potential-field data in space domain is the equivalent-layer technique.
The equations deductions of the equivalent layer as a solution of the Laplace's equation in the region above the source was first presented by \cite{kellogg1929} and detailed explanations can also be found in \cite{blakely1996}. 
Although the equivalent-layer technique  has been known since the 1960s in geophysical literature \citep{danes1961structure,bott1967solution,dampney1969}, its use has become feasible only recently 
because of the advances in computational power.
In magnetic data procesing, some authors explored this technique for calculating the first and second vertical derivatives of the fields \citep{emilia1973}, reduction to the pole \citep{silva1986,oliveirajr-etal2013,li2014using}, upward/downward continuations \citep{hansen-miyazaki1984,li-oldenburg2010} and total magnetic induction vector components calculation \citep{sun2019constrained}.

Together with the rise in computational processing power, some works tried new implementations to increase the efficiency of the equivalent layer. 
In \cite{leao-silva1989} the authors used a shifting window over the layer to increase the number of linear systems to be solved, but to reduced the size of each linear system at the same time. 
Another approach for a fast equivalent layer was proposed by \cite{li-oldenburg2010}  who transformed the full sensitivity matrix into a sparse one by using  the compression of the coefficient matrix 
using fast wavelet transforms based on orthonormal, compactly supported wavelets.  
\cite{oliveirajr-etal2013} divided the equivalent layer into a grid of fixed source windows.
Instead of directly calculating the the physical-property distribution of a finite set of equivalent
sources (e.g., dipoles, point of masses) arranged in the entire equivalent layer,
\cite{oliveirajr-etal2013} estimated the coefficients of a bivariate polynomial function describing 
the physical-property distribution within each equivalent-source window.
The estimated polynomial-coefficients are transformed into the physical-property distribution
and thus any standard linear transformation of the data can be performed.
Grounded on excess mass constraint, \cite{siqueira-etal2017} proposed an iterative method 
for processing large gravimetric data using the equivalent layer without requiring the solution 
of a linear system. 
In \cite{siqueira-etal2017}, the initial mass distribution over the equivalent layer is
proportional to observed gravity data and it is updated at each iteration by adding mass corrections that are proportional to the residuals of observed and estimated data.

One of the greatest obstacles to the use of the equivalent-layer technique for processing potential-field data is the solution of the associated linear system.
A wide variety of applications in mathematics and engineering that fall into Toeplitz systems propelled the development of a large variety of  methods for solving them. Direct methods were conceived by \cite{levinson1946} and by \cite{trench1964}. Currently, the iterative method of conjugate gradient is used in most cases, in \cite{chan-jin2007} the authors presented an introduction on the topic for 1D data structures of Toeplitz matrices and also for 2D data structures, which they called block-Toeplitz Toeplitz-block matrices. In both cases, the solving strategy is to embed the Toeplitz/BTTB matrix into a Circulant/Block-Circulant Circulant-Block matrix, calculate its eigenvalues by a 1D or 2D fast Fourier transform of its first column, respectively and carry the matrix-vector product between kernel and parameters at each iteration of the conjugate gradient method in a very fast manner.

In potential field methods, the properties of Toeplitz system have been used for downward continuation \citep{zhang-etal2016} and for 3D gravity-data inversion using a 2D multilayer model \citep{zhang-wong2015}. More recently, \cite{hogue2020tutorial} provided an overview on modeling the gravity and magnetic kernels using the BTTB structures and \cite{renaut2020fast} used BTTB the structures for inversion of both gravity and magnetic data to recover sparse subsurface structures.
\cite{takahashi2020convolutional} combined the fast equivalent source technique presented by \cite{siqueira-etal2017} with the concept of symmetric block-Toeplitz Toeplitz-block (BTTB) matrices to introduce the convolutional equivalent layer for gravimetric data technique. 
\cite{takahashi2020convolutional} showed that the BTTB structure appears when the sensitivity matrix of the linear system, required to solve the gravimetric equivalent layer, is calculated on a regular spaced grid of dataset with constant height and each equivalent source is exactly beneath each observed data point. 
This work showed an decrease in the order of $10^4$ in floating-point operations needed to estimate the equivalent sources; thus, the \cite{takahashi2020convolutional} method was able to efficiently process very large gravity data sets. 
Moreover, \cite{takahashi2020convolutional} method yielded neither significant boundary effects nor noise amplification.

In this work, the convolutional equivalent layer using the block-Toeplitz Toeplitz-block idea, presented in \cite{takahashi2020convolutional}, will be used to solve the linear system required to estimate the physical property that produces a magnetic field on regular grids. 
Here, we achieve very fast solutions using a conjugate gradient algorithm combined with the fast Fourier transform. We present a novel method of exploring the symmetric structures of the second derivatives of the inverse of the distance contained in the magnetic kernel, to keep the memory RAM usage to the minimal by using only one equivalent source to carry the calculations of the forward problem. We also show tests of the magnetic convolutional equivalent layer when irregular grids are used. The convergence of the conjugate gradient maintains in an acceptable level even using irregular grids. 
Our results show the good performance of our method in producing fast and robust solutions for processing large amounts of magnetic data using the equivalent layer technique.


\section{Methodology}

%======================================================================================
\subsection{Classical equivalent layer for magnetic data}
%======================================================================================

Let $\mathbf{d}^{o}$ be the $N \times 1$ observed data vector $(x_{i}, y_{i}, z_{i})$, $i =  1, \dots, N $,
whose $i$th element is the total-field anomaly $d^{o}_{i}$ produced by arbitrarily magnetized sources
aranged in a right-handed Cartesian coordinate system with $x$-, $y$- and $z$-axis 
pointing to north, east and down, respectively.
We consider that the total-field anomaly data $d^{o}_{i}$ represent the discrete
values of a harmonic function. Besides, we consider that the main geomagnetic field 
direction is constant at the limited study area and can be defined by the unit vector
\begin{equation}
	\hat{\mathbf{F}} = \begin{bmatrix}
		F_x \\
		F_y \\
		F_z
	\end{bmatrix} =
	\begin{bmatrix}
		\cos(I_{0}) \, \cos(D_{0}) \\
		\cos(I_{0}) \, \sin(D_{0}) \\
		\sin(I_{0})
	\end{bmatrix} \: ,
	\label{eq:unit_vector_F}
\end{equation}
with constant inclination $I_{0}$ and declination $D_{0}$.
In this case, $d^{o}_{i}$ can be approximated by the predicted total-field anomaly
\begin{equation}
	\Delta T_{i} = \sum_{j=1}^{M} \, p_{j} a_{ij} \: ,
	\label{eq:integral-sum_mag}
\end{equation}
which describes the magnetic induction exerted at the observation point,
by a discrete layer of $M$ dipoles (equivalent sources) defined on the horizontal plane $z = z_{c}$, 
where $p_{j}$ is the magnetic moment intensity (in A~m~$^{2}$)~of the $j$th dipole, 
that has unit volume and is located at the point $(x_{j}, y_{j}, z_{c})$. In equation
\ref{eq:integral-sum_mag}, $a_{ij}$ is the harmonic function
\begin{equation}
	a_{ij}
	= c_{m} \, \frac{\mu_{0}}{4\pi} \, \hat{\mathbf{F}}^{\top} \mathbf{H}_{ij} \: \hat{\mathbf{u}} \: ,
	\label{eq:aij_mag}
\end{equation}
the unit vector
\begin{equation}
	\hat{\mathbf{u}} = \begin{bmatrix}
		u_x \\
		u_y \\
		u_z
	\end{bmatrix} =
	\begin{bmatrix}
		\cos(I) \, \cos(D) \\
		\cos(I) \, \sin(D) \\
		\sin(I)
	\end{bmatrix} \: ,
	\label{eq:u_hat}
\end{equation}
defines the magnetization direction of all dipoles, with constant inclination $I$ and declination $D$,
$\mu_{0} = 4\pi \, 10^{-7}$ H/m is the magnetic constant, $c_{m} = 10^{9}$ is a factor that transforms
the magnetic induction from Tesla (T) to nanotesla (nT) and $\mathbf{H}_{ij}$ is a $3 \times 3$ matrix 
\begin{equation}
	\mathbf{H}_{ij} = \begin{bmatrix}
		h^{xx}_{ij} & h^{xy}_{ij} & h^{xz}_{ij} \\
		h^{xy}_{ij} & h^{yy}_{ij} & h^{yz}_{ij} \\
		h^{xz}_{ij} & h^{yz}_{ij} & h^{zz}_{ij}
	\end{bmatrix} \: ,
	\label{eq:Hij}
\end{equation}
where 
\begin{equation}
	h^{\alpha\beta}_{ij} = 
	\begin{cases}
		\frac{3 \left( \alpha_{i} - \alpha_{j} \right)^{2}}{r_{ij}^{5}} - \frac{1}{r_{ij}^{3}} \: , \quad \alpha = \beta \\
		\frac{3 \left( \alpha_{i} - \alpha_{j} \right) \left( \beta_{i} - \beta_{j} \right)}{r_{ij}^{5}} \: , \quad \alpha \ne \beta
	\end{cases} \: , \quad \alpha, \beta = x, y, z \: ,
	\label{eq:hij_alpha_beta}
\end{equation}
are the second derivatives of the inverse distance function
\begin{equation}
	\frac{1}{r_{ij}} = 
	\frac{1}{\sqrt{\left(x_{i} - x_{j} \right)^{2} + 
			\left(y_{i} - y_{j} \right)^{2} + \left(z_{i} - z_{c} \right)^{2}}}
	\label{eq:1_rij}
\end{equation}
with respect to the coordinates of the observation point $(x_{i}, y_{i}, z_{i})$.

Equation \ref{eq:integral-sum_mag} can be rewritten in matrix notation as follows:
\begin{equation}
	\mathbf{d}(\mathbf{p}) = \mathbf{A} \mathbf{p} \: ,
	\label{eq:predicted-data-vector_mag}
\end{equation}
where $\mathbf{d}(\mathbf{p})$ is the $N \times 1$ predicted data vector with $i$th element defined
as the predicted total-field anomaly $\Delta T_{i}$ (equation \ref{eq:integral-sum_mag}),
$\mathbf{p}$ is the $M \times 1$ parameter vector whose $j$th element is the magnetic moment intensity
$p_{j}$ of the $j$th dipole and $\mathbf{A}$ is the $N \times M$ sensitivity matrix with element 
$ij$ defined by the harmonic function $a_{ij}$ (equation \ref{eq:aij_mag}).
In the classical equivalent-layer technique, the common approach for 
estimating the parameter vector $\mathbf{p}$ from the observed 
total-field anomaly data $\mathbf{d}^{o}$ is solving the least-squares normal equations
\begin{equation}
	\mathbf{A}^{\top}\mathbf{A} \: \mathbf{p} = 
	\mathbf{A}^{\top} \mathbf{d}^{o} \: .
	\label{eq:normal-equations}
\end{equation}
Equation \ref{eq:normal-equations} is usually solved by first computing the Cholesky 
factor $\mathbf{G}$ of the positive-definite matrix $\mathbf{A}^{\top}\mathbf{A}$ and then using it to solve the linear 
systems \citep[][ p. 262]{golub-vanloan2013}:
\begin{equation}
	\begin{split}
		\mathbf{G} \mathbf{w} &= \mathbf{A}^{\top}\mathbf{d}^{o} \\
		\mathbf{G}^{\top} \tilde{\mathbf{p}} &= \mathbf{w}
	\end{split} \: ,
	\label{eq:classical-method}
\end{equation}
where $\mathbf{w}$ is a temporarily variable.
This approach to estimating the parameter vector will be 
referenced throughout this work as the \textit{classical method}.
The computational cost associated with the classical method can be very high
when dealing with large datasets. In the following subsections, we will show how to 
explore the structure of the sensitivity matrix $\mathbf{A}$ and 
efficiently solve the least-squares normal equations (equation \ref{eq:normal-equations}).


%=====================================================================================================
\subsection{Matrix $\mathbf{A}$ in terms of matrix components $\mathbf{A_{\boldsymbol{\alpha\beta}}}$}
%=====================================================================================================

To access the structure of the sensitivity matrix $\mathbf{A}$ 
(equation \ref{eq:predicted-data-vector_mag}), let us first rewrite its elements 
$a_{ij}$ (equation \ref{eq:aij_mag}) in the following way:
\begin{equation}
	\begin{split}
		a_{ij} = a^{xx}_{ij} + a^{xy}_{ij} + a^{xz}_{ij} + a^{yy}_{ij} + a^{yz}_{ij} + a^{zz}_{ij} \: ,
	\end{split}
	\label{eq:aij_mag_expand}
\end{equation}
where
\begin{equation}
	a^{\alpha\beta}_{ij} = 
	\begin{cases}
		c_{m} \, \frac{\mu_{0}}{4\pi} 
		\left( F_{\alpha} u_{\beta} \right) h^{\alpha\beta}_{ij} \: &, \quad \alpha = \beta \\
		c_{m} \, \frac{\mu_{0}}{4\pi} 
		\left( F_{\alpha} u_{\beta} + F_{\beta} u_{\alpha} \right) h^{\alpha\beta}_{ij} \: &, \quad \alpha \ne \beta \\
	\end{cases}
	\: , \quad \alpha, \beta = x, y, z \: ,
	\label{eq:aij_alpha_beta}
\end{equation}
are defined by the elements of $\hat{\mathbf{F}}$ 
(equation \ref{eq:unit_vector_F}), $\hat{\mathbf{u}}$ (equation \ref{eq:u_hat}) and 
$\mathbf{H}_{ij}$ (equations \ref{eq:Hij} and \ref{eq:hij_alpha_beta}).
Then, we can rewrite the sensitivity matrix $\mathbf{A}$ 
(equation \ref{eq:predicted-data-vector_mag}) according to:
\begin{equation}
	\mathbf{A} = \mathbf{A_{xx}} + \mathbf{A_{xy}} + \mathbf{A_{xz}} + 
	\mathbf{A_{yy}} + \mathbf{A_{yz}} + \mathbf{A_{zz}} \: ,
	\label{eq:A_expand}
\end{equation}
where $\mathbf{A_{\boldsymbol{\alpha\beta}}}$ are $N \times M$ matrices with elements 
$ij$ defined by $a^{\alpha\beta}_{ij}$ (equation \ref{eq:aij_alpha_beta}).

Now we can define the structure of $\mathbf{A}$ in terms of its components 
$\mathbf{A_{\boldsymbol{\alpha\beta}}}$ (equation \ref{eq:A_expand}). To do this, 
we consider the particular case in which the observed total-field anomaly is located 
on an $N_{x} \times N_{y}$ 
regular grid of points spaced by $\Delta_{x}$ and $\Delta_{y}$ along the $x$- and $y$-directions,
respectively, on a constant vertical coordinate $z_{0}$. We also consider that the equivalent layer
is aranged such that one dipole is located right below each observation point, at a constant coordinate $z_{c}$.
In this case, the number of equivalent sources $M$ is equal to the number of data $N$ and, 
consequently, matrices $\mathbf{A}$ and $\mathbf{A_{\boldsymbol{\alpha\beta}}}$ become 
square ($N \times N$). 
Besides, the horizontal coordinates $x_{i}$ and $y_{i}$ of the observation points 
can be defined by
\begin{equation}
	x_{i} = x_{1} + \left[ k(i) - 1 \right] \, \Delta_{x}
	\label{eq:xi}
\end{equation}
and
\begin{equation}
	y_{i} = y_{1} + \left[ l(i) - 1 \right] \, \Delta_{y} \: ,
	\label{eq:yi}
\end{equation}
where $x_{1}$ and $y_{1}$ are the start grid coordinates for $x_{i}$ and $y_{i}$, respectively,
and $k(i)$ and $l(i)$ are integer functions defined according to the orientation
of the data grid (Figure \ref{fig:regular-grids}). 
For $x$-\textit{oriented grids}, the integer functions are given by
\begin{equation}
	k(i)  = i - \Bigg\lceil \frac{i}{N_{x}} \Bigg\rceil N_{x} + N_{x}
	\label{eq:k-x-oriented}
\end{equation}
and
\begin{equation}
	l(i) = \Bigg\lceil \frac{i}{N_{x}} \Bigg\rceil \: .
	\label{eq:l-x-oriented}
\end{equation}
For $y$-\textit{oriented grids}, the integer functions are given by
\begin{equation}
	k(i) = \Bigg\lceil \frac{i}{N_{y}} \Bigg\rceil
	\label{eq:k-y-oriented}
\end{equation}
and
\begin{equation}
	l(i) = i - \Bigg\lceil \frac{i}{N_{y}} \Bigg\rceil N_{y} + N_{y} \: .
	\label{eq:l-y-oriented}
\end{equation}
In equations \ref{eq:k-x-oriented}--\ref{eq:l-y-oriented}, $\lceil \cdot \rceil$ denotes the ceiling 
function \citep[e.g.,][ p. 67-68]{graham-etal1994}.
Equations \ref{eq:xi}--\ref{eq:l-y-oriented} can also be used to define the coordinates 
$x_{j}$ and $y_{j}$ of the equivalent sources, but with index $j$ instead of $i$.

By using equations \ref{eq:xi}--\ref{eq:l-y-oriented} to define the coordinates $x_{i}$ and 
$y_{i}$ of the observation points and $x_{j}$ and $y_{j}$ of the equivalent sources, we can
rewrite the elements $h^{\alpha\beta}_{ij}$ (equation \ref{eq:hij_alpha_beta}) of matrix 
$\mathbf{H}_{ij}$ (equation \ref{eq:Hij}) as follows:
\begin{equation}
	h^{xx}_{ij} = 
	\frac{3 \left( \Delta k_{ij} \, \Delta_{x} \right)^{2}}{r_{ij}^{5}} - \frac{1}{r_{ij}^{3}} \: ,
	\label{eq:hxx_regular}
\end{equation}
\begin{equation}
	h^{yy}_{ij} = 
	\frac{3 \left( \Delta l_{ij} \, \Delta_{y} \right)^{2}}{r_{ij}^{5}} - \frac{1}{r_{ij}^{3}} \: ,
	\label{eq:hyy_regular}
\end{equation}
\begin{equation}
	h^{zz}_{ij} = 
	\frac{3 \Delta_{z}^{2}}{r_{ij}^{5}} - \frac{1}{r_{ij}^{3}} \: ,
	\label{eq:hzz_regular}
\end{equation}
\begin{equation}
	h^{xy}_{ij} = 
	\frac{3 \left( \Delta k_{ij} \, \Delta_{x} \right)\left( \Delta l_{ij} \, \Delta_{y} \right)}{r_{ij}^{5}} \: ,
	\label{eq:hxy_regular}
\end{equation}
\begin{equation}
	h^{xz}_{ij} = 
	\frac{3 \left( \Delta k_{ij} \, \Delta_{x} \right) \Delta_{z}}{r_{ij}^{5}}
	\label{eq:hxz_regular}
\end{equation}
and
\begin{equation}
	h^{yz}_{ij} = 
	\frac{3 \left( \Delta l_{ij} \, \Delta_{y} \right) \Delta_{z}}{r_{ij}^{5}} \: ,
	\label{eq:hyz_regular}
\end{equation}
where $\Delta_{z} = z_{c} - z_{0}$, 
\begin{equation}
	\Delta k_{ij} = \frac{x_{i} - x_{j}}{\Delta_{x}} = k(i) - k(j) \: ,
	\label{eq:Delta_kij}
\end{equation}
\begin{equation}
	\Delta l_{ij} = \frac{y_{i} - y_{j}}{\Delta_{y}} = l(i) - l(j)
	\label{eq:Delta_lij}
\end{equation}
and 
\begin{equation}
	\frac{1}{r_{ij}} = 
	\frac{1}{\sqrt{\left( \Delta k_{ij} \, \Delta_{x} \right)^{2} + \left( \Delta l_{ij} \, \Delta_{y} \right)^{2} + \Delta_{z}^{2}}} \: .
	\label{eq:1_rij_regular}
\end{equation}
Note that the integer functions $k(i)$, $k(j)$, $l(i)$ and $l(j)$ (equations 
\ref{eq:k-x-oriented}--\ref{eq:l-y-oriented}) defining $\Delta k_{ij}$ (equation
\ref{eq:Delta_kij}), $\Delta l_{ij}$ (equation \ref{eq:Delta_lij}) and 
$\tfrac{1}{r_{ij}}$ (equation \ref{eq:1_rij_regular}) assume different 
forms depending on the defined grid orientation.
Despite of that, it can be shown that
\begin{equation}
	\Delta k_{ij} = - \Delta k_{ji} \: ,
	\label{eq:Delta_kij_symmetry}
\end{equation}
\begin{equation}
	\Delta l_{ij} = - \Delta l_{ji}
	\label{eq:Delta_lij_symmetry}
\end{equation}
and 
\begin{equation}
	\frac{1}{r_{ij}} = \frac{1}{r_{ji}}
	\label{eq:1_rij_symmetry}
\end{equation}
for any grid orientation.

%=================================================================================
\subsection{General structure of matrices $\mathbf{A_{\boldsymbol{\alpha\beta}}}$}
%=================================================================================

By using equations \ref{eq:hxx_regular}--\ref{eq:1_rij_regular} to compute 
$a^{\alpha\beta}_{ij}$ (equation \ref{eq:aij_alpha_beta}), we can show that 
matrices $\mathbf{A_{\boldsymbol{\alpha\beta}}}$ (equation \ref{eq:A_expand}) assume
well-defined structures that can be conveniently
represented with \textit{block indices} $q$ and $p$ \citep{takahashi2020convolutional}.
These indices are defined by the integer functions $\Delta k_{ij}$ and $\Delta l_{ij}$ 
(equations \ref{eq:Delta_kij} and \ref{eq:Delta_lij}) in terms of the indices $i$ 
of the observation points $(x_{i}, y_{i}, z_{0})$ and $j$ of the equivalent sources
$(x_{j}, y_{j}, z_{c})$.
For $x$-\textit{oriented grids} (Figure \ref{fig:regular-grids}), $Q = N_{y}$, $P = N_{x}$ 
and the block indices $q$ and $p$ are given by:
\begin{equation}
	q \equiv q(i, j) = \Delta l_{ij}
	\label{eq:q-x-oriented}
\end{equation}
and
\begin{equation}
	p \equiv p(i, j) = \Delta k_{ij} \: ,
	\label{eq:p-x-oriented}
\end{equation}
where $\Delta k_{ij}$ and $\Delta l_{ij}$ (equations \ref{eq:Delta_kij} and \ref{eq:Delta_lij}) 
are defined by integer functions $k(i)$, $k(j)$, $l(i)$ and $l(j)$ given by equations 
\ref{eq:k-x-oriented} and \ref{eq:l-x-oriented}.
For $y$-\textit{oriented grids} (Figure \ref{fig:regular-grids}), $Q = N_{x}$, $P = N_{y}$ and 
the block indices $q$ and $p$ are given by:
\begin{equation}
	q \equiv q(i, j) = \Delta k_{ij}
	\label{eq:q-y-oriented}
\end{equation}
and
\begin{equation}
	p \equiv p(i, j) = \Delta l_{ij} \: ,
	\label{eq:p-y-oriented}
\end{equation}
where $\Delta k_{ij}$ and $\Delta l_{ij}$ (equations \ref{eq:Delta_kij} and \ref{eq:Delta_lij}) 
are defined by integer functions $k(i)$, $k(j)$, $l(i)$ and $l(j)$ given by equations 
\ref{eq:k-y-oriented} and \ref{eq:l-y-oriented}.
Equations \ref{eq:q-x-oriented}--\ref{eq:p-y-oriented} show that $q$ varies from $-Q+1$
to $Q-1$ and $p$ from $-P+1$ to $P-1$, regardless of the grid orientation. Notice these equations differ 
from those presented by \citet{takahashi2020convolutional} by not being represented as the absolute values.

% in review ==========>

Let us consider the small regular grid of $N_{x} = 3$ and $N_{y} = 2$ points shown by
Figure \ref{fig:regular-grids}. This grid may represent observation points 
$(x_{i}, y_{i}, z_{0})$ with constant vertical coordinate $z_{0}$ or equivalent sources
$(x_{j}, y_{j}, z_{c})$ with constant vertical coordinate $z_{c} > z_{0}$. In both cases,
the horizontal coordinates are defined by equations \ref{eq:xi} and \ref{eq:yi}.
Given an index $i$, associated with an observation point, and an index $j$, associated with
an equivalent source, we can compute $\Delta k_{ij}$ (equation \ref{eq:Delta_kij}), 
$\Delta l_{ij}$ (equation \ref{eq:Delta_lij}) and $\tfrac{1}{r_{ij}}$ 
(equation \ref{eq:1_rij_regular}). The matrices $\Delta\mathbf{K}$ and $\Delta\mathbf{L}$ 
having elements $ij$ 
defined by $\Delta k_{ij}$ and $\Delta l_{ij}$, respectively, assume different forms, depending on
the grid orientation. For $x$-oriented grids (Figure \ref{fig:regular-grids}), they are given by:
\begin{equation}
	\Delta\mathbf{K} = \begin{bmatrix}
		0 &  -1 &  -2 &   0 &  -1 &  -2 \\
		1 &   0 &  -1 &   1 &   0 &  -1 \\
		2 &   1 &   0 &   2 &   1 &   0 \\
		0 &  -1 &  -2 &   0 &  -1 &  -2 \\
		1 &   0 &  -1 &   1 &   0 &  -1 \\
		2 &   1 &   0 &   2 &   1 &   0 \\
	\end{bmatrix}
	\label{eq:DK-matrix-x-oriented}
\end{equation}
and
\begin{equation}
	\Delta\mathbf{L} = \begin{bmatrix}
		0 &   0 &   0 &  -1 &  -1 &  -1 \\
		0 &   0 &   0 &  -1 &  -1 &  -1 \\
		0 &   0 &   0 &  -1 &  -1 &  -1 \\
		1 &   1 &   1 &   0 &   0 &   0 \\
		1 &   1 &   1 &   0 &   0 &   0 \\
		1 &   1 &   1 &   0 &   0 &   0 \\
	\end{bmatrix} \: .
	\label{eq:DL-matrix-x-oriented}
\end{equation}
For $y$-oriented grids (Figure \ref{fig:regular-grids}), they are given by:
\begin{equation}
	\Delta\mathbf{K} = \begin{bmatrix}
		0 &   0 &  -1 &  -1 &  -2 &  -2 \\
		0 &   0 &  -1 &  -1 &  -2 &  -2 \\
		1 &   1 &   0 &   0 &  -1 &  -1 \\
		1 &   1 &   0 &   0 &  -1 &  -1 \\
		2 &   2 &   1 &   1 &   0 &   0 \\
		2 &   2 &   1 &   1 &   0 &   0 \\
	\end{bmatrix}
	\label{eq:DK-matrix-y-oriented}
\end{equation}
and
\begin{equation}
	\Delta\mathbf{L} = \begin{bmatrix}
		0 &  -1 &   0 &  -1 &   0 &  -1 \\
		1 &   0 &   1 &   0 &   1 &   0 \\
		0 &  -1 &   0 &  -1 &   0 &  -1 \\
		1 &   0 &   1 &   0 &   1 &   0 \\
		0 &  -1 &   0 &  -1 &   0 &  -1 \\
		1 &   0 &   1 &   0 &   1 &   0 \\
	\end{bmatrix} \: .
	\label{eq:DL-matrix-y-oriented}
\end{equation}
For example, consider the matrix coordinates $x_{5}, y_{5}$ for the observation point and $x_{3}, y_{3}$ for the source in a $x$-oriented grid, these are defined by the matrix index $i = 5$ and $j = 3$.
Using equations \ref{eq:k-x-oriented} and \ref{eq:l-x-oriented} the grid indices $k(i) = 2$, $l(i) = 2$ and $k(j) = 3$, $l(j) = 1$ can be calculated. Equations \ref{eq:Delta_kij} and \ref{eq:Delta_lij} define $\Delta k_{ij} = -1$ and $\Delta l_{ij} = 1$, which represents the the value in the fifth row and third column of matrices $\Delta\mathbf{K}$ and $\Delta\mathbf{L}$ (equations \ref{eq:DK-matrix-x-oriented} and \ref{eq:DL-matrix-x-oriented}), respectively. If we use the matrix coordinates $x_{4}$ ($y_{4}$) for the observation point and $x_{2}$ ($y_{2}$) for the source, lead us to the same values for $\Delta k_{ij}$ and $\Delta l_{ij}$ (fourth row and second column of equations \ref{eq:DK-matrix-x-oriented} and \ref{eq:DL-matrix-x-oriented}).
These examples (equations \ref{eq:DK-matrix-x-oriented}--\ref{eq:DL-matrix-y-oriented})
show that different combinations of indices $i$ and $j$ result in integer functions 
$\Delta k_{ij}$ and $\Delta l_{ij}$ (equations \ref{eq:Delta_kij} and \ref{eq:Delta_lij}) 
having the same numerical value. In these cases, not only the numerical values of
the corresponding elements $a^{\alpha\beta}_{ij}$ (equation \ref{eq:aij_alpha_beta}),
but also their associated block indices $q$ and $p$ (equations 
\ref{eq:q-x-oriented}--\ref{eq:p-y-oriented}) are the same.
The contrary is also true: if the elements $a^{\alpha\beta}_{ij}$ have different 
associated block indices $q$ and $p$, they also have different numerical values.
Because of that, using the alternative notation $a^{\alpha\beta}_{qp}$ to define the elements 
$a^{\alpha\beta}_{ij}$ in terms of its associated block indices $q$ and $p$ is a good
approach to investigating the structure of a given matrix component 
$\mathbf{A_{\boldsymbol{\alpha\beta}}}$ (equation \ref{eq:A_expand}).
This approach allows identifying elements $a^{\alpha\beta}_{ij}$ having the same numerical
value only by inspecting their associated block indices.

Note that for $x$-oriented grids, matrices $\Delta\mathbf{K}$ (equation \ref{eq:DK-matrix-x-oriented})
and $\Delta\mathbf{L}$ (equation \ref{eq:DL-matrix-x-oriented}) define the block indices
$p$ (equation \ref{eq:p-x-oriented}) and $q$ (equation \ref{eq:q-x-oriented}), respectively.
In this case, they are composed of $Q \times Q$ blocks with $P \times P$ elements each, where 
$Q = N_{y}$ and $P = N_{x}$. 
For $y$-oriented grids, matrices $\Delta\mathbf{K}$ (equation \ref{eq:DK-matrix-y-oriented})
and $\Delta\mathbf{L}$ (equation \ref{eq:DL-matrix-y-oriented}) define the block indices
$q$ (equation \ref{eq:q-y-oriented}) and $p$ (equation \ref{eq:p-y-oriented}), respectively.
In this case, they are also composed of $Q \times Q$ blocks with $P \times P$ elements each, 
but now $Q = N_{x}$ and $P = N_{y}$.
The examples shown by equations \ref{eq:DK-matrix-x-oriented}--\ref{eq:DL-matrix-y-oriented}
also illustrate that, regardless of grid orientation, (i) the block index $q$ is constant 
inside each block; (ii) blocks disposed along the same block diagonal are equal to each other; 
(iii) the block index $p$ is constant on each diagonal of a given block; 
(iv) elements of a given block located on the same diagonal are also equal do each other.
The results obtained with the small grid shown in Figure \ref{fig:regular-grids}
can be easily generalized for larger grids.
Based on the well-defined structure of block indices, we can define 
matrices $\mathbf{A_{\boldsymbol{\alpha\beta}}}$ in a general form
\begin{equation}
	\mathbf{A}_{\boldsymbol{\alpha\beta}} = \begin{bmatrix}
		\mathbf{A}_{\boldsymbol{\alpha\beta}}^{0}   & \mathbf{A}_{\boldsymbol{\alpha\beta}}^{-1} & \cdots          & \mathbf{A}_{\boldsymbol{\alpha\beta}}^{-Q+1} \\
		\mathbf{A}_{\boldsymbol{\alpha\beta}}^{1}   & \ddots          & \ddots          & \vdots           \\ 
		\vdots           & \ddots          & \ddots          & \mathbf{A}_{\boldsymbol{\alpha\beta}}^{-1}   \\
		\mathbf{A}_{\boldsymbol{\alpha\beta}}^{Q-1} & \cdots          & \mathbf{A}_{\boldsymbol{\alpha\beta}}^{1}  & \mathbf{A}_{\boldsymbol{\alpha\beta}}^{0}
	\end{bmatrix}_{N \times N} \: ,
	\label{eq:BTTB_A_alpha_beta}
\end{equation}
with blocks $\mathbf{A}_{\boldsymbol{\alpha\beta}}^{q}$, $q = -Q+1, \dots, Q-1$, given by
\begin{equation}
	\mathbf{A}_{\boldsymbol{\alpha\beta}}^{q} = \begin{bmatrix}
		a^{\alpha\beta}_{q0}   & a^{\alpha\beta}_{q(-1)} & \cdots  & a^{\alpha\beta}_{q(-P+1)} \\
		a^{\alpha\beta}_{q1}   & \ddots     & \ddots  & \vdots       \\ 
		\vdots      & \ddots     & \ddots  & a^{\alpha\beta}_{q(-1)}   \\
		a^{\alpha\beta}_{q(P-1)} & \cdots     & a^{\alpha\beta}_{q1}  & a^{\alpha\beta}_{q0}
	\end{bmatrix}_{P \times P} \: ,
	\label{eq:Aq_block}
\end{equation}
formed by elements $a^{\alpha\beta}_{qp}$, $p = -P+1, \dots, P-1$.
This well-defined structure (equations \ref{eq:BTTB_A_alpha_beta} and \ref{eq:Aq_block}) 
of matrix components $\mathbf{A_{\boldsymbol{\alpha\beta}}}$ 
(equation \ref{eq:A_expand}) is called Block-Toeplitz Toeplitz-Block (BTTB) 
\citep[e.g., ][ p. 67]{chan-jin2007}.

%=====================================================================================================
\subsection{Detailed structure of matrices $\mathbf{A_{xx}}$, $\mathbf{A_{yy}}$ and $\mathbf{A_{zz}}$}
%=====================================================================================================

Equations \ref{eq:BTTB_A_alpha_beta} and \ref{eq:Aq_block} define the general BTTB
structure of all matrix components $\mathbf{A_{\boldsymbol{\alpha\beta}}}$, but 
there are some differences between them.
Let us consider the matrix component $\mathbf{A}_{\boldsymbol{xx}}$, with elements
$a^{xx}_{ij}$ (equation \ref{eq:aij_alpha_beta}) defined by the second derivative calculated in
$h^{xx}_{ij}$ (equation \ref{eq:hxx_regular}). It can be easily verified from equations
\ref{eq:Delta_kij_symmetry} and \ref{eq:1_rij_symmetry} that $h^{xx}_{ij} = h^{xx}_{ji}$.
As a consequence, $a^{xx}_{ij} = a^{xx}_{ji}$, which means that 
\begin{equation}
	\mathbf{A}_{\boldsymbol{xx}} = \left( \mathbf{A}_{\boldsymbol{xx}} \right)^{\top}
	\label{eq:Axx_symmetry}
\end{equation}
for any grid orientation.
Now, let us investigate the elements $a^{xx}_{qp}$ forming the blocks $\mathbf{A}_{\boldsymbol{xx}}^{q}$.
For $x$-oriented grids (Figure \ref{fig:regular-grids}), the block indices $q$ and $p$ are defined 
by equations \ref{eq:q-x-oriented} and 
\ref{eq:p-x-oriented} and $a^{xx}_{qp}$ can be rewritten as follows:
\begin{equation}
	a^{xx}_{qp} = c_{m} \, \frac{\mu_{0}}{4\pi} 
	\left( F_{x} u_{x} \right) \frac{3 \left( p \, \Delta_{x} \right)^{2}}{r_{qp}^{5}} - 
	\frac{1}{r_{qp}^{3}} \: ,
	\label{eq:aqp_xx_x_oriented}
\end{equation}
where
\begin{equation}
	\frac{1}{r_{qp}} = 
	\frac{1}{\sqrt{\left( p \, \Delta_{x} \right)^{2} + \left( q \, \Delta_{y} \right)^{2} + \Delta_{z}^{2}}} \: .
	\label{eq:1_rqp_x_oriented}
\end{equation}
For $y$-oriented grids (Figure \ref{fig:regular-grids}), the block indices $q$ and $p$ are 
defined by equations \ref{eq:q-y-oriented} and 
\ref{eq:p-y-oriented} and $a^{xx}_{qp}$ can be rewritten as follows:
\begin{equation}
	a^{xx}_{qp} = c_{m} \, \frac{\mu_{0}}{4\pi} 
	\left( F_{x} u_{x} \right) \frac{3 \left( q \, \Delta_{x} \right)^{2}}{r_{qp}^{5}} - 
	\frac{1}{r_{qp}^{3}} \: ,
	\label{eq:aqp_xx_y_oriented}
\end{equation}
where
\begin{equation}
	\frac{1}{r_{qp}} = 
	\frac{1}{\sqrt{\left( q \, \Delta_{x} \right)^{2} + \left( p \, \Delta_{y} \right)^{2} + \Delta_{z}^{2}}} \: .
	\label{eq:1_rqp_y_oriented}
\end{equation}
From equations \ref{eq:aqp_xx_x_oriented}--\ref{eq:1_rqp_y_oriented}, we can easily verify that
\begin{equation}
	\mathbf{A}_{\boldsymbol{xx}}^{q} = \mathbf{A}_{\boldsymbol{xx}}^{(-q)}
	\label{eq:Axx_q_external_block_symmetry}
\end{equation}
and
\begin{equation}
	\mathbf{A}_{\boldsymbol{xx}}^{q} = \left(\mathbf{A}_{\boldsymbol{xx}}^{q} \right)^{\top} \: .
	\label{eq:Axx_q_internal_block_symmetry}
\end{equation}
Note that these symmetries are valid for 
any grid orientation.
From this results we conclude the matrix component 
$\mathbf{A}_{\boldsymbol{xx}}$ is \textit{symmetric-Block-Toeplitz symmetric-Toeplitz-Block} 
for any grid orientation.
The same reasoning can be used to show that matrices $\mathbf{A}_{\boldsymbol{yy}}$ and
$\mathbf{A}_{\boldsymbol{zz}}$ also have this symmetric structure.

%==========================================================
\subsection{Detailed structure of matrix $\mathbf{A_{xy}}$}
%==========================================================

Let $\mathbf{A}_{\boldsymbol{xy}}$ be a matrix component with elements
$a^{xy}_{ij}$ (equation \ref{eq:aij_alpha_beta}) defined by the second derivative calculated in
$h^{xy}_{ij}$ (equation \ref{eq:hxy_regular}). It can be easily verified from equations
\ref{eq:Delta_kij_symmetry}--\ref{eq:1_rij_symmetry} that $h^{xy}_{ij} = h^{xy}_{ji}$.
As a consequence, $a^{xy}_{ij} = a^{xy}_{ji}$, which means that 
\begin{equation}
	\mathbf{A}_{\boldsymbol{xy}} = \left( \mathbf{A}_{\boldsymbol{xy}} \right)^{\top}
	\label{eq:Axy_symmetry}
\end{equation}
for any grid orientation.
For $x$-oriented grids (Figure \ref{fig:regular-grids}), the block indices $q$ and $p$ are defined 
by equations \ref{eq:q-x-oriented} and 
\ref{eq:p-x-oriented} and $a^{xy}_{qp}$ can be rewritten as follows:
\begin{equation}
	a^{xy}_{qp} = c_{m} \, \frac{\mu_{0}}{4\pi} 
	\left( F_{x} u_{y} + F_{y} u_{x} \right) \frac{3 \left( p \, \Delta_{x} \right)\left( q \, \Delta_{y} \right)}{r_{qp}^{5}}
	\: ,
	\label{eq:aqp_xy_x_oriented}
\end{equation}
with $\tfrac{1}{r_{qp}}$ defined by equation \ref{eq:1_rqp_x_oriented}.
For $y$-oriented grids (Figure \ref{fig:regular-grids}), the block indices $q$ and $p$ are 
defined by equations \ref{eq:q-y-oriented} and 
\ref{eq:p-y-oriented} and $a^{xy}_{qp}$ can be rewritten as follows:
\begin{equation}
	a^{xy}_{qp} = c_{m} \, \frac{\mu_{0}}{4\pi} 
	\left( F_{x} u_{y} + F_{y} u_{x} \right) \frac{3 \left( q \, \Delta_{x} \right)\left( p \, \Delta_{y} \right)}{r_{qp}^{5}} \: ,
	\label{eq:aqp_xy_y_oriented}
\end{equation}
with $\tfrac{1}{r_{qp}}$ defined by equation \ref{eq:1_rqp_y_oriented}.
From equations \ref{eq:1_rqp_x_oriented}, \ref{eq:1_rqp_y_oriented}, \ref{eq:aqp_xy_x_oriented} 
and \ref{eq:aqp_xy_y_oriented}, we can show that
\begin{equation}
	\mathbf{A}_{\boldsymbol{xy}}^{q} = -\mathbf{A}_{\boldsymbol{xy}}^{(-q)}
	\label{eq:Axy_q_external_block_symmetry}
\end{equation}
and 
\begin{equation}
	\mathbf{A}_{\boldsymbol{xy}}^{q} = -\left( \mathbf{A}_{\boldsymbol{xy}}^{q} \right)^{\top} \: .
	\label{eq:Axy_q_internal_block_symmetry}
\end{equation}
Note that these symmetries are valid for any grid orientation.
From this results we conclude the matrix component 
$\mathbf{A}_{\boldsymbol{xy}}$ is \textit{skew symmetric-Block-Toeplitz skew symmetric-Toeplitz-Block} 
for any grid orientation.

%==================================================================================
\subsection{Detailed structure of matrices $\mathbf{A_{xz}}$ and $\mathbf{A_{yz}}$}
%==================================================================================

Let $\mathbf{A}_{\boldsymbol{xz}}$ be a matrix component with elements
$a^{xz}_{ij}$ (equation \ref{eq:aij_alpha_beta}) defined by the second derivative calculated in
$h^{xz}_{ij}$ (equation \ref{eq:hxz_regular}). It can be easily verified from equations
\ref{eq:Delta_kij_symmetry}--\ref{eq:1_rij_symmetry} that $h^{xz}_{ij} = -h^{xz}_{ji}$.
As a consequence, $a^{xz}_{ij} = -a^{xz}_{ji}$, which means that 
\begin{equation}
	\mathbf{A}_{\boldsymbol{xz}} = -\left( \mathbf{A}_{\boldsymbol{xz}} \right)^{\top}
	\label{eq:Axz_symmetry}
\end{equation} 
for any grid orientation.
For $x$-oriented grids (Figure \ref{fig:regular-grids}), the block indices $q$ and $p$ are defined 
by equations \ref{eq:q-x-oriented} and 
\ref{eq:p-x-oriented} and $a^{xz}_{qp}$ can be rewritten as follows:
\begin{equation}
	a^{xz}_{qp} = c_{m} \, \frac{\mu_{0}}{4\pi} 
	\left( F_{x} u_{z} + F_{z} u_{x} \right) \frac{3 \left( p \, \Delta_{x} \right) \Delta_{z}}{r_{qp}^{5}}
	\: ,
	\label{eq:aqp_xz_x_oriented}
\end{equation}
with $\tfrac{1}{r_{qp}}$ defined by equation \ref{eq:1_rqp_x_oriented}.
In this case, we can see that
\begin{equation}
	\mathbf{A}_{\boldsymbol{xz}}^{q} = \mathbf{A}_{\boldsymbol{xz}}^{(-q)}
	\label{eq:Axz_q_external_block_symmetry_x_oriented}
\end{equation}
and 
\begin{equation}
	\mathbf{A}_{\boldsymbol{xz}}^{q} = -\left( \mathbf{A}_{\boldsymbol{xz}}^{q} \right)^{\top} \: .
	\label{eq:Axz_q_internal_block_symmetry_x_oriented}
\end{equation}
This structure is called \textit{symmetric-Block-Toeplitz skew symmetric-Toeplitz-Block} and is 
valid only for $x$-oriented grids.
For $y$-oriented grids (Figure \ref{fig:regular-grids}), the block indices $q$ and $p$ are 
defined by equations \ref{eq:q-y-oriented} and 
\ref{eq:p-y-oriented} and $a^{xz}_{qp}$ can be rewritten as follows:
\begin{equation}
	a^{xz}_{qp} = c_{m} \, \frac{\mu_{0}}{4\pi} 
	\left( F_{x} u_{z} + F_{z} u_{x} \right) \frac{3 \left( q \, \Delta_{x} \right) \Delta_{z}}{r_{qp}^{5}} \: ,
	\label{eq:aqp_xz_y_oriented}
\end{equation}
with $\tfrac{1}{r_{qp}}$ defined by equation \ref{eq:1_rqp_y_oriented}.
Now, we conclude that
\begin{equation}
	\mathbf{A}_{\boldsymbol{xz}}^{q} = -\mathbf{A}_{\boldsymbol{xz}}^{(-q)}
	\label{eq:Axz_q_external_block_symmetry_y_oriented}
\end{equation}
and 
\begin{equation}
	\mathbf{A}_{\boldsymbol{xz}}^{q} = \left( \mathbf{A}_{\boldsymbol{xz}}^{q} \right)^{\top} \: .
	\label{eq:Axz_q_internal_block_symmetry_y_oriented}
\end{equation}
This structure is called \textit{skew symmetric-Block-Toeplitz symmetric-Toeplitz-Block} and is 
valid only for $y$-oriented grids.

The same reasoning can be followed to show that
\begin{equation}
	\mathbf{A}_{\boldsymbol{yz}} = -\left( \mathbf{A}_{\boldsymbol{yz}} \right)^{\top}
	\label{eq:Ayz_symmetry}
\end{equation} 
for any grid orientation. Besides, we can also show that
\begin{equation}
	\mathbf{A}_{\boldsymbol{yz}}^{q} = -\mathbf{A}_{\boldsymbol{yz}}^{(-q)}
	\label{eq:Ayz_q_external_block_symmetry_x_oriented}
\end{equation}
and 
\begin{equation}
	\mathbf{A}_{\boldsymbol{yz}}^{q} = \left( \mathbf{A}_{\boldsymbol{yz}}^{q} \right)^{\top}
	\label{eq:Ayz_q_internal_block_symmetry_x_oriented}
\end{equation}
for $x$-oriented grids (\textit{skew symmetric-Block-Toeplitz symmetric-Toeplitz-Block}), while
\begin{equation}
	\mathbf{A}_{\boldsymbol{yz}}^{q} = \mathbf{A}_{\boldsymbol{yz}}^{(-q)}
	\label{eq:Ayz_q_external_block_symmetry_y_oriented}
\end{equation}
and 
\begin{equation}
	\mathbf{A}_{\boldsymbol{yz}}^{q} = -\left( \mathbf{A}_{\boldsymbol{yz}}^{q} \right)^{\top}
	\label{eq:Ayz_q_internal_block_symmetry_y_oriented}
\end{equation}
for $y$-oriented grids (\textit{symmetric-Block-Toeplitz skew symmetric-Toeplitz-Block}).

%======================================================================================
%\subsection{Standard Conjugate Gradient Least Squares (CGLS) method}
\subsection{Convolutional equivalent layer}
%======================================================================================

The computational cost associated with the classical method to estimate the parameter 
vector $\mathbf{p}$ by solving the linear system \ref{eq:normal-equations} can be very high 
or even prohibitive when dealing with large data sets. In these cases, a well-known alternative
is solving the normal equations (equation \ref{eq:normal-equations}) iteratively by 
using the \textit{standard Conjugate Gradient Least Squares (CGLS) method}:

\begin{algorithm}[H]
	Input: $\mathbf{A}$ and $\mathbf{d}^{o}$.
	
	Output: Estimated parameter vector $\tilde{\mathbf{p}}$.
	
	Set $it = 0$, $\tilde{\mathbf{p}}_{(it)} = \mathbf{0}$, $\mathbf{c}_{(it-1)} = \mathbf{0}$, $\beta_{(it)} = 0$, $\mathbf{s}_{(it)} = \mathbf{d}^{o}$ and $\mathbf{r}_{(it)} = \mathbf{A}^{\top} \mathbf{s}_{(it)}$.
	
	1 - If $it > 0$, $\beta_{(it)} = \dfrac{\| \mathbf{r}_{(it)} \|_{2}^{2}}{\| \mathbf{r}_{(it - 1)} \|_{2}^{2}}$
	
	2 - $\mathbf{c}_{(it)} = \mathbf{r}_{(it)} + \beta_{(it)} \, \mathbf{c}_{(it - 1)}$
	
	3 - $\alpha_{(it)} = \dfrac{{\| \mathbf{r}_{(it)}\|_{2}^{2}}}{\| \mathbf{A} \, \mathbf{c}_{(it)} \|_{2}^{2}}$
	
	4 - $\tilde{\mathbf{p}}_{(it + 1)} = \tilde{\mathbf{p}}_{(it)} + \alpha_{(it)} \, \mathbf{c}_{(it)}$
	
	5 - $\mathbf{s}_{(it + 1)} = \mathbf{s}_{(it)} - \alpha_{(it)} \, \mathbf{A} \, \mathbf{c}_{(it)}$
	
	6 - $\mathbf{r}_{(it + 1)} = \mathbf{A}^{\top} \, \mathbf{s}_{(it + 1)}$
	
	7 - $it = it + 1$
	
	8 - Repeat previous steps until convergence ($\delta$).
	
	\caption{Standard CGLS pseudocode \citep[][ p. 166]{aster2019parameter}.}
	\label{al:std-cgls-algorithm}
\end{algorithm}

Setting a convergence criterion $\delta$ (Algorithm \ref{al:std-cgls-algorithm}) based on the minimum tolerance 
of the residuals is a good option to carry out this algorithm efficiently and still obtaining very good results. 
Another possibility is to set an invariance to the Euclidean norm of residuals between 
iterations, which would increase algorithm runtime, but with smaller residuals. 
We chose the latter option, as we could achieve better results.

Note that the standard CGLS solution (Algorithm \ref{al:std-cgls-algorithm}) requires 
neither inverse matrix nor matrix-matrix product. Instead, it only requires: one matrix-vector 
product out of the loop and two matrix-vector products per iteration (in steps 3 and 6). 
These products can be efficiently computed by using the 2D FFT, as a discrete convolution
(see Appendix A). \citet{takahashi2020convolutional} used this approach
to develop an efficient algorithm for gravity data processing. This modified approach in which
the standard CGLS method is modified to efficiently compute the matrix-vector products will be 
referenced throughout this work as the \textit{convolutional equivalent layer method}.

%======================================================================================
\subsection{Computational performance}
%======================================================================================

In this section we compare the efficiency of the classical (equation \ref{eq:classical-method}), 
standard CGLS (Algorithm \ref{al:std-cgls-algorithm}) and the convolutional equivalent 
layer method (Algorithm \ref{al:std-cgls-algorithm} with matrix-vector products computed 
according to Appendix A). To do this, we compute the total number of 
\textit{flops} associated to them \citep[][ p. 12]{golub-vanloan2013}.

For the classical method, we have $\tfrac{1}{2} N^3$ flops to compute the lower triangle of
$\mathbf{A}^{\top}\mathbf{A}$; $\tfrac{1}{3} N^3$ flops to compute the Cholesky factor
$\mathbf{G}$ of $\mathbf{A}^{\top}\mathbf{A}$ \citep[][ p.~164]{golub-vanloan2013};
$2 \, N^2$ flops to compute the matrix-vector product $\mathbf{A}^{\top} \mathbf{d}^{o}$;
and $2 \, N^2$ flops to solve the triangular systems given by equation \ref{eq:classical-method}
\citep[][ p.~106]{golub-vanloan2013}. The resultant flop count for the classical method is
\begin{equation}
	f_{classical} =  \dfrac{5}{6} N^{3} + 4 \, N^{2}\: .
	\label{eq:flops-classical-method}
\end{equation}

%======================================================================================
%\subsection{CGLS flops count}
%======================================================================================

For the standard CGLS method (Algorithm \ref{al:std-cgls-algorithm}) we have $2 \, N^2$ to compute
the matrix-vector product $\mathbf{A}^{\top} \mathbf{s}_{(it)}$ out of the loop;
$4 \, N$ in step 1; $2 \, N$ in step 2; $2 \, N^2 + 2 \, N$ in step 3; $2 \, N$ in step 4;
$2 \, N$ in step 5; and $2 \, N^2$ in step 6. The resultant flop count is given by:
\begin{equation}
	f_{cgls} =  2 N^{2} + it \, (4 N^{2} + 12 N) \: .
	\label{eq:flops-standard-cgls}
\end{equation}

%======================================================================================
%\subsection{Our modified CGLS flops count}
%======================================================================================

To compute the flops count of our method, we need only to replace the flops associated with 
matrix-vector products in the standard CGLS method by those associated with
2D convolution defined in Appendix A, which consists of $\kappa  \, 4 N \log_2(4N)$ flops to
compute the 2D FFT for each matrix $\mathbf{L}_{\boldsymbol{\alpha\beta}}$ (equation 
\ref{eq:L_alpha_beta}); $\kappa  \, 4 N \log_2(4N)$ flops to compute 
$\mathbf{F}_{2Q} \, \mathbf{V} \, \mathbf{F}_{2P}$ via 2D FFT; $24 \, N$ flops to compute the 
Hadamard product $\mathbf{L} \circ \left(\mathbf{F}_{2Q} \, \mathbf{V} \, \mathbf{F}_{2P} \right)$; 
and $\kappa  \, 4 N \log_2(4N)$ flops to compute the IDFT (inverse discrete Fourier transform) in equation 
\ref{eq:2d-discrete-convolution-complete}. We use $\kappa = 5$ for the \emph{radix-2} algorithm
\citep[][ p.~15]{vanloan1992}. By replacing these flops into Algorithm \ref{al:std-cgls-algorithm},
we obtain the complete number of flops
\begin{equation}
	f_{conv} =  \kappa  \, 16 N \log_2(4 N) + 24 N + it \, (\kappa  \, 16 N \log_2 (4 N) + 60 N) \: .
	\label{eq:flops-convolutional-method}
\end{equation}

Figure \ref{fig:flops} shows a comparison between 
$f_{classical}$ (equation \ref{eq:flops-classical-method}), 
$f_{cgls}$ (equation \ref{eq:flops-standard-cgls}) and 
$f_{conv}$ (equation \ref{eq:flops-convolutional-method})
for different numbers of observation points up to $1,000,000$. As we can see, 
the total flops count associated with our method is $7$ orders of magnitude smaller 
than that associated with the classical method and $3$ orders of magnitude smaller than
that associated with the standard CGLS method when using a maximum number
of iterations $N^{it} = 50$. 

Figure \ref{fig:solve_time} shows the time necessary to build matrix $\mathbf{A}$ 
(equation \ref{eq:A_expand}) and solve the linear system for $N$ varying up to $10,000$. 
With $N = 10,000$, the classical method takes more than sixty-three seconds, the standard 
CGLS more than twelve seconds, while our method takes only half a second. 
The CPU used for this test was a Intel Core i7-7700HQ@2.8GHz.

%In Figure \ref{fig:sources_time} a comparison between the time to complete the task to calculate the first column of the BCCB matrix embbeded from the from $\mathbf{A}$ (equation \ref{eq:aij_mag}) by using only one equivalent source, i.e., calculating all six first column of the second derivatives matrices from $\mathbf{H}$ (equation \ref{eq:Hi}) and using four equivalent sources to calculate the four necessary columns from the non-symmetric matrix $\mathbf{A}$ (equation \ref{eq:aij_mag}). Although, very similar in time, with one source a small advantage can be observed as the number of data $N$ increases and goes beyond $N = 200,000$. This test was done from $N = 10,000$ to $N = 700,000$ with increases of $5,625$ observation points.

Table \ref{tab:RAM-usage} shows a comparison between the RAM memory storage 
associated with each method. In the classical and standard CGLS methods, the whole 
matrix $\mathbf{A}$ (equation \ref{eq:A_expand}) has to be stored. For example, a dataset with 
$N = 10,000$ observation points has an associated sensitivity matrix $\mathbf{A}$ formed by 
$N^2 = 100,000,000$ elements and takes approximately $763$ Megabytes of memory (8 bytes per element). 
Using the same number of observation points $N = 10,000$, our method requires only 
$1.831$ Megabytes to store the first columns of the BCCB matrices
$\mathbf{C}_{\boldsymbol{\alpha\beta}}$ (equation \ref{eq:w_alpha_beta}) and 
$0.6104$ Megabytes to store the complex matrix $\mathbf{L}$ (equation \ref{eq:L}) 
(16 bytes per element). For a bigger dataset with $N = 1,000,000$, the amount of necessary RAM 
goes to $7,629,395$, $183.096$ and $61.035$ Megabytes, respectively.

\section{Application to synthetic data}

Our convolutional equivalent layer method requires a regular data grid located on a 
horizontal and flat observation surface.
Here, we evaluate the performance of our method by applying it to simulated airborne magnetic 
surveys formed by
i) a regular data grid on a flat surface;
ii) irregular data grids on a flat surface; and 
iii) regular data grid on undulating surfaces.
Note that the simulated surveys in (ii) and (iii) violate the premises of our method. 

%=======================================
\subsection*{Simulated airborne surveys}
%=======================================

The first and second rows in Figure \ref{fig:synthetic_data_comparison_v2} show, respectively, 
the simulated flight patterns and noise-corrupted total-field anomalies of the airborne magnetic 
surveys used in our tests. The third row in Figure \ref{fig:synthetic_data_comparison_v2} shows 
the true upward-continued total-field anomalies at $z = -1, \, 300$ m. The fourth row in Figure \ref{fig:synthetic_data_comparison_v2} shows the true reduced to pole total-field anomalies.
All magnetic data (second and lower rows in Figure \ref{fig:synthetic_data_comparison_v2}) 
are produced by the same three synthetic bodies: two prisms and one sphere with 
constant total-magnetization vector having inclination, declination and intensity of 
$35.26^{\circ}$, $45^{\circ}$, and $3.4641$ A/m, respectively. 
The simulated main geomagnetic field has inclination and declination of $35.26^{\circ}$ and $45^{\circ}$,
respectively. 

%% regular grid on a flat surface

Figure \ref{fig:synthetic_data_comparison_v2}a shows the simulated airborne survey on
a regular grid of $100 \times 50$ observation points (totaling  $N = 5,\, 000$ observation points),
with a grid spacing of $\Delta x = 101.01$ m and $\Delta y = 163.265$ m along the
$x$- and $y$-axis, respectively.
The noise-corrupted total-field anomaly (second panel of Figure \ref{fig:synthetic_data_comparison_v2}a) 
is calculated at $z = -900$ m, with pseudorandom Gaussian noise added having null mean and standard deviation of $0.2961$ nT.

%% irregular grids on a flat surface

Figures \ref{fig:synthetic_data_comparison_v2}b and \ref{fig:synthetic_data_comparison_v2}c 
show the simulated surveys on irregular grids obtained by perturbing the horizontal coordinates
of the regular grid (upper panel in Figure \ref{fig:synthetic_data_comparison_v2}a).
For the survey shown in Figure \ref{fig:synthetic_data_comparison_v2}b, the $x$ and $y$ coordinates 
are perturbed with sequences of pseudorandom Gaussian noises having null mean and standard deviations
equal to $20\%$ of the corresponding grid spacing, which results in
absolute values of $20.2$ m and $32.6$ m, along the $x$- and $y$-directions, respectively.
For the survey shown in Figure \ref{fig:synthetic_data_comparison_v2}c, the standard deviations
are equal to $30\%$ of the corresponding grid spacing, which results in absolute values of 
$30.3$ m and $49.0$ m along the $x$- and $y$-directions, respectively.
Their noise-corrupted total-field anomalies (second panels in Figures 
\ref{fig:synthetic_data_comparison_v2}b and \ref{fig:synthetic_data_comparison_v2}c) are calculated 
on their corresponding irregular grids, on a flat observation surface at $z = -900$ m, 
with pseudorandom Gaussian noise added having null mean and standard deviation of $0.2961$ nT.

%% regular grid on undulating surfaces

Figures \ref{fig:synthetic_data_comparison_v2}d and \ref{fig:synthetic_data_comparison_v2}e 
show the simulated surveys on the same regular grid as shown in Figure 
\ref{fig:synthetic_data_comparison_v2}a (upper panel). The difference is that observation points
are located no longer on a flat, but on undulating surfaces.
For the survey shown in Figure \ref{fig:synthetic_data_comparison_v2}d, the $z$ coordinates 
of the undulating surface are defined by a sequence of pseudorandom Gaussian noise having mean 
$-900$ m and standard deviation equal to $5\%$ of $900$ m, which corresponds to $45$ m.
For the survey shown in Figure \ref{fig:synthetic_data_comparison_v2}e, the standard deviation 
is equal to $10\%$ of $900$ m, which corresponds to $90$ m.
The noise-corrupted total-field anomalies of these simulated surveys (second panels in Figures 
\ref{fig:synthetic_data_comparison_v2}d and \ref{fig:synthetic_data_comparison_v2}e) are calculated 
on their corresponding undulating surfaces (upper panels in Figures 
\ref{fig:synthetic_data_comparison_v2}d and \ref{fig:synthetic_data_comparison_v2}e),
on the same regular grid shown in Figure \ref{fig:synthetic_data_comparison_v2}a,
with pseudorandom Gaussian noise added having null mean and standard deviation of $0.2961$ nT.


%% Regular grid
%======================================================================================
\subsection*{Tests with a regular data grid on a flat surface}
%======================================================================================

Figure \ref{fig:synthetic_residuals_convergence_comparison_v2} show the 
difference between the simulated (second row in Figure \ref{fig:synthetic_data_comparison_v2})
and predicted data (not shown) obtained by using the classical (the upper row) and 
our method (the second row). From now on, we designate this difference as data residuals. 
The lower row in Figure \ref{fig:synthetic_residuals_convergence_comparison_v2} shows the 
convergence curve of our method.

The data residuals using the classical method (equation \ref{eq:classical-method})  
are shown in the upper panel of Figure \ref{fig:synthetic_residuals_convergence_comparison_v2}a, 
with mean $0.4118$ nT and standard deviation $0.3780$ nT. This process took $17.10$ seconds.
Using our method, the data residuals (the middle panel in Figure
\ref{fig:synthetic_residuals_convergence_comparison_v2}a) have mean $0.9972$ nT and standard
deviation $1.3904$ nT. In this case, however, the processing time was only $0.25$ seconds.
As expected, the Euclidean norm of the data residuals produced by our method 
(lower panel in Figure \ref{fig:synthetic_residuals_convergence_comparison_v2}a) decreases. 
The convergence criterion was satisfied close to iteration $50$.
%This result shows that, in practice, our method converges way before $N$ iterations,
%where $N$ is the number of observations.
%Setting the convergence to $N$ iterations, besides being unnecessary, it also demands a larger 
%computer processing time, even in this synthetic test with a small number 
%of $N = 5,\, 000$ observations.

%======================================================================================
\subsection*{Tests with irregular data grids on a flat surface}
%======================================================================================

Figure \ref{fig:synthetic_residuals_convergence_comparison_v2}b shows the results obtained
with the irregular data grid perturbed by using $20\%$ of the regular grid spacing.
In this Figure we can see that the data residuals 
using the classical method (upper panel) yield a good data fit with mean $0.4084$ nT and standard
deviation $0.3862$ nT. 
Using our method, the data residuals (middle panel in Figure 
\ref{fig:synthetic_residuals_convergence_comparison_v2}b) also produced an acceptable data 
fitting with mean of  $1.3125$ nT and standard deviation of $1.7187$ nT. 
The Euclidean norm of the data residuals obtained by our method 
(lower panel in Figure \ref{fig:synthetic_residuals_convergence_comparison_v2}b) decreases, 
as expected, and converges to a constant value close to iteration $50$. 

Figure \ref{fig:synthetic_residuals_convergence_comparison_v2}c shows the results obtained
with the irregular data grid perturbed by using $30\%$ of the regular grid spacing.
The data residuals 
obtained by the classical method (upper panel in Figure \ref{fig:synthetic_residuals_convergence_comparison_v2}c) produced an acceptable data fit, having mean 
$0.4070$ nT and standard deviation $0.3899$ nT. 
%Using our method, the data residuals (middle panel in Figure 
%\ref{fig:synthetic_residuals_convergence_comparison_v2}c) with mean $0.9788$ nT and 
%standard deviation $0.7462$ nT produced a worse data fitting.
%The convergence of our method (lower panel in Figure 
%\ref{fig:synthetic_residuals_convergence_comparison_v2}c) shows that, 
%similarly to the previous results, the Euclidean norm of the residuals decreases; however it starts
%increasing without achieving an invariance. Hence, the convergence is not achieved. 
%
%\textbf{Alternative text} $\rightarrow$
%
Using our method, the data residuals (middle panel in Figure 
\ref{fig:synthetic_residuals_convergence_comparison_v2}c) with mean $1.5129$ nT and 
standard deviation $1.8526$ nT also produced a good data fitting.
The convergence of our method (lower panel in Figure 
\ref{fig:synthetic_residuals_convergence_comparison_v2}c) shows that, 
similarly to the previous results, the Euclidean norm of the residuals decreases; converging 
to a constant value close to iteration $50$. Note that this good result was obtained by 
using a very perturbed data grid (upper panel in Figure \ref{fig:synthetic_data_comparison_v2}c).
%
%$\leftarrow$ \textbf{Alternative text}

%======================================================================================
\subsection*{Tests with regular data grid an undulating surfaces}
%======================================================================================

Figure \ref{fig:synthetic_residuals_convergence_comparison_v2}d shows the results obtained
with data on the undulating surface varying $5\%$ of $z = 900$ m.
In this case, the data residuals either using the classical method 
(upper panel in Figure \ref{fig:synthetic_residuals_convergence_comparison_v2}d) or
our method (middle panel in Figure \ref{fig:synthetic_residuals_convergence_comparison_v2}d) reveal
acceptable data fittings.
Using the classical method, data residuals have mean $0.4316$ nT and standard deviation $0.4762$ nT.
Using our method, they have mean $2.1069$ nT and standard deviation $2.5023$ nT. 
Likewise, the Euclidean norm of the data residuals produced by our method 
(lower panel in Figure \ref{fig:synthetic_residuals_convergence_comparison_v2}d) decreases up to 
iteration 50 and reaches the convergence criterion in the subsequent iterations (mean residulas are less than 
0.00015 between iterations).

Figure \ref{fig:synthetic_residuals_convergence_comparison_v2}e shows the results obtained
with data on the undulating surface varying $10\%$ of $z = 900$ m.
By using the classical approach, the data residuals (upper panel in 
Figure \ref{fig:synthetic_residuals_convergence_comparison_v2}e) 
yielded a good data fitting, with mean $0.4818$ nT and standard deviation $0.6565$ nT. 
By using our method, the data residuals (middle panel in 
Figure \ref{fig:synthetic_residuals_convergence_comparison_v2}e) yielded a worse data fitting 
with mean $3.4981$ nT and standard deviation $3.8153$ nT.
The convergence curve (lower panel in Figure \ref{fig:synthetic_residuals_convergence_comparison_v2}e)
reveals the inadequacy of our method in dealing with observations on rugged surfaces, as 
the Euclidean norm of the data residuals do not decrease as much as in previous tests. 
We stress that, in this test, the undulating surface (upper panel in Figure 
\ref{fig:synthetic_data_comparison_v2}e) varies in a broad range of flight values, from $z = - 570$ m to about 
$z = -1,\, 230$ m. Thus, this simulated airborne magnetic survey greatly violates the requirement 
of a flat observation surface demanded by our method.

Although our method is formulated to deal with magnetic observations measured on 
a horizontally regular grid, on a flat surface, the results obtained with synthetic 
data show that our method is robust in dealing either with irregular grids in the 
horizontal directions or with uneven surfaces.
However, the robustness of our method has limitations.
%The performance limitation of our method depends on the degree of 
%departure of the (i) $x$- and $y$-coordinates from those of the closest 
%regular grid and (ii) the $z$ coordinates from a constant value.
High discrepancies in the $x$-, $y$, and $z$-coordinates lead to unacceptable 
data fittings (large data residuals), as shown the middle panels in Figures 
\ref{fig:synthetic_residuals_convergence_comparison_v2}c and
\ref{fig:synthetic_residuals_convergence_comparison_v2}e.

%======================================================================================
\subsection*{Magnetic data processing}
%======================================================================================

We performed the upward continuations of the synthetic total-field anomalies 
(second row in Figure \ref{fig:synthetic_data_comparison_v2}) by using 
the classical method, our convolutional equivalent layer method, and 
the classical approach in the Fourier domain,
which consists in computing the Fourier transform of the total-field anomaly 
\citep[e.g.,][ p. 317]{blakely1996}. 

Figure \ref{fig:synthetic_upward_residuals_comparison_v2} shows the continuation residuals defined as the differences
between the true upward-continued total-field anomalies (third row in Figure
\ref{fig:synthetic_data_comparison_v2}) and the predicted upward-continued total-field 
anomalies (not shown). We conveniently denote these differences as continuation
residuals.
%% Classical and our method 
The continuation residuals 
obtained by using the classical method (upper row) and our method (middle row) are 
similar to each other in most of the tests.
The exceptions are the synthetic test with data over irregular grid (Figures \ref{fig:synthetic_data_comparison_v2}c and \ref{fig:synthetic_residuals_convergence_comparison_v2}c) and over an undulating surface 
(Figures \ref{fig:synthetic_data_comparison_v2}e and 
\ref{fig:synthetic_residuals_convergence_comparison_v2}e), which greatly violates the 
requirement of regular grids or a flat observation surface, demanded by our method.
Note that the maximum absolute value of the continuation residuals produced by using our 
method (middle panel in Figure \ref{fig:synthetic_upward_residuals_comparison_v2}e) 
are $\approx 2$ times greater than those produced by the classical method 
(upper panel in Figure \ref{fig:synthetic_upward_residuals_comparison_v2}e).
%Besides, they are generally concentrated at the boundaries of the study area.

%%  Fourier 
In contrast, the continuation residuals obtained by using the 
classical Fourier approach (lower row in Figure \ref{fig:synthetic_upward_residuals_comparison_v2})
are, in most of the tests, approximately $2$ times greater than those produced by the classical method 
(upper row in Figure \ref{fig:synthetic_upward_residuals_comparison_v2}) and $1.5$ times greater than
those produced by our method (middle row in Figure \ref{fig:synthetic_upward_residuals_comparison_v2}).
Note that, similar to our method, the maximum absolute values of the continuation residuals 
obtained by using the classical Fourier approach are located at the boundaries of the simulated area.
However, the values are significantly higher.

%% Reduction to pole
Figure \ref{fig:synthetic_zrtp_residuals_comparison_v2} shows the differences
between the true reduced to pole total-field anomalies (fourth row in Figure
\ref{fig:synthetic_data_comparison_v2}) and the predicted reduced to pole total-field 
anomalies (not shown). The true reduced to pole total-field anomalies are generated by using only induced magnetization, with $I_{0} = 90^{\circ}$ and $D_{0} = 0^{\circ}$.
%% Classical and our method 
Figure \ref{fig:synthetic_zrtp_residuals_comparison_v2} shows that the reduced to pole residuals 
obtained by using the classical method (upper row) and our method (middle row) have differences when high irregular grids or non flat surfaces are used (Figures \ref{fig:synthetic_zrtp_residuals_comparison_v2}c and \ref{fig:synthetic_zrtp_residuals_comparison_v2}e). The absolute values of the reduced to pole residuals are almost $\approx 2$ times greater than those of classical method when the $10\%$ standard deviation was used (upper and middle panels in Figure \ref{fig:synthetic_zrtp_residuals_comparison_v2}e, respectively).
As in the the continuation test, they are generally concentrated at the boundaries of the study area.

%%  Fourier 

The reduced to pole residuals obtained by using the 
classical Fourier approach (lower row in Figure \ref{fig:synthetic_zrtp_residuals_comparison_v2})
are approximately $3.5$ times greater than those produced by the classical method 
(upper row in Figure \ref{fig:synthetic_zrtp_residuals_comparison_v2}) and $3$ times greater than
those produced by our method (middle row in Figure \ref{fig:synthetic_upward_residuals_comparison_v2}).

Important to note that the reduction to pole, either using the equivalent layer or the Fourier approach, has the requirement of a previously knowledge of the sources magnetization directions (equation \ref{eq:u_hat}) to obtain a correct source parameter estimative, otherwise, only non-phase dependent processing can be used (upward continuation for example).

We also call attention to the following aspects:
In applying the classical method, our method, or the classical Fourier approach, we do not expand 
the data by using a padding scheme.
The data residuals (upper and middle rows in Figure 
\ref{fig:synthetic_residuals_convergence_comparison_v2}), 
the continuation (Figure \ref{fig:synthetic_upward_residuals_comparison_v2}) and reduction to pole residuals (Figure \ref{fig:synthetic_zrtp_residuals_comparison_v2}) are shown without removing edge effects. 
The computational time required by our method is much lower than that required by the classical method
and has the same order of magnitude of that required by the classical Fourier approach.
However, the classical Fourier approach shows upward-continued and reduced to pole data with strong border effects if no padding scheme is applied to expand the data.


\section{Application to field data}

We applied the convolutional equivalent layer method to the aeromagnetic data of Carajás, 
northern Brazil.
The survey is composed of $131$ flight lines along north-south direction with line spacing of 
$\Delta y = 3,000$ m. 
Data were measured with average spacing $\Delta x = 7.65$ m along lines,
with an average distance to the ground of $900$ m. 
The total number of observation points is $N = 6,081,345$. Figure \ref{fig:carajas_residuals_comparison}a
shows the observed total-field anomaly data over the study area.

We compare the results obtained with an interpolated regular grid of $10,000 \times 131$ points, 
by using the nearest neighbor algorithm, and a decimated irregular grid, also with $10,000 \times 131$
points. In both cases the total $N = 1,310,000$ observation points are in the original undulating surface of the flight lines. 
The decimated grid was generated by 
choosing the nearest observation points in comparison of the regular grid presented in the interpolation.
The mean and standard deviation of this irregular decimated from the regular interpolated are $6.8386$ m and $107.7343$ m in the $x$-direction and $30.8799$ m and $28.3849$ m in the $y$-direction, respectively.
Both application were made with an Intel core i7 7700HQ@2.8GHz processor in single-processing and 
single-threading modes. 

As the study area is very large, the main magnetic field varies with position.
For this application, we set the main field direction as that of a mid location 
(latitude $-6.5^{\circ}$ and longitude $-50.75^{\circ}$) where the declination is $-19.86^{\circ}$ and
the inclination is $-7.4391^{\circ}$. Both values were calculated using the magnetic field calculator from NOAA
at 1st January, 2014 (epoch of the survey). We set the equivalent layer at $300$ meters above the ground ($600$ m below the data).
Figure \ref{fig:carajas_residuals_comparison}b shows the residuals obtained after using our method to fit
the interpolated data with mean $0.0762$ nT and the standard deviation  $0.4886$ nT, revealing an acceptable data fitting. Our method took $\approx 390.80$ seconds to converge at about $200$ iterations.
Figure \ref{fig:carajas_residuals_comparison}c shows the residuals obtained after using our method to fit
the decimated data with mean $0.0717$ nT and standard deviation 
$0.3144$ nT with a better fit than the produced by the interpolated data. In this case, our method took $\approx 385.56$ seconds to converge at about $200$ iterations (Figure \ref{fig:carajas_residuals_comparison}d). 
The convergence curve reveals a good convergence rate obtained with the decimated 
irregular grid. This result shows the robustness of our method in processing irregular grid. Notice that we used $200$ iterations in our method of the interpolated regular grid and the mean residual still decreasing up to $2000$. This happens because the invariance convergence criterion was met as the mean residuals are very small and each iteration decreases less than $0.00001$

With $1,310,000$ observation points, it would be necessary $12.49$ Terabytes of RAM to store the full
sensitivity matrix with the classical method. 
In this case, our method uses only $59.97$ Megabytes, allowing regular desktop computers to be able 
to process this amount of data.


%
%\textbf{Alternative text} $\rightarrow$
%
%As the study area is very large, the main magnetic field varies with position.
%For this application, we set the main field direction as that of a mid location 
%(latitude $-6.5^{\circ}$ and longitude $-50.75^{\circ}$) where the declination is $-19.86^{\circ}$ 
%and inclination is $XXXXXX^{\circ}$ according to IGRF model at 1st January, 2014 (epoch of the survey).
%(É ESTRANHO CALCULAR A DECLINAÇÃO COM UM MODELO E A INCLINAÇÃO COM OUTRO)
%
%$\leftarrow$ \textbf{Alternative text}
%

%(Figure \ref{fig:carajas_real_data_decimated_gridline}a), we obtain the  predicted data shown in 
%Figure \ref{fig:carajas_gz_predito_mag_gridline}a and data residuals 
%(Figure \ref{fig:carajas_gz_predito_mag_gridline}b) with mean $0.0762$ nT and the standard deviation 
%$0.4886$ nT, revealing an acceptable data fitting.
%Our method took $\approx 390.80$ seconds to converge at about $200$ iterations.
%
%POR QUE A CURVA DE CONVERGENCIA NÃO FOI MOSTRADA?
%

%By applying our method to the decimated irregular grid 
%(Figure \ref{fig:carajas_real_data_decimated_gridline}b), we obtain the predicted data shown in 
%Figure \ref{fig:carajas_gz_predito_mag_decimated}a and data residuals 
%(Figure \ref{fig:carajas_gz_predito_mag_decimated}b) with mean $0.0717$ nT and standard deviation 
%$0.3144$ nT. In this case, our method took $\approx 385.56$ seconds 
%to converge at about $200$ iterations (Figure \ref{fig:convergence_carajas_mag_decimated}).
%(TEM ALGO ESTRANHO AQUI. COMO QUE 2000 ITERAÇÕES FOI MAIS MAIS RÁPIDO DO QUE AS 200 ITERAÇÕES DO GRID INTERPOLADO?).

%The convergence curve reveals a good convergence rate obtained with the decimated 
%irregular grid. This result shows the robustness of our method in processing irregular grid. Notice that we used in our %method of the interpolated regular grid $200$ iterations and the mean residual still decreasing up to $2000$. This happens %because the invariance convergence criterion was met as the mean residuals are very small and each iteration decreases less %than $0.00001$

%We have found, in applying our method to the decimated irregular grid, that the data residual amplitude (Figure \ref{fig:carajas_gz_predito_mag_decimated}b) is lower than the data residual amplitude (Figure 
%\ref{fig:carajas_gz_predito_mag_gridline}b) obtained by applying our method to the interpolated regular grid.
%It occurs because  the  process of decimating the original irregularly data creates neither new observation points nor new data. Rather, the interpolation of the original irregularly data creates either new observation points or new data. 

Finally, Figure \ref{fig:carajas_upward_comparison}a shows the upward-continued magnetic data to a
horizontal plane located at an altitude of $5, \,000$ m using the estimated equivalent layer obtained by applying our
method to the decimated irregular grid. This process took $\approx 2.64$ seconds, showing good results 
without visible errors or border effects. 
Figure \ref{fig:carajas_upward_comparison}b shows the upward-continued magnetic data to a
horizontal plane located at an altitude of $5, \,000$ m using the classical Fourier filtering method to the decimated irregular grid.
This process took $\approx 0.5$ seconds. Altough, no border effect could be observed, this result showed unexpected high values instead of the expected attenuation of the anomalies. We stress that we did not use a padding scheme to expand the data.

\section{Conclusions}

We have proposed a fast equivalent-layer technique for processing magnetic data called
convolutional equivalent layer method.
We have demonstrated that the sensitivity matrix associated with planar 
equivalent layers of dipoles has a BTTB structure for the particular case in which 
the dipoles are aligned with the horizontal and regular grid of magnetic data.
The product of such matrices and arbitrary vectors represents a 2D discrete convolution
that can be efficiently computed via 2D Fast Fourier Transform by using only the 
elements forming the first column of the matrix.
By using this property, we have developed a fast and memory efficient iterative method for 
estimating the physical-property distribution on the equivalent layer.

Comparisons between the estimated physical-property distribution obtained with our method and 
the classical approach that solves the least-squares normal equations via Cholesky decomposition 
show similar results. 
The differences in total number of floating-point operations (flops), memory usage and computation 
time, however, are noticeable. 
For a mid-size grid of $100 \times 50$ points, the total number of flops is about four orders of
magnitude smaller than that required by the classical method. Besides, our method uses
less than $1\%$ of the RAM and takes about $3\%$ of the computation time associated with
the classical method in this case.% (TEM QUE CHECAR ESTAS VALORES)
Significantly better results can be obtained with larger data sets.

Tests with synthetic data show that the computational time required by our method has 
the same order of magnitude of that required by the classical approach in the Fourier domain
to perform magnetic data processing.
However, the classical Fourier approach shows considerable larger border effects if no previous 
padding scheme is used to expand the data. 
Besides, although both methods require the magnetic data be on a planar and regular grid, 
tests with synthetic data show the robustness of our method to deal with data either on irregular grids
or on undulating observation surfaces.

While the classical equivalent-layer method would require $12.49$ Terabytes of RAM to store the full
sensitivity matrix associated with the irregular grid of $1,310,000$ observation points over the
Carajás Province, northern Brazil, our method requires only $59.97$ Megabytes.
When performed on a standard laptop computer with an Intel Core i7 7700HQ@2.8GHz processor in
single-processing and single-threading modes, the total times spent by our method to estimate the
physical-property distribution over the equivalent layer and to compute the upward continuation
of the $1,310,000$ magnetic observations over the Carajás province was approximately $385.56$ 
seconds and $2.64$ seconds.

Further investigation could usefully explore different preconditioning strategies to improve the
convergence rate of our method. Besides, considerably more work will need to be done to generalize
our convolutional equivalent layer method for dealing with irregularly spaced data sets on 
undulating observation surfaces.

\clearpage

\section{Figures}

%% Methodology
%\plot{Figure1}{width=\textwidth}{
%	{Schematic representation of an $N_{x} \times N_{y}$ regular grid of points (black dots) defined by 
%	$N_{x} = 4$ and $N_{y} = 3$. The grids are oriented along the (a) $x$-axis and (b) $y$-axis. The grid 
%	coordinates are $x_{k}$ and $y_{l}$, where the $k = 1, \dots, N_{x}$ and $l = 1, \dots, N_{y}$ are 
%	called the grid indices. The insets show the grid indices $k$ and $l$.}
%	\label{fig:methodology}
%}

\plot{schematic_regular_grids}{width=\textwidth}{
	{Schematic representation of an $N_{x} \times N_{y}$ regular grid of points (black dots) with
		$N_{x} = 3$ and $N_{y} = 2$, where each point has an associated index. This index may represent
		$i$ or $j$, that are associated with observation points $(x_{i}, y_{i}, z_{0})$ and 
		equivalent sources $(x_{j}, y_{j}, z_{c})$. Left panel shows an example of $x$-oriented grid,
		with indices varying along $x$-axis, while right panel shows an example of $y$-oriented grid, 
		with indices varying along $y$-axis.}
	\label{fig:regular-grids}
}

%\plot{4_equivalent_sources}{width=\textwidth}{
%	{Representation of the four equivalent sources (black dots) needed to reconstruct the non-symmetric matrix $\mathbf{A}$ (equation \ref{eq:aij_mag}). Each of the equivalent sources are located in the corner of the simulated regular grid of $M_x = 4$ and $M_y = 3$ sources. The influence of these sources on each of the observation points (blue dots) i the regular grid of $N_x = 4$ and $N_y = 3$ will give the four columns necessary of matrix $\mathbf{A}$.}
%	\label{fig:4_equivalent_sources}
%}

%% Computational performance

\plot{flops_mag}{width=\textwidth}{
	{Number of flops associated with classical method (equation \ref{eq:flops-classical-method}), 
		the standard CGLS method (equation \ref{eq:flops-standard-cgls}) and our method (equation 
		\ref{eq:flops-convolutional-method}, all of them with $N^{it} = 50$. 
		The number of observation points $N$ varies from $5,000$ to $1,000,000$.}
	\label{fig:flops}
}

\plot{time_comparison_mag}{width=\textwidth}{
	{Comparison between the runtime of the equivalent-layer technique using the classical method, 
		standard CGLS method and our method. The values for the standard CGLS and our method use
		$N^{it} = 50$ iterations.}
	\label{fig:solve_time}
}

%\plot{time_sources_mag}{width=\textwidth}{
%	{Comparison between the runtime to calculate the first column of the BCCB matrix embbeded from $\mathbf{A}$ (equation \ref{eq:aij_mag}) using only one and using four equivalent sources. Although the time is very similar, with one source a small advantage can be observed as the number of data $N$ increases. This test was done from $N = 10,000$ to $N = 700,000$ with increases of $5,625$ observation points.}
%	\label{fig:sources_time}
%}

%% Synthetic data part I

\plot{synthetic_data_comparison_v2}{width=9cm}{
	{Synthetic tests: the simulated airborne magnetic surveys  - 
		The first row shows the grids of observation points and the undulating observation surfaces that simulate the airborne magnetic surveys. 
		The second row shows the noise-corrupted total-field anomalies produced by the synthetic sources and calculated on the simulated airborne magnetic survey shown in the first row. 
		The third row shows the  noise-free total-field anomalies produced by the synthetic sources 
		at $z = -1,300$ m (the true upward-continued total-field anomalies).
		The fourth row shows the noise-free total-field anomalies produced by the synthetic sources at inclination $I_{0} = 90\circ$ (the true reduced to pole total-field anomalies).
		The results shown in these last three rows were obtained by using the simulated airborne magnetic surveys as follows:
		(a) A regular grid of $100 \times 50$ observation points in the $x-$ and $y-$directions and a flat observation surface at $z= -900$ m. 
		An irregular grid with uncertainties of (b) $20\%$ and (c) $30\%$   in the $x-$ and $y-$coordinates and a flat observation surface at $z= - 900$ m . 
		A regular grid  of $100 \times 50$ observation points in the $x-$ and $y-$coordinates and an undulating observation surface with uncertainties of (d) $5\%$ and  (e) $10\%$.
		The black lines represent the horizontal projection of the sources
		.}
	\label{fig:synthetic_data_comparison_v2}
}

\plot{synthetic_residuals_convergence_comparison_v2}{width=\textwidth}{
	{Synthetic tests: the data residuals and convergence - 
		The first row shows the data residuals using the classical method.
		The second and third rows show, respectively, the data residuals and the convergence curves using the convolutional equivalent layer (our method).
		The results shown in these three rows were obtained by using the simulated airborne magnetic surveys shown in Figure \ref{fig:synthetic_data_comparison_v2}, i.e.:
		(a) A regular grid of $100 \times 50$ observation points in the $x-$ and $y-$directions and a flat observation surface at $z= -900$ m. 
		An irregular grid with uncertainties of (b) $20\%$ and (c) $30\%$   in the $x-$ and $y-$coordinates and a flat observation surface at $z= - 900$ m . 
		A regular grid  of $100 \times 50$ observation points in the $x-$ and $y-$coordinates and an undulating observation surface with uncertainties of (d) $5\%$ and  (e) $10\%$.
		The black lines represent the horizontal projection of the sources
		.}
	\label{fig:synthetic_residuals_convergence_comparison_v2}
}


\plot{synthetic_upward_residuals_comparison_v2}{width=\textwidth}{
	{Synthetic tests: the data residuals of the upward-continued total-field anomalies (third row 
		in Figure \ref{fig:synthetic_data_comparison_v2}).
		The data residuals of the upward-continued total-field anomalies are defined as the difference between 
		the noise-free total-field anomaly produced by the synthetic sources at $z = -1,300$ m 
		(third  row in Figure \ref{fig:synthetic_data_comparison_v2}) and
		the predicted total-field anomaly at $z = -1,300$ m  obtained by using three methods:
		the classical method (first row); the convolutional equivalent layer (second row); and 
		the classic approach in the Fourier domain (third row).
		The results shown in these three rows were obtained by using the simulated airborne magnetic surveys shown in Figure \ref{fig:synthetic_data_comparison_v2}, i.e.:
		(a) A regular grid of $100 \times 50$ observation points in the $x-$ and $y-$directions and a flat observation surface at $z= -900$ m. 
		An irregular grid with uncertainties of (b) $20\%$ and (c) $30\%$   in the $x-$ and $y-$coordinates and a flat observation surface at $z= - 900$ m . 
		A regular grid  of $100 \times 50$ observation points in the $x-$ and $y-$coordinates and an undulating observation surface with uncertainties of (d) $5\%$ and  (e) $10\%$.
		The black lines represent the horizontal projection of the sources
		.}
	\label{fig:synthetic_upward_residuals_comparison_v2}
}

\plot{synthetic_zrtp_residuals_comparison_v2}{width=\textwidth}{
	{Synthetic tests: the data residuals of the reduced to pole total-field anomalies (fourth row 
		in Figure \ref{fig:synthetic_data_comparison_v2}).
		The data residuals of the reduced to pole total-field anomalies are defined as the difference between 
		the noise-free total-field anomaly produced by the synthetic sources at the pole (fourth  row in Figure \ref{fig:synthetic_data_comparison_v2}) and the predicted total-field anomaly obtained by using three methods:
		the classical method (first row); the convolutional equivalent layer (second row); and 
		the classic approach in the Fourier domain (third row).
		The results shown in these three rows were obtained by using the simulated airborne magnetic surveys shown in Figure \ref{fig:synthetic_data_comparison_v2}, i.e.:
		(a) A regular grid of $100 \times 50$ observation points in the $x-$ and $y-$directions and a flat observation surface at $z= -900$ m. 
		An irregular grid with uncertainties of (b) $20\%$ and (c) $30\%$   in the $x-$ and $y-$coordinates and a flat observation surface at $z= - 900$ m . 
		A regular grid  of $100 \times 50$ observation points in the $x-$ and $y-$coordinates and an undulating observation surface with uncertainties of (d) $5\%$ and  (e) $10\%$.
		The black lines represent the horizontal projection of the sources
		.}
	\label{fig:synthetic_zrtp_residuals_comparison_v2}
}

%% Field Data

\plot{carajas_residuals_comparison_edit}{width=\textwidth}{
	{(a) Observed total-field anomaly over the Carajás Province, northen Brazil. The aeromagnetic survey was flown in $131$ north-south flight lines at an average altitude of $900$ m, totaling $N = 6,081,345$ observation points.
		(b) Data residuals, defined as the difference between the regular interpolated grid data (not shown) and the predicted data (not shown), with mean of $0.0762$ nT and standard deviation of $0.4886$ nT. 
		(c) Data residuals, defined as the difference between the irregular decimated grid data (not shown) and the predicted data (not shown), with mean of $ 0.0717$ nT and standard deviation of $0.3144$ nT.
		(d) Convergence curve using our method to the decimated irregular grid of the real data of Carajás Province, Brazil.}
	\label{fig:carajas_residuals_comparison}
}

\plot{carajas_upward_comparison}{width=13cm}{
	{Upward continuations of real data of Carajás Province, Brazil at altitude of 5,000 m by using:
		(a) the convolutional equivalent layer (our method) and (b) the classical Fourier method.}
	\label{fig:carajas_upward_comparison}
}

%\plot{carajas_real_data_mag}{width=\textwidth}{
%	{Observed total-field anomaly over the Carajás Province, northen Brazil. The aeromagnetic survey was flown in $131$ north-south flight lines at an average altitude of $900$ m, totaling $N = 6,081,345$ observation points.}
%	\label{fig:carajas_real_data_mag}
%}
%
%\plot{carajas_real_data_decimated_gridline}{width=8cm}{
%	{Observed total-field anomalies over the Carajás Province, northen Brazil, considering: 
%	(a) a regular grid ($10,000 \times 131$, totaling $N = 1,310,000$ observation points) of interpolated values from the original irregularly sampled data (Figure \ref{fig:carajas_real_data_mag}); and  	
%	(b) an irregular grid ($10,000 \times 131$, totaling $N = 1,310,000$ observation points) of decimated values from the the original irregularly sampled data (Figure \ref{fig:carajas_real_data_mag}).}
%	\label{fig:carajas_real_data_decimated_gridline}
%}
%
%\plot{carajas_tf_predicted_gridline}{width=8cm}{
%	{Results using our method applied to the interpolated $10,000 \times 131$ regular grid
%	(Figure \ref{fig:carajas_real_data_decimated_gridline}a) - 
%	(a) Predicted data and (b) Data residuals, defined as the difference between the observed 
%	(Figure \ref{fig:carajas_real_data_decimated_gridline}a) and the predicted data (panel a), with mean of $0.0762$ nT and standard deviation of $0.4886$ nT.}
%	\label{fig:carajas_gz_predito_mag_gridline}
%}
%
%\plot{carajas_tf_predicted_decimated}{width=8cm}{
%	{Results using our method applied to the decimated  $10,000 \times 131$ irregular grid
%	(Figure \ref{fig:carajas_real_data_decimated_gridline}b) - 	
%	(a) Predicted data and (b) Data residuals, defined as the difference between the observed  (Figure \ref{fig:carajas_real_data_decimated_gridline}b) and the predicted data (panel a), with mean of $ 0.0717$ nT and standard deviation of $0.3144$ nT.}
%	\label{fig:carajas_gz_predito_mag_decimated}
%}
%
%\plot{convergence_carajas_mag_decimated}{width=\textwidth}{
%	{Convergence curve using our method to the decimated irregular grid of the real data of Carajás Province, Brazil.}
%	\label{fig:convergence_carajas_mag_decimated}
%}
%
%\plot{up5000_carajas_mag_decimated}{width=\textwidth}{
%	{Upward continuation transformation of real data of Carajás Province, Brazil at $5, \,000$ m after $2.64$ seconds.}
%	\label{fig:up5000_carajas_decimated_mag}
%}

\section{Tables}

\tabl{RAM-usage}{This table shows the RAM memory usage (in Megabytes) for storing the whole 
	$N \times N$ matrix $\mathbf{A}$ (equation \ref{eq:A_expand}), the first columns of the BCCB
	matrices $\mathbf{C}_{\boldsymbol{\alpha\beta}}$ (equation \ref{eq:w_alpha_beta}) (both need 8 
	bytes per element) and the matrix $\mathbf{L}$ (equation \ref{eq:L}) (16 bytes per element).
	\label{tab:RAM-usage}}
{
	\begin{center}
		\begin{tabular}[]{|l|c|c|c|}
			\hline
			\textbf{$N$} & \textbf{$\mathbf{A}$} & \textbf{First columns of matrices $\mathbf{C}_{\boldsymbol{\alpha\beta}}$} & \textbf{$\mathbf{L}$}\\
			\hline 
			$100$ & 0.0763 & 0.0183 & 0.00610\\
			\hline
			$400$ & 1.22 & 0.0744 & 0.0248\\
			\hline
			$2,500$ & 48 & 0.458 & 0.1528\\
			\hline
			$10,000$ & 763 & 1.831 & 0.6104\\
			\hline
			$40,000$ & 12,207 & 7.32 & 2.4416 \\
			\hline
			$250,000$ & 476,837 & 45.768 & 15.3 \\
			\hline
			$500,000$ & 1,907,349 & 91.56 & 30.518 \\
			\hline
			$1,000,000$ & 7,629,395 & 183.096 & 61.035 \\
			\hline
		\end{tabular}
	\end{center} 
}
\clearpage

\append{BTTB matrix-vector product}

This appendix follows a similar approach to that presented by \citet{takahashi2020convolutional}
to efficiently compute the product of the sensitivity matrix
$\mathbf{A}$ (equations \ref{eq:A_expand}) and a generic vector $\mathbf{b}$. 
Let this product be represented by
\begin{equation}
	\mathbf{t} = 
	\mathbf{A} \, \mathbf{b} \: ,
	\label{eq:t}
\end{equation}
where 
\begin{equation}
	\mathbf{t} = \mathbf{t}_{\boldsymbol{xx}} + \mathbf{t}_{\boldsymbol{xy}} + \mathbf{t}_{\boldsymbol{xz}} +
	\mathbf{t}_{\boldsymbol{yy}} + \mathbf{t}_{\boldsymbol{yz}} + \mathbf{t}_{\boldsymbol{zz}}
	\label{eq:t-components}
\end{equation}
and
\begin{equation}
	\mathbf{t}_{\boldsymbol{\alpha\beta}} = 
	\mathbf{A_{\boldsymbol{\alpha\beta}}} \, \mathbf{b} \: .
	\label{eq:t-alpha-beta}
\end{equation}
Let us also consider that vectors
\begin{equation}
	\mathbf{t}_{\boldsymbol{\alpha\beta}} = \begin{bmatrix}
		\mathbf{t}^{0}_{\boldsymbol{\alpha\beta}} \\
		\vdots \\
		\mathbf{t}^{Q-1}_{\boldsymbol{\alpha\beta}} \\
	\end{bmatrix}_{N \times 1}
	\label{eq:t-alpha-beta-partitioned}
\end{equation} 
and
\begin{equation}
	\mathbf{b} = \begin{bmatrix}
		\mathbf{b}^{0} \\
		\vdots \\
		\mathbf{b}^{Q-1} \\
	\end{bmatrix}_{N \times 1}
	\label{eq:b-partitioned}
\end{equation}
are composed of $P \times 1$ vectors $\mathbf{t}^{q}_{\boldsymbol{\alpha\beta}}$ and $\mathbf{b}^{q}$,
respectively, where $q$ is the block index (equations \ref{eq:q-x-oriented} and \ref{eq:q-y-oriented}). 
From equation \ref{eq:t-alpha-beta}, we obtain an auxiliary matrix-vector product given by
\begin{equation}
	\mathbf{w}_{\boldsymbol{\alpha\beta}} = \mathbf{C}_{\boldsymbol{\alpha\beta}} \, \mathbf{v} \: ,
	\label{eq:w_alpha_beta}
\end{equation}
where $\mathbf{C}_{\boldsymbol{\alpha\beta}}$ is a $4N \times 4N$ 
block circulant matrix with circulant blocks (BCCB) \citep[e.g., ][ p. 184]{davis1979},
\begin{equation}
	\mathbf{w}_{\boldsymbol{\alpha\beta}} = \begin{bmatrix}
		\mathbf{w}_{\boldsymbol{\alpha\beta}}^{0} \\
		\vdots \\
		\mathbf{w}_{\boldsymbol{\alpha\beta}}^{Q - 1} \\
		\mathbf{0}_{2N \times 1}
	\end{bmatrix}_{4N \times 1} \quad ,
	\label{eq:w_alpha_beta_partitioned}
\end{equation}
\begin{equation}
	\mathbf{w}_{\boldsymbol{\alpha\beta}}^{q} = \begin{bmatrix}
		\mathbf{t}^{q}_{\boldsymbol{\alpha\beta}} \\
		\mathbf{0}_{P \times 1}
	\end{bmatrix}_{2P \times 1}
	\label{eq:wq-vector} \quad ,
\end{equation}
\begin{equation}
	\mathbf{v} = \begin{bmatrix}
		\mathbf{v}^{0} \\
		\vdots \\
		\mathbf{v}^{Q - 1} \\
		\mathbf{0}_{2N \times 1}
	\end{bmatrix}_{4N \times 1}
	\label{eq:v-vector}
\end{equation}
and
\begin{equation}
	\mathbf{v}^{q} = \begin{bmatrix}
		\mathbf{b}^{q} \\
		\mathbf{0}_{P \times 1}
	\end{bmatrix}_{2P \times 1} \: ,
	\label{eq:vq-vector} 
\end{equation}
with $\mathbf{0}_{2N \times 1}$ and $\mathbf{0}_{P \times 1}$ being vectors of zeros.
The key point here is that the auxiliary matrix-vector product
(equation \ref{eq:w_alpha_beta}) represents a 2D discrete convolution and can be 
efficiently computed by using the 2D Fast Fourier Transform (2D FFT). 

The BCCB matrix $\mathbf{C}_{\boldsymbol{\alpha\beta}}$ (equation \ref{eq:w_alpha_beta})
is formed by $2Q \times 2Q$ blocks, 
where each block $\mathbf{C}_{\boldsymbol{\alpha\beta}}^{q}$ is a $2P \times 2P$ circulant matrix.
The entire BCCB matrix $\mathbf{C}_{\boldsymbol{\alpha\beta}}$ is defined by properly
downshifting its first block column
\begin{equation}
	\left[ \mathbf{C}_{\boldsymbol{\alpha\beta}} \right]_{(0)} = \begin{bmatrix}
		\mathbf{C}_{\boldsymbol{\alpha\beta}}^{0} \\
		\vdots \\
		\mathbf{C}_{\boldsymbol{\alpha\beta}}^{Q-1} \\
		\mathbf{0}_{2P \times 2P} \\
		\mathbf{C}_{\boldsymbol{\alpha\beta}}^{-Q+1} \\
		\vdots \\
		\mathbf{C}_{\boldsymbol{\alpha\beta}}^{-1}
	\end{bmatrix}_{4N \times 2P} \: ,
	\label{eq:C_alpha_beta_first_block_column}
\end{equation}
where $\mathbf{0}_{2P \times 2P}$ is a matrix of zeros. Similarly, 
each block $\mathbf{C}_{\boldsymbol{\alpha\beta}}^{q}$, $q = -Q+1, \dots, Q-1$,
is obtained by properly downshifting its first column
\begin{equation}
	\mathbf{c}_{\boldsymbol{\alpha\beta}}^{q} = \begin{bmatrix}
		a^{\alpha\beta}_{q0} \\
		\vdots \\
		a^{\alpha\beta}_{q(P-1)} \\
		0 \\
		a^{\alpha\beta}_{q(-P+1)} \\
		\vdots \\
		a^{\alpha\beta}_{q(-1)}
	\end{bmatrix}_{2P \times 1} \: ,
	\label{eq:cq_alpha_beta}
\end{equation}
where $a^{\alpha\beta}_{qp}$, $p = -P+1, \dots, P-1$, are the elements of
matrix component $\mathbf{A_{\boldsymbol{\alpha\beta}}}$ described in terms of
block indices $q$ and $p$ (equations \ref{eq:q-x-oriented}--\ref{eq:p-y-oriented}). 
The BCCB matrix $\mathbf{C}_{\boldsymbol{\alpha\beta}}$ is diagonalized by
$\mathbf{F}_{2Q} \otimes \mathbf{F}_{2P}$, where ``$\otimes$" denotes the Kronecker product
\citep[e.g.,][ p. 242]{horn_johnson1991} and $\mathbf{F}_{2Q}$ and $\mathbf{F}_{2P}$ are 
the $2Q \times 2Q$ and $2P \times 2P$ unitary DFT matrices \citep[][ p. 31]{davis1979}.
Due to this property, the auxiliary matrix-vector product (equation \ref{eq:w_alpha_beta}) 
can be computed as follows \citep{takahashi2020convolutional}:
\begin{equation}
	\mathbf{F}_{2Q}^{\ast} \left[ 
	\mathbf{L}_{\boldsymbol{\alpha\beta}} \circ \left(\mathbf{F}_{2Q} \, \mathbf{V} \, \mathbf{F}_{2P} \right) 
	\right] \mathbf{F}_{2P}^{\ast} = \mathbf{W}_{\boldsymbol{\alpha\beta}} \: ,
	\label{eq:2d-discrete-convolution}
\end{equation}
where ``$\circ$" denotes the Hadamard (element-wise) product \citep[e.g.,][ p. 298]{horn_johnson1991},
``$\ast$" denotes the complex conjugate, 
$\mathbf{W}_{\boldsymbol{\alpha\beta}}$ and $\mathbf{V}$ are $2Q \times 2P$ matrices obtained
by rearranging, respectively, vectors $\mathbf{w}_{\boldsymbol{\alpha\beta}}$ 
(equation \ref{eq:w_alpha_beta_partitioned}) and $\mathbf{v}$ (equation \ref{eq:v-vector})
along their rows and $\mathbf{L}_{\boldsymbol{\alpha\beta}}$ is a $2Q \times 2P$ matrix
given by
\begin{equation}
	\mathbf{L}_{\boldsymbol{\alpha\beta}} = \sqrt{4QP} \; 
	\mathbf{F}_{2Q} \, \mathbf{G}_{\boldsymbol{\alpha\beta}} \, \mathbf{F}_{2P} \: ,
	\label{eq:L_alpha_beta}
\end{equation}
with
\begin{equation}
	\mathbf{G}_{\boldsymbol{\alpha\beta}} = \begin{bmatrix}
		\left( \mathbf{c}_{\boldsymbol{\alpha\beta}}^{0} \right)^{\top} \\
		\vdots \\
		\left( \mathbf{c}_{\boldsymbol{\alpha\beta}}^{Q-1} \right)^{\top} \\
		\mathbf{0}_{1 \times 2P} \\
		\left( \mathbf{c}_{\boldsymbol{\alpha\beta}}^{-Q+1} \right)^{\top} \\
		\vdots \\
		\left( \mathbf{c}_{\boldsymbol{\alpha\beta}}^{-1} \right)^{\top}
	\end{bmatrix}_{2Q \times 2P} \: ,
	\label{eq:G_alpha_beta}
\end{equation}
defined by the first columns $\mathbf{c}_{\boldsymbol{\alpha\beta}}^{q}$ 
(equation \ref{eq:cq_alpha_beta}), $q = -Q+1, \dots, Q-1$, of all circulant blocks
$\mathbf{C}_{\boldsymbol{\alpha\beta}}^{q}$ (equation \ref{eq:C_alpha_beta_first_block_column}).
Hence, the whole BCCB matrix $\mathbf{C}_{\boldsymbol{\alpha\beta}}$ does not have to be
formed, but only its first column. 
Besides, the symmetries defined by equations 
\ref{eq:Axx_symmetry}--\ref{eq:Ayz_q_internal_block_symmetry_y_oriented} imply
that all elements of $\mathbf{G}_{\boldsymbol{\alpha\beta}}$ can be obtained by 
using only the first column of $\mathbf{A_{\boldsymbol{\alpha\beta}}}$. Consequently, 
the whole matrices $\mathbf{A_{\boldsymbol{\alpha\beta}}}$ do not have to be formed
as well, but only their first columns.

It is important noting that the left side of equation \ref{eq:2d-discrete-convolution} represents 
the 2D Inverse Discrete Fourier Transform (2D IDFT) of the term in brackets. This term, in turn,
represents the Hadamard product of $\mathbf{L}_{\boldsymbol{\alpha\beta}}$ 
(equation \ref{eq:L_alpha_beta}) and the 2D Discrete Fourier Transform (2D DFT) of $\mathbf{V}$.
Similarly, equation \ref{eq:L_alpha_beta} shows that $\mathbf{L}_{\boldsymbol{\alpha\beta}}$ is
obtained by computing the 2D DFT of matrix $\mathbf{G}_{\boldsymbol{\alpha\beta}}$
(equation \ref{eq:G_alpha_beta}). Hence, equations \ref{eq:2d-discrete-convolution} and
\ref{eq:L_alpha_beta} can be efficiently computed by using the 2D FFT. After that, the elements of
vector $\mathbf{t}_{\boldsymbol{\alpha\beta}}$ (equation \ref{eq:t-alpha-beta}) can be
retrieved from the first quadrant of matrix $\mathbf{W}_{\boldsymbol{\alpha\beta}}$
(equation \ref{eq:2d-discrete-convolution}). By combining the results obtained for all
components $\boldsymbol{\alpha\beta}$, 
$\boldsymbol{\alpha}, \boldsymbol{\beta} = \boldsymbol{x}, \boldsymbol{y}, \boldsymbol{z}$,
we can show that
\begin{equation}
	\mathbf{F}_{2Q}^{\ast} \left[ 
	\mathbf{L} \circ \left(\mathbf{F}_{2Q} \, \mathbf{V} \, \mathbf{F}_{2P} \right) 
	\right] \mathbf{F}_{2P}^{\ast} = \mathbf{W} \: ,
	\label{eq:2d-discrete-convolution-complete}
\end{equation}
where
\begin{equation}
	\mathbf{W} = \mathbf{W}_{\boldsymbol{xx}} + \mathbf{W}_{\boldsymbol{xy}} + \mathbf{W}_{\boldsymbol{xz}} + \mathbf{W}_{\boldsymbol{yy}} + \mathbf{W}_{\boldsymbol{yz}} + \mathbf{W}_{\boldsymbol{zz}}
	\label{eq:W}
\end{equation}
and
\begin{equation}
	\mathbf{L} = \mathbf{L}_{\boldsymbol{xx}} + \mathbf{L}_{\boldsymbol{xy}} + \mathbf{L}_{\boldsymbol{xz}} + \mathbf{L}_{\boldsymbol{yy}} + \mathbf{L}_{\boldsymbol{yz}} + \mathbf{L}_{\boldsymbol{zz}} \: ,
	\label{eq:L}
\end{equation}
with $\mathbf{L}_{\boldsymbol{\alpha\beta}}$ defined by equation \ref{eq:L_alpha_beta}.
Then, the elements of $\mathbf{t}$ (equation \ref{eq:t}) are obtained from the first
quadrant of $\mathbf{W}$ (equations \ref{eq:2d-discrete-convolution-complete} and
\ref{eq:W}).

Finally, it can be shown that the product 
\begin{equation}
	\mathbf{t} = 
	\mathbf{A}^{\top} \mathbf{b}
	\label{eq:t_AT}
\end{equation}
can be computed by using equation \ref{eq:2d-discrete-convolution-complete}.
The difference is that, in this case, matrices $\mathbf{G}_{\boldsymbol{\alpha\beta}}$
(equation \ref{eq:G_alpha_beta}) are defined by using the new vectors
\begin{equation}
	\mathbf{c}_{\boldsymbol{\alpha\beta}}^{q} = \begin{bmatrix}
		a^{\alpha\beta}_{q0} \\
		\vdots \\
		a^{\alpha\beta}_{q(-P+1)} \\
		0 \\
		a^{\alpha\beta}_{q(P-1)} \\
		\vdots \\
		a^{\alpha\beta}_{q1}
	\end{bmatrix}_{2P \times 1} \: .
	\label{eq:cq_alpha_beta_trnasposed}
\end{equation}

\clearpage

\bibliographystyle{seg}  % style file is seg.bst
\bibliography{references}

\clearpage


\end{document}
